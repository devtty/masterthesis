%%
%% Beuth Hochschule für Technik --  Abschlussarbeit
%%
%% Anhang
%%
%%%%%%%%%%%%%%%%%%%%%%%%%%%%%%%%%%%%%%%%%%%%%%%%%%%%%%%%%%%%%%%%%%%%%


\chapter{Angehängtes: Die Dateien des Pakets}

\subsection*{Tabellen}

\begin{table}[h]
  \centering
  \resizebox{\textwidth}{!}{
    \begin{tabular}{|c | c | c | c |} 
 \hline
      \textbf{WAF Name} & \textbf{Hersteller} & \textbf{WAF Name} & \textbf{Hersteller} \\
      \hline
      ACE XML Gateway & Cisco & FirePass & F5 Networks\\
      Airlock & Phion/Ergon & FortiWeb & Fortinet\\
      Alert Logic & Alert Logic & GoDaddy Website Protection & GoDaddy\\
      AliYunDun & Alibaba Cloud Computing & Huawei Cloud Firewall & Huawei\\
      AnYu & AnYu Technologies & HyperGuard & Art of Defense\\
      AppWall & Radware & Imunify360 & CloudLinux\\
      Armor Defense & Armor & Incapsula & Imperva Inc.\\
      ArvanCloud & ArvanCloud & ISA Server & Microsoft\\
      ASP.NET Generic & Microsoft & Janusec Application Gateway & Janusec\\
      ASPA Firewall & ASPA Engineering Co. & Kona SiteDefender & Akamai\\
      Astra & Czar Securities & KS-WAF & KnownSec\\
      AWS Elastic Load Balancer & Amazon & LimeLight CDN & LimeLight\\
      AzionCDN & AzionCDN & Oracle Cloud & Oracle\\
      Azure Front Door & Microsoft & ModSecurity & SpiderLabs\\
      Barracuda & Barracuda Networks & NetContinuum & Barracuda Networks\\
      Bekchy & Faydata Technologies Inc. & NetScaler AppFirewall & Citrix Systems\\
      Beluga CDN & Beluga & NullDDoS Protection & NullDDoS\\
      BinarySec & BinarySec & PentaWAF  & Global Network Services\\
      BitNinja & BitNinja &   PT Application Firewall & Positive Technologies\\
      BlockDoS & BlockDoS &   Profense  & ArmorLogic\\
      Bluedon & Bluedon IST &   Qcloud   & Tencent Cloud\\
      CacheWall & Varnish &   RequestValidationMode   & Microsoft\\
      CacheFly CDN & CacheFly &   Sabre Firewall     & Sabre\\
      Comodo cWatch & Comodo CyberSecurity &   eEye SecureIIS & BeyondTrust\\
      CdnNS Application Gateway & CdnNs/WdidcNet &   SecuPress WP Security  & SecuPress\\
      ChinaCache Load Balancer & ChinaCache &   SecureSphere  & Imperva Inc.\\
      Chuang Yu Shield & Yunaq &   SEnginx       & Neusoft\\
      Cloudbric & Penta Security &   SiteGuard  & Sakura Inc.\\
      Cloudflare & Cloudflare Inc. &   SonicWall   & Dell\\
      Cloudfloor & Cloudfloor DNS &   UTM Web Protection              & Sophos\\
      Cloudfront & Amazon &   SquidProxy IDS   & SquidProxy\\
      CrawlProtect & Jean-Denis Brun &   Tencent Cloud Firewall   & Tencent Technologies\\
      DataPower & IBM &   Teros   & Citrix Systems\\
      DenyALL & Rohde+Schwarz CyberSecurity &  Trafficshield  & F5 Networks\\
      Distil & Distil Networks &   Varnish        & OWASP\\
      DotDefender & Applicure Technologies &   WebSEAL  & IBM\\
      DynamicWeb Injection Check & DynamicWeb &   XLabs Security WAF   & XLabs\\
      BIG-IP AppSec Manager & F5 Networks &   ZScaler & Accenture\\
      BIG-IP AP Manager & F5 Networks & & \\
      Fastly & Fastly & & \\
      \hline
      
      
\end{tabular}}
\caption{Auszug WAFw00f Plugins(aus https://github.com/EnableSecurity/wafw00f)}
\label{tab:my_label}
\end{table}

\subsection*{Beispiele}

\lstset{language=bash,
 	basicstyle=\ttfamily\color{black}\small,
 	keywordstyle=\bfseries\color{bhtBlue},
 	identifierstyle=\color{black}, 
 	commentstyle=\color{gray}\textsl
      }
      \begin{figure}
        \caption{Auszug aus REQUEST-942-APPLICATION-ATTACK-SQLI.conf}
        \label{fig:saprule}
\begin{lstlisting}
# This is a stricter sibling of rule 942430.
#
# This rule is also triggered by the following exploit(s):
# [ SAP CRM Java vulnerability CVE-2018-2380 - Exploit tested: https://www.exploit-db.com/exploits/44292 ]
#

SecRule ARGS_NAMES|ARGS|XML:/* "@rx ((?:[~!@#\$%\^&\*\(\)\-\+=\{\}\[\]\|:;\"'´’‘`<>][^~!@#\$%\^&\*\(\)\-\+=\{\}\[\]\|:;\"'´’‘`<>]*?){2})" \
    "id:942432,\
    phase:2,\
    block,\
    capture,\
    t:none,t:urlDecodeUni,\
    msg:'Restricted SQL Character Anomaly Detection (args): # of special characters exceeded (2)',\
    logdata:'Matched Data: %{TX.1} found within %{MATCHED_VAR_NAME}: %{MATCHED_VAR}',\
    tag:'application-multi',\
    tag:'language-multi',\
    tag:'platform-multi',\
    tag:'attack-sqli',\
    tag:'OWASP_CRS',\
    tag:'capec/1000/152/248/66',\
    tag:'PCI/6.5.2',\
    tag:'paranoia-level/4',\
    ver:'OWASP_CRS/4.0.0-rc1',\
    severity:'WARNING',\
    setvar:'tx.inbound_anomaly_score_pl4=+%{tx.warning_anomaly_score}',\
    setvar:'tx.sql_injection_score=+%{tx.warning_anomaly_score}'"
\end{lstlisting}
      \end{figure}


\subsection*{Stylefile}
Die  Styledatei für diese  Abschlussarbeit ist  \texttt{bhtThesis.sty}, die  in der
Archivdatei vorliegt.  Diese muss von \LaTeX\  auffindbar sein, muss  also in einem
\LaTeX\ bekannten Ordner liegen:
\begin{itemize}
\item Ubuntu-Linux: \verb|$HOME/texmf/tex/latex/bhtThesis/bhtThesis.sty|
\item MikTeX: \verb|c:\localtexmf\tex\latex\bhtThesis/bhtThesis.sty|
\end{itemize}


\subsection*{Beispieldokument}
Dieses  Dokument befindet sich  im Unterordner  \texttt{tryout} des  zip-files. Sie
können diese  Dateien in  einen Ordner kopieren,  in dem Sie  schliesslich arbeiten
werden. Die Dateien sind die folgenden

\begin{itemize}
\item \texttt{abstract\_de.tex} Kurzfassung in deutscher Sprache
\item \texttt{abstract\_en.tex} Kurzfassung in englischer Sprache
\item \texttt{anhang.tex} der Anhang
\item \texttt{bhtThesis.bib} beinhaltet die zu zitierenden Literaturstellen und
  wird von bib\TeX ausgewertet 
\item \texttt{main.pdf} ist die Ausgabendatei mit der Druckvorlage
\item \texttt{main.tex} beinhaltet das Hauptdokument
\item \texttt{makefile} realsiert das automatische mehrfache Übersetzen, hierfür
  muss \texttt{make} auf dem System installiert sein.
\item \texttt{myapalike.bst} beinhaltet die Formatierung für das
  Literaturverzeichnis 
\item \texttt{personalMacros.tex} kann einzelne, persönliche Macros beinhalten, die
  das Schreiben erleichtern
\item \texttt{titelseiten.tex} realisiert alle Seiten bis zum Beginn des ersten
  Abschnittes  

\item Ordner \texttt{pictures}
  \begin{itemize}
  \item \texttt{BHT-Logo-Basis.eps}
  \item \texttt{BHT-Logo-Basis.pdf}
  \end{itemize}

\item Ordner \texttt{kapitel1}
  \begin{itemize}
  \item \texttt{ch1.tex} Quelltext des Kapitel 1
  \item Ordner \texttt{pictures}
    \begin{itemize}
    \item \texttt{schaltbild.pdf}
    \end{itemize}
  \end{itemize}
  
\item Ordner \texttt{kapitel2}
  \begin{itemize}
  \item \texttt{ch2.tex} Quelltext des Kapitel 2
  \item Ordner \texttt{pictures}
    \begin{itemize}
    \item leer
    \end{itemize}
  \end{itemize}  
\end{itemize}

