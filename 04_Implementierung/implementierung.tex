\chapter{Implementierung}

\anno{max. 5 Seiten (10)} %Ref hat 12

% Was lesen wir in diesem Kapitel?
% Warum muss ich das als Gutachter lesen
% Wie verknüpft sich der Inhalt mit dem vorhergehenden Kapitel?
% Welche Implmentierungsentscheidungen? Welche Alternativen? Vor- und Nachteile des eigenen Ansatzes?

% Highlight 1 der Implementierung
Wie bereits in Abschnitt \ref{sec:Probleme} angemerkt sind offene Implementationen von Web Application Firewalls eher die Ausnahme und selbst Umsetzungen der in Abschnitt \ref{sec:relatedwork} erwähnten Produkte sind praktisch nicht auffindbar. In diesem Kapitel gibt es einen kurzen Überblick wie die Vorschläge aus dem vorhergehenden Kapitel dennoch praktisch umgesetzt werden können.

Eine komplette Umsetzung ohne vorherige Basis wäre zeitlich (von einer einzelnen Person) nicht umsetzbar gewesen, daher sollte ein vorhandenes Open Source Produkt entsprechend erweitert werden. Bei der Auswahl wurden zuerst die zwei (leichtgewichtigen) Lösungen \emph{OWASP ESAPIFilter}\footnote{\url{https://owasp.org/www-project-enterprise-security-api/} abgerufen am 05.07.2023} und {SpringSecurity HttpFirewall}\footnote{\url{https://docs.spring.io/spring-security/reference/servlet/exploits/firewall.html} abgerufen am 05.07.2023} betrachtet. Beide Produkte verstehen sich als Ausgangsbasis für die Entwicklung sicherheitsrelevanter HTTP-(Servlet)-Filter auf Anwendungsebene, sämtliche Funktionalität muss jedoch vom Nutzer erst implementiert werden. Bei der Nutzung wäre ein erhöhter Arbeitsaufwand absehbar. Diogo Sampaio und Jorge Bernardino verglichen 2017 verfügbare WAFs die ebenfalls als Erweiterungskandidaten in Frage kamen (siehe Tabelle \ref{tab:my_vergos}).

% Tabelle ggf. in Anhang verschieben

\begin{table}[h]  \centering
  \begin{tabular}{lccc} 
    \toprule
    & \textbf{ModSecurity} & \textbf{WebCastellum} & \textbf{IronBee} \\ [0.5ex] 
    \midrule
    Simple filtering & Yes & Yes & Yes \\ 
    Regular expression based filtering & Yes &  & Yes \\
    Auditing & Yes &  & Yes \\
    Null byte attack prevention & Yes & Yes &  \\
    URL Encryption & Yes & Yes &  \\ [1ex] 
    Stateful Attack Detection & & Yes & Yes \\
    \bottomrule
  \end{tabular}
  \caption{Vergleich Open Source Web Application Firewalls (aus \cite{Sampaio2017}) }
  \label{tab:my_vergos}
\end{table}

\emph{ModSecurity} und \emph{IronBee}s Stärken liegen in der regelbasierten Auswertung (\emph{regular expression based filtering}). Beide werden als Reverse-Proxy vor der eigentlichen Anwendung eingesetzt und sind keine HTTP-Servlet-Filter. Dies hat den Nachteil das kein Zugriff auf die eigentlichen Anwendungsdaten besteht und nur anhand des HTTP-Datenverkehrs gefiltert werden kann. \emph{WebCastellum} scheint hier Gutes aus beiden Welten bereit zustellen, zum einen (als Servlet-Filter) direkten Zugriff auf die Anwendungsdaten anzubieten, als auch regelbasiertes Filtern \emph{out-of-the-box}. 
Mit Hinsicht auf einen späteren produktiven Einsatz wurde daher die Software \emph{WebCastellum} als Grundlage für die Implementierung der Firewall-Funktionalität gewählt.

\section{Vorarbeiten}

Leider haben WebCastellum und der CSIC2010 Datensatz eine Gemeinsamkeit - das Alter. Die derzeit aktuelle offizielle Version 1.8.3 von WebCastellum erschien bereits im Jahr 2009 und eine Weiterentwicklung fand seit dem nicht statt. Dennoch befindet sich das Produkt weiter in Verwendung und wird wohl auch in neuen Anwendungen genutzt. Zumindest lässt die aktuelle Downloadstatistik (siehe \ref{fig:downloadwc}) dieses vermuten. Insgesamt wurde das Binärpaket von WebCastellum in der Version 1.8.3 seit 2009 mehr als 8500 mal heruntergeladen (Stand vom 05. Juli 2023) - nicht berücksichtigt sind alle Vorgängerversionen und die Downloads die über Repository-Dienste, wie z.B. Maven Central, stattfanden.

\begin{figure}[h]
  \centering
  \begin{gnuplot}[terminal=png,scale=.7]
    set datafile separator ','
    set xdata time
    set timefmt "%Y-%m"
    set xrange ["2009-01":"2023-09"]
    set format x "%b %y"
    set key autotitle columnhead
    plot '04_Implementierung/downloads.csv' using 1:2 with boxes fs solid 1.0 fc 'steelblue'
  \end{gnuplot}
  \caption{Download-Statistik SourceForge WebCastellum 1.8.3 binary}
  \label{fig:downloadwc}
\end{figure}

Vor den eigentlichen Arbeiten waren also noch ein paar kleinere Vorarbeiten notwendig. In einem ersten Schritt wurde der Quellcode aus dem SourceForge-Repository heruntergeladen und in ein eigenes git-Repository\footnote{\url{https://github.com/devtty/webcastellum/} abgerufen am 05.07.2023} übertragen. 

\subsection{Version 1.8.4. - Review des Quellcode und IT-Sicherheit}
Während die Binär-Distribution mit der Versionsnummer 1.8.3 verteilt wird, sind die Quellen bereits mit der nächsten Minor-Version versehen. Diese lassen sich auch fast problemlos kompilieren und zu einem verwendungsfähigen Paket zusammenbauen. Das Projekt hat nur sehr wenig Abhängigkeiten zu weiteren Bibliotheken und diese (laut durchgeführtem OWASP dependency-check) keine bekannten Schwachstellen. Etwas problematisch ist die Abhängigkeit zu älteren Bibliotheken (\emph{javax.jms} und \emph{javax.mail}), da diese nicht mehr (direkt) über den Standard-Build-Mechanismus verfügbar sind, sich aber über Umwege auftreiben lassen.

Im weiteren Verlauf wurden die Quellen noch an Githubs eigenes CI/CD-System angebunden und automatisiert einer Codeanalyse auf der Sonarcloud-Plattform \footnote{\url{https://sonarcloud.io/project/overview?id=devtty_webcastellum} abgerufen am 05.07.2023} unterzogen. Insgesamt fielen dabei im Quelltext sechs direkte Schwachstellen und 58 potentielle Schwachstellen (siehe Abbildung~\ref{fig:my_sonar1}) auf. Bei den direkten Schwachstellen handelt es sich um:

\begin{itemize}
    \item 3 x Verwendung schwacher Kryptographie (hier AES; Cryptographic Failures / Security Misconfiguration )
    \item 2 x Ausgabe der Session-ID in Log-Dateien (Insecure Design/Broken Authentication)
    \item 1 x XML External Entity Injection Attack (Security Misconfiguration / XML External Entities (XXE))
\end{itemize}

Wobei die XXE-Attacke von Sonar als Blocker eingestuft wird. Bei den 58 Security Hotspots handelt es sich meistens um direkte Ausgaben im Fehlerfall (\verb=printStackTrace()=) mit der Bitte zur Überprüfung, dass diese im Produktivbetrieb nicht ausgegeben werden. 

Inklusive Code Smells bemängelte Test per SonarCloud-Dienst den Quellcode in insgesamt 2657 Fällen und berechnet die angehäuften technischen Schulden auf zirka 52 Personen-Tage Nacharbeit. 

\begin{figure}
    \centering
    \includegraphics[width=\textwidth]{first_sonar2.jpg}
    \caption{SonarCloud erste Sichtung}
    \label{fig:my_sonar1}
\end{figure}

\begin{figure}
    \centering
    \includegraphics[width=\textwidth]{first_sonar_sql.jpg}
    \caption{SonarCloud SQL-Injection}
    \label{fig:my_sonar2}
\end{figure}



\subsection{ToDos, Unit-Tests,  etc.}

Interessant war ebenfalls die Suche nach \verb=TODO=-Markierungen im Quelltext. Insgesamt 243 mal traten entsprechende Kommentare im Quelltext auf und markierten Stellen an denen die Entwickler Probleme und Verbesserungen vermuteten bzw. Aufgaben für die nächste Version notierten. Diese Notizen reichten von einfachen Aufgabenstellungen, wie \glqq\emph{add proper logging}\grqq, über Refactoring-Ideen bis hin zu Anmerkungen deren Umsetzung die Funktionsweise des Filters beeinflussen würden. 

Einige der Todo-Anmerkungen ließen auf eine geplante Überführung des Quellcodes von Java in der Version 1.4 auf Java 5 schließen. Häufig fanden sich auch auskommentierte dementsprechende Fragmente und, insbesondere bei der Verwendung listenähnlicher Objekte, fallen die in Java 5 eingeführten \emph{Generics} (s. Abbildung~\ref{fig:my_l2}) auf.

\begin{figure}[h]
  \begin{small}
    \begin{lstlisting}[language=java]
    // TODO: in Utility-Klasse packen (ServerUtils)
    private static void concatenateParameterMaps(
      final Map/*<String,String[]>*/ parameters, 
      final Map/*<String,String[]>*/ parameterMapToAdd) {
    \end{lstlisting}
  \end{small}
  \caption{Auszug aus WebCastellumFilter als Beispiel für ToDos und auskommentierte Generics}
  \label{fig:my_l2}
\end{figure}
  
Dabei sollte berücksichtigt werden, dass diese Kommentare aus dem Jahr 2009 stammen und auch die Programmiersprache Java in den letzten Jahren deutliche Fortschritte gemacht hat. Die Sprache erlaubt mittlerweile auch Konstrukte wie Lambda-Expressions (Java 8) und Typinferenz (Java 10). Ein Refactoring des Quelltextes diesbezüglich wäre sicher lohnenswert.

Tests, im Sinne von Unit-Tests z.B. im Rahmen eines etablierten Testframeworks, lassen sich in den Quellen der Version 1.8.3 nicht finden. Für eine Software, die zur Sicherheit von Anwendungen beitragen sollte, eher ungewöhnlich. Stattdessen existieren in einigen Klassen \verb=main=-Methoden mit entsprechender Absicht. Die Klasse \verb=LargeFormPostRequestTester= dient beispielsweise ausschließlich Testzwecken. \verb=ResponseUtils= behinhaltet gleich fünf auskommentierte \verb=main=-Methoden zum Testen verschiedener Funktionalitäten inklusive der Anmerkung \emph{for local testing only}. 


\subsection{Fehlerbereinigung}
\label{sec:centralconf}
Die minimale (persönliche) Anforderung an das Ziel der Vorarbeiten war den seit 2009 nicht bearbeiteten Quelltext auf eine fehlerfreie Basis anzuheben und ein kompilierbares und nutzbares Endprodukt zu erhalten. Dabei sollte mindestens ein Anwendungsfall eines Angriffs komplett durchgetestet werden.\\

Umgesetzt wurde dann ein leichtes Refactoring und die Erstellung von 222 Unit-Tests\footnote{\url{https://github.com/devtty/webcastellum/actions/runs/5441412243/jobs/9895332455\#step:8:437} aufgerufen am 07.07.2023}. Dabei wurden:

\begin{itemize}
    \item 33 Bugs beseitigt
    \item 6 Sicherheitslücken beseitigt
    \item 54 Security-Hotspots überprüft
    \item über 300 Code-Smells beseitigt
    \item und die Testabdeckung des Quelltextes von 0 auf 18,8 Prozent angehoben.
\end{itemize}

Zwei Testklassen (\verb=WebFilterDroneTest= und \verb=WebFilterHttpClientTest=) mit insgesamt 14 Tests spielen dabei den Anwendungsfall eines Angriffs per SQL-Injection auf eine kleine Beispiel-Anwendung durch und überprüfen die korrekte Funktionalität des Filters. Mit Hilfe der Arquillian-Erweiterungen \emph{Drone} und \emph{Graphene} wird im Fall der ersten Testklasse ein Browser instantiiert und der Angriff automatisiert über diesen durchgeführt. Im zweiten Fall geschieht dieses, mit Hilfe eines leichtgewichtigen Clients, direkt über das HTTP-Protokoll.

\begin{figure}[h]
    \centering
    \includegraphics[width=\textwidth]{scfinalc.png}
    \caption{Ergebnis des Bugfixings (Juni 2023)}
    \label{fig:my_sonarf}
\end{figure}

Mit der Möglichkeit der Überwachung der qualitativen Metriken der Anwendung und der erweiterten Testabdeckung können die Arbeiten zur Erweiterung der Web Application Firewall beginnen. Hierzu zählen insbesondere der Datenaustausch zwischen verschiedenen Instanzen über eine zentrale Komponente und die (hoffentlich) effektivere Angriffserkennung.

% Highlight 2 der Implementierung
\section{Zentralisierung}

Im Ursprungszustand handelt es sich bei WebCastellum um einen relativ leichtgewichtigen Servlet-Filter der sämtliche Abwehrmaßnahmen direkt verarbeitet. Abgesehen von den Interaktionen mit der eigentlichen Anwendung existieren nur wenige Berührungspunkte mit externen Systemen. Die zwei wichtigsten Fälle sind dabei die \emph{Konfiguration} der Anwendung und die \emph{Benachrichtung} des Betreibers über den Zustand (Funktion, Angriffe, usw.) der Web Application Firewall. Beide Fälle benutzen derzeit textuelle Schnittstellen. Die Konfiguration erfolgt über \emph{manuelles} Editieren des Bereitstellungsdeskriptors (\emph{web.xml}) und die Bereitstellung des Status über Logdateien. Um den Datenaustausch mehrerer Instanzen der WAF nach dem Muster aus Abschnitt~\ref{sec:zentralisierungkern} zu ermöglichen, müssten zusätzliche (maschinell lesbare) Schnittstellen geschaffen werden. Diese Schnittstellen sorgen dann für den Datenaustausch mit einer zentralen Schnittstelle (Abbildung~\ref{fig:my_verbund}).\\

\begin{figure}[h]
    \centering
    \includegraphics[width=8cm]{central.png}
    \caption{Arbeit der WAFs im Verbund}
    \label{fig:my_verbund}
\end{figure}

Glücklicherweise haben die Entwickler der WebCastellum-Software entsprechende Schnittstellen (nun im Sinne der objektorientierten Programmierung) mitgeliefert. Das System bietet mit dem Interface \verb=org.webcastellum.ConfigurationLoader= sozusagen bereits eine technische Blaupause für die Implementierung einer alternativen Konfigurationsmöglichkeit. Tatsächlich finden sich in der Software bereits zwei Implementierungen des \emph{Interfaces}:

\begin{description}
  \small
\item[DefaultConfigurationLoader] \hfill \\
  lädt die Konfiguration, wie zuvor beschrieben, aus Parametern im Bereitstellungsdeskriptor (\emph{web.xml})
\item[PropertiesFileConfigurationLoader] \hfill \\
  lädt die Konfiguration aus einer benannten Datei im \emph{properties}-Format mit der Möglichkeit bei fehlenden Werten auf die Daten der \emph{web.xml} zurückzufallen.
\end{description}

Die Klasse \verb=org.webcastellum.ConfigurationManager= implementiert als sogenannter \emph{Service-Locator}\footnote{s. \url{https://martinfowler.com/articles/injection.html} abgerufen am 14.08.2023} eine einfache Möglichkeit zwischen den beiden Implementationen zu wechseln.\\

Eine zusätzliche Implementation der Schnittstelle, wie der in Abbildung~\ref{fig.impkonfig} dargestellte\\ \verb=RemoteConfigurationLoader=, könnte sich die Konfigurationsparameter über beliebige Quellen, zum Beispiel in Form eines REST-API-Clients/MicroServices, \glqq\emph{besorgen}\grqq. 

\begin{figure}[h]
  \begin{center}
    \includegraphics[width=14cm]{configclasses}
    \caption{Implementierung Konfiguration}
    \label{fig.impkonfig}
  \end{center}
\end{figure}

Wenn die Daten zur Konfiguration der Web Application Firewall von einem zentralen Dienst bezogen werden, muss dieser seinen \emph{Kunden} auch identifizieren können. Beispielsweise über einen eindeutigen Parameter (z.B. eine UUID - Universally Unique Identifier) der beim Abruf der Konfiguration mitgeliefert wird. \\

\textcolor{bhtGray}{\ding{110} Beispiel:} In der folgenden Abbildung~\ref{fig:confjson} sehen Sie eine Konfiguration des WebCastellumFilters wie sie von der beschriebenen Implementation abgerufen wird. Die Anwendung identifiziert sich dabei beim Abruf über eine eindeutige Identifikationsnummer (UUID), hier z.B. \url{https://devtty.de/wc/550e8400-e29b-11d4-a716-446655440000/config.json}. In diesem Fall kann der Abruf über den direkten Aufruf der URL erfolgen. Im produktiven Einsatz sollte eine Absicherung des Abrufs von Konfigurationsparametern über geeignete Sicherheitsmechanismen bedacht werden, da die Konfiguration besonders sicherheitsrelevant ist.\\

\lstset{%basicstyle=\ttfamily\color{black}\small,
    string=[s]{"}{"},
    stringstyle=\color{bhtBlue},
    comment=[l]{:},
    commentstyle=\color{black},
    showstringspaces=false
% 	commentstyle=\color{gray}\textsl
}
\begin{figure}[h]
  \centering
  \begin{lstlisting}

{
    "ApplicationName": "TestApplicationRemote",
    "LogVerboseForDevelopmentMode": "true",
    "DefaultProductionModeCheckerValue": "false",
    "LogClientUserData": "true",
    "Debug": "true",
    "CharacterEncoding": "UTF-8",
    "DefaultAttackerLoggerDirectory": "/home/user/temp",
    "LearningModeAggregationDirectory": "/home/user/temp",
    "BlockInvalidEncodedQueryString": "true",
    "ParameterAndFormProtection": "false",
    "SecretTokenLinkInjection": "false",
    "QueryStringEncryption": "false",
    "ExtraCheckboxProtection": "false"    
}
\end{lstlisting}
\caption{Konfiguration im JSON-Format}
\medskip
\small
\label{fig:confjson}

\end{figure}

Diese Schnittstelle berücksichtigt jedoch nur den Abruf von Konfigurationen, mehr ist für eine einzelne Instanz der Firewall auch nicht notwendig. Im Verbund sollten Konfigurationsparameter jedoch anpassbar sein und die Möglichkeit bestehen diese auch zu speichern. Glücklicherweise bietet WebCastellum bereits den sogenannten \glqq\emph{Lernmodus}\grqq{} an.  



\begin{figure}[h]
  \begin{center}
    \includegraphics[width=12cm]{classp}
    \caption{Implementierung Nachrichtenversand}
    \label{fig.impversand}
  \end{center}
\end{figure}

\section{Lernmodus}
\label{sec:lernmodus}
Der Lernmodus erfasst alle HTTP-Anfragen und speichert diese in einer eigenen Text-Datei. Bei aktiviertem Lernmodus geht WebCastellum davon aus, dass alle Anfragen auch gewünschte Anfragen sind. Aus den aufgezeichneten Anfragen lässt sich beispielsweise eine Whitelist an URL-Pfaden oder Stichwörtern generieren, die für eine weitere Absicherung nützlich sind. Technisch wird an dieser Stelle das gleiche Logging-Framework wie bei der Benachrichtigung über mögliche Angriffe genutzt und eine Erweiterung kann in Anlehnung an Abbildung~\ref{fig.impversand} erfolgen. Der Nachrichtenversand erfolgt auch in diesem Fall an die zentrale Komponente, wo beispielsweise auch eine weitere Verarbeitung stattfinden kann. Manaseers Vorschlag (s.~Abschnitt~\ref{sec:zentralisierungkern}) eines zentralen \glqq\emph{Command and Control Centers}\grqq{} läss sich somit recht einfach mit WebCastellum als \emph{Client} umsetzen.


\section{ML-Fähigkeit}
\label{sec:mlfk}

Zur Integration von Funktionen im Bereich \emph{Machine Learning} aus Implementierungssicht sollte zuerst überlegt werden in welchen Phasen der Anwendung einzelne Schritte ausgeführt werden. Da der WebCastellum-Filter bei jeder HTTP-Anforderung aktiv, wird ist dieses wohl der schlechteste Zeitpunkt für die Trainingsphase einer ML-Komponente. Die einfachste Möglichkeit diese auszuführen wäre bei Anwendungsstart, ggf. in einer eigenen Klasse, und das resultierende Model dem Filter als Singleton zur Verfügung zu stellen. Der WebCastellum-Filter bietet mit der Methode \verb=restartCompletlyWhenRequired()= jedoch selbst die Möglichkeit seine Konfiguration in bestimmten Intervallen und bei Bedarf neu nachzuladen. Innerhalb dieser Methode erfolgt ein Aufruf von \verb=checkRequirementsAndInitialize()=, hier wird die Konfiguration explizit neugeladen und der Filter neu initialisiert. Der Start der Trainingsphase sollte in dieser Methode erfolgen und möglichst an derem Ende ausgeführt werden, um etwaige Konfigurationparameter mit zu berücksichtigen.\\

Für die Traininingsphase wird ein Entscheidungsbaum benötigt. Dafür lässt sich auf bereits bestehende Software-Komponenten zurückgreifen. Sowohl Carmen Torrano-Giménez als auch Pete Scully nutzten in ihren Arbeiten (s.~Kapitel~3) die Analysesoftware Weka der Universität von Waikato. Zum Aufbau des Entscheidungsbaumes in WebCastellum werden Komponenten aus der Weka-Software genutzt, diese sind als Bibliothek im \emph{Maven Central Repository} verfügbar und können mit einem einfach Eintrag in der \verb=pom.xml= eingebunden werden.

\lstset{language=XML,
% 	basicstyle=\ttfamily\color{black}\small,
 	keywordstyle=\bfseries\color{bhtBlue},
 	identifierstyle=\color{bhtGray},
        showstringspaces=false
% 	commentstyle=\color{gray}\textsl
}
\begin{figure}[h]
  \centering
  \begin{lstlisting}
  <dependency>
    <groupId>nz.ac.waikato.cms.weka</groupId>
    <artifactId>weka-dev</artifactId>
    <version>3.9.6</version>
  </dependency>
  \end{lstlisting}
  \caption{Weka als Abhängigkeit in der WebCastellum-POM}
\label{fig:wekapom}

\end{figure}

Mit dem Eintrag als Abhängigkeit für das Projekt (Abbildung~\ref{fig:wekapom}) stehen für die weitere Entwicklung der Funktionalität alle Möglichkeiten von Weka ebenfalls zur Verfügung. Für die Trainingsphase wird die bereits erwähnte Methode \verb=checkRequirementsAndInitialize()= wie in Abbildung~\ref{fig:wekatrain} erweitert. Als Erstes wird als Datenquelle der Datensatz geladen. Wekas Klasse \verb=DataSource= erlaubt dabei verschiedene Datenformate. Es können Scullys CSIC2010-Daten im CSV-Format geladen werden oder auch Daten in Wekas ARFF-Format, dass bereits aus Abschnitt~\ref{sec:formatw} bekannt ist. Danach folgend wird der Entscheidungsbaum erstellt. Hierfür wird der \verb=Classifier= mit einer Instanz der Klasse \verb=weka.classifiers.trees.J48= initialisiert. Diese stellt eine Implementierung des C4.5-Algorithmus aus \cite{Quinlan1993} zur Verfügung. Die Daten aus dem geladenen Datensatz werden dann in den Baum eingepflegt. Aus Abschnitt~\ref{sec:formatw} ist bekannt, dass nicht alle Attribute des Datensatzes in den Entscheidungsbaum einfliessen müssen. Daher werden einige der Attribute \verb=Remove= aus dem Datensatz entfernt bevor der \emph{Classifier} erzeugt wird. Im Anschluss wird der erzeugte Classifier/Entscheidungsbaum in einer Variable die dem Rest der Anwendung zur Verfügung steht hinterlegt.\\

\lstset{language=Java,
% 	basicstyle=\ttfamily\color{black}\small,
 	keywordstyle=\bfseries\color{bhtBlue},
 	identifierstyle=\color{bhtGray},
        showstringspaces=false
% 	commentstyle=\color{gray}\textsl
}
\begin{figure}[h]
  \centering
  \begin{lstlisting}
  private void checkRequirementsAndInitialize() throws UnavailableException {
    if (this.filterConfig == null)
      throw new IllegalStateException("Filter init...");
        .
        .
    DataSource source = new DataSource("csic_2010_norm.csv");
    Instances dataset = source.getDataSet();
                       
    dataset.setClassIndex(dataset.numAttributes() - 1);

    Classifier j48 = new J48();
    dataset.deleteAttributeType(NUMERIC);

    //Entferne alle Attribute ausser Methode und URL-Pfad
    Remove filter = new Remove();
    filter.setAttributeIndices("3,4,5,6,7,8,9,10,11,12,13,14,15,16,17");
    filter.setInputFormat(dataset);

    Instances filtered = Filter.useFilter(dataset, filter);
    filtered.setClassIndex(filtered.numAttributes() - 1);
    j48.buildClassifier(filtered);

    this.traindata = filtered;
    this.classifier = j48;
        .
        .
  }
\end{lstlisting}
\caption{Aufruf der Trainingsphase}
\medskip
\small
Aufbau eines Entscheidungsbaums (j48) für die Klassifizierung mit den Daten aus dem CSIC2010-Datensatzes unter Verwendung der WEKA-Bibliothek. Die Daten werden dabei aus einer CSV entnommen.
\label{fig:wekatrain}

\end{figure}


Die Testphase (\emph{Evaluation}), in der die aktuelle HTTP-Anforderung gegen das Classifier bzw. den Entscheidungsbaum geprüft wird, muss jedoch für jeden Request einzeln ausgeführt werden. Innerhalb der Überprüfung des Requests (s.~Abbildung~\ref{fig:wcfilter} sollte die Evaluation nach allen anderen Prüfungen stattfinden. ML-Operationen sind rechen- und speicherintensiv und sollten daher nach Möglichkeit eher vermieden werden. Lösen bereits die vorherigen Prüfungen nach Logik- oder Regelverletzungen einen Alarm aus, ist eine weitere Prüfung nicht notwendig.\\

\begin{figure}[h]
  %% \begin{center}
  \centering
  \includesvg[inkscapelatex=false,width=0.7\textwidth]{seq}
  \caption{Ablauf einer Prüfung im WebCastellum Filter}
  \label{fig:wcfilter}
  \medskip
  \small
  schematische Darstellung nach Methodenaufruf von \verb=WebCastellumFilter#internalDoFilter=; die Testphase wird als letzter Punkt hinzugefügt
%%  \end{center}
\end{figure}

Für Erweiterungen von WebCastellum die \emph{nach} der regelbasierten Überprüfung ausgeführt werden, bietet sich die Methode \verb=doAfterProcessing= an. An dieser Stelle sollte die aktuelle HTTP-Anfrage gegen den zuvor erstellten Entscheidungsbaum geprüft werden. Dazu muss der Http-Request vor der Auswertung (\verb=Evaluation=) in eine für Weka interpretierbare Form (\verb=Instance=) übertragen werden.\\

\lstset{language=Java,
% 	basicstyle=\ttfamily\color{black}\small,
 	keywordstyle=\bfseries\color{bhtBlue},
 	identifierstyle=\color{bhtGray}, 
% 	commentstyle=\color{gray}\textsl
}
\begin{figure}[h]
  \centering
  \begin{lstlisting}
  private void doAfterProcessing(final RequestWrapper request,
    final ResponseWrapper response) throws IOException, ServletException {
        .
        .
      Instance requestInstance = new DenseInstance(2);
      requestInstance.setValue(1, request.getMethod());
      requestInstance.setValue(2, request.getRequestURI());
        .
      Evaluation eval = new Evaluation(this.traindata);
      double prediction = eval.evaluateModelOnce(this.classifier,
          requestInstance);

      if(prediction < 1.0){
          //handle response
      }
     
        .
        .
  }
\end{lstlisting}
\caption{Übergabe des aktuellen Requests an die Testphase}
\label{fig:wekatest}

\end{figure}

Im Filter steht nun eine Implementierung bereit, die neben den ursprünglichen Prüfungen auch die Fähigkeit besitzt \emph{Erlerntes} zu nutzen und eine Absicherung der von Torrano-Giménez genutzten Testanwendung anhand ihres Datensatzes mittels WebCastellum wäre nun auch praktisch möglich. Die Testergebnisse würden sich jedoch massiv unterscheiden. Zum Einen werden die regelbasierten Filtermechanismen ausgeführt und zum Anderen erfolgt die derzeitige Prüfung nur gegen die beiden Attribute \emph{HTTP-Methode} und \emph{URL-Pfad}. Für eine Verwendung im realen Umfeld sind noch zwei Änderungen der bestehenden Implementierung notwendig.\\

Die Datenquelle für die Trainingsphase sollte ihre Daten nicht aus einer statischen Datei beziehen. Ideal wäre hier der Bezug aus dem \emph{Command and Control Center} (s.~Abschnitt~\ref{sec:lernmodus}), z.B. zusammen mit der zentralisierten Konfiguration. Hierbei sollte jedoch unbedingt ein performanter Weg gefunden werden, da die anfallenden Datenmengen je nach Anwendung entsprechend groß werden könnten. Gleiches gilt für die zweite, und wesentlich wichtigere, Änderung. In der derzeitigen Implementierung werden nur zwei der Argumente aus dem Datensatz genutzt. Um die Genauigkeit zu erhöhen sollten noch weitere Argumente im Entscheidungsbaum ausschlaggebend sein. Insbesondere die \emph{Payload} ist für eine genaue Auswertung von Bedeutung, da hier das größte Potential an Möglichkeiten einer nicht-erwünschten Manipulation besteht. Mit jedem zusätzlichen Argument steigt jedoch der Speicherbedarf des Filters um ein Vielfaches an.\\
Dieses Problem ist bereits bekannt und kann gelöst werden indem statt der zahlreichen Varianten eines Arguments ein abgeleiteter Wert (mit geringerem Umfang) genutzt wird. Einen konkreten Vorschlag für die \emph{Payload} bzw. die Request-Parameter existiert beispielsweise in \cite{kozik2014}. Rafal Kozik und seine Kollegen schlugen vor, anstatt der eigentlichen Parameter, kodierte Werte in \glqq\emph{Histogramme fester Länge}\grqq umzuwandeln und diese stattdessen als Feature zu verwenden.\\

\textcolor{bhtGray}{\ding{110} Aus~\cite{kozik2014}:}Therefore, to make it possible to learn a classifier, we transform this vector to histograms of constant length. The final feature vector is extended with information whenever a given request is on the whitelist or not.\\

Die Verwendung des von Kozik vorgeschlagenen Algorithmus würde als Nebeneffekt auch die Performance-Probleme bei der Übertragung des Datensatzes an die zentrale Komponente entsprechend abmildern. Leider fehlte im Rahmen dieser Arbeit die Zeit zur praktischen Umsetzung der letzten beiden Vorschläge und die folgende Evaluierung muss sich auf die Basis-Implementierung beschränken.


% Highlight des Deployments beim Kunden

% Zusammenfassung: ca. 0,5 Seiten
\section{Zusammenfassung}

Dieses, sehr technische, Kapitel führte uns von einem alten und etwas angestaubten Projekt zu einer moderneren Lösung. Mit einem kleinen Einblick in die praktische Softwareentwicklung und die damit verbundene Beseitigung technischer Schulden erreichte die Anwendung einen ausführbaren Zustand. Es folgten Beispiele für die technische Umsetzung einiger Ideen aus dem vorhergehenden Kapitel. Es wurde dargestellt wie eine praktische Umsetzung zur Anbindung an ein zentrales System (siehe Abbildung~\ref{fig:my_future}) implementiert werden kann und wie die das Produkt WebCastellum um Funktionen des Maschinellen Lernens erweitert wird. 

\begin{figure}[h]
    \centering
    \includegraphics[width=9cm]{webcastellumcentral.png}
    \caption{Komponentendiagramm für zentrale Klassifizierung}
    \label{fig:my_future}
  \end{figure}

Das nächste Kapitel beginnt mit der Beschreibung des Testaufbaus zur Überprüfung der Funktion der Web Application Firewall. Mit Hilfe des Zed Attack Proxys wird die Anwendung überprüft und auf Schwachstellen getestet. Gleichzeitig können die Tests aufgezeichnet werden und der Datensatz mit diesen Daten erweitert werden.

% Was haben wir in diesem Kapitel gelernt?
% Wie passt das zur Zielstellung der Arbeit?
% Wie passt das zum nächsten Kapitel?