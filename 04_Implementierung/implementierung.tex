\chapter{Implementierung}

\anno{max. 5 Seiten} %Ref hat 12

% Highlight 1 der Implementierung

Als Grundlage für die Implementierung der Firewall-Funktionalität wurde mit der Software \emph{WebCastellum} eine etwas ältere Implementierung eines Servlet-Filters gewählt.

\section{Vorarbeiten}

%\begin{figure}[h]
%    \centering
%    \begin{gnuplot}[terminal=latex, scale=1.0]
%    set datafile separator ','
%	set xdata time
%    set timefmt "%Y-%m"
%    set xrange ["2009-01":"2022-04"]
%    set format x "%b %y"
%    set key autotitle columnhead
%	plot 'downloads.csv' using 1:2 with lines
%\end{gnuplot}
%    \caption{Downloads Statistics WebCastellum 1.8.3 binary}
%    \label{fig:my_label}
%  \end{figure}
  
\subsection{Unit-Tests etc.}
\subsection{Fehlerbereinigung}

% Highlight 2 der Implementierung
\section{Zentralisierung}
\begin{figure}[ht]
    \centering
    \includegraphics[width=8cm]{central.png}
    \caption{Arbeit der WAFs im Verbund}
    \label{fig:my_verbund}
\end{figure}



\section{ML-Fähigkeit}

\begin{figure}[h]
    \centering
    \includegraphics[width=9cm]{webcastellumcentral.png}
    \caption{Komponentendiagramm für zentrale Klassifizierung}
    \label{fig:my_future}
\end{figure}


% Highlight des Deployments beim Kunden

% Zusammenfassung: ca. 0,5 Seiten
\section{Zusammenfassung}



