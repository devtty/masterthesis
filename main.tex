%%
%% Berliner Hochschule für Technik -- Abschlussarbeit
%%
%% Hauptdokument
%%
%% 23.01.09 Tschirley V.01 Beuth Hochschule
%% 26.08.21 Tschirley V2.0 Umbenennung zur Berliner Hochschule für Technik
%%
%%%%%%%%%%%%%%%%%%%%%%%%%%%%%%%%%%%%%%%%%%%%%%%%%%%%%%%%%%%%%%%%%%%%%
\documentclass[11pt, a4paper]{book}
%\documentclass[11pt, a4paper, oneside]{book}
%% Übersetzen als Entwurf
\usepackage[entwurf]{bhtThesis}
%% Übersetzen für die Abgabe
%\usepackage[abgabe]{bhtThesis}
\typeout{BHT-Abschlussarbeit V2.0 26.08.21 S.Tschirley}

\usepackage{blindtext}   %für Blindtext in Kapitel 2
\usepackage{listings}
\lstset{ 
  literate={ö}{{\"o}}1
           {ä}{{\"a}}1
           {ü}{{\"u}}1
           {Ö}{{\"O}}1
           {Ä}{{\"A}}1
           {Ü}{{\"U}}1
           {ß}{{\ss}}1
}
\usepackage[hidelinks]{hyperref}

%%
%% Es folgen einige Zusätze, die in Kapitel 1 beschriben sind. 
%% Alles was nicht notwendig ist, kann auskommentiert werden
%%
\usepackage{trsym}
%\usepackage{showkeys}
\usepackage{bytefield}
\usepackage{svg}
%\usepackage{pict2e}
\usepackage{gnuplottex}
\usepackage{rotating}
\usepackage{booktabs}
%%
%% Pfad zu den Bildern
%%
\graphicspath{
  {pictures/},
  {einleitung/pictures},
  {kapitel1/pictures/},
  {kapitel2/pictures/}
}

%%
%% Einbinden persönlicher macros 
%%
\input{personalMacros.tex}

%% Message
\typeout{-----------------------------------------------------------}
\typeout{----> main.tex ---- Zentrales Dokument---------------------}
\typeout{-----------------------------------------------------------}

\version{0.1$\alpha$}    % word im Entwurf auf der Titelseite vermerkt
\datum{\today}
%%
%% Titel, Autor und Betreuer
%%
\fachbereich{VI -- Informatik und Medien} 
\studiengang{Medieninformatik Online}
\thesistyp{Masterarbeit} 
\autor{Denis Renning}
%\edvnr{123 456}
\titel{Web Application Firewalls und Machine Learning} 
\untertitel{Migration und Erweiterung am Beispiel}
\betreuerFeld{
  \begin{tabular}{lr}
    \multicolumn{2}{l}{\textbf{Gutachter}}\\
    Prof.~Dr.~S.~Edlich & Berliner Hochschule für Technik\\
    Prof.~Dr.~S.~Haschemi& Berliner Hochschule für Technik
  \end{tabular}
}

%%\renewcommand{\baselinestretch}{1.05} 
\begin{document}
\pagestyle{fancy}

\input{titelseiten.tex}

\pagenumbering{arabic}
%%%%%%%%%%%%%%%%%%%%%%%%%%%%%%%%%%%%%%%%%%%%%%%%%%%%%%%%%%%%%%%
%% Die Kapitel der Arbeitw

%Einfuehrung (insgesamt ca. 5 Seiten)
%  Das grosse Problem
%  Ziel der Arbeit (oder das kleine machbare Problem)
%  Methodik (1 Seite)
%  Gliederung und Aufbau (0,5 Seiten)
%\input{kapitel1/ch1.tex}
\chapter{Einleitung}

\anno{ca. 5 Seiten}

% das grosse Problem

% Ziel der Arbeit

% Methodik (1 Seite)

% Gliederung und Aufbau (ca. 0,5 Seiten)


%Grundlagen und verwandte Arbeiten (Bitte nicht mehr als 8 Seiten)
%  Grundbegriffe
%  Thema 1
%  Thema 2 (optional)
%  Thema 3 (optional)
%  Zusammenfassung (ca. 0,5 Seiten)
%\input{kapitel2/ch2.tex}
\chapter{Grundlagen}

\anno{bitte nicht mehr als 8 Seiten}

% Grundbegriffe 2 Seiten
\section{Grundbegriffe}
\subsection{WAF allgemein}
Um die Sicherheit einer Anwendung zu gewährleisten sind viele verschiedene Schritte notwendig, so sollten natürlich bereits in der Anwendung selbst Ein- und Ausgabemöglichkeiten, z.B. durch Validierung, Encoding, usw. überprüft und gegebenenfalls eingeschränkt werden. Dabei lässt sich nicht jeder mögliche Angriffsfall vorhersehen oder eine eigene Implementierung ist zu aufwendig und teuer, weil häufig nicht ausreichend personelle Ressourcen oder Bugdet vorhanden sind um eine Web Applikation auf alle möglichen Sicherheitslücken zu prüfen. Im Laufe des Betriebes einer Web Anwendungen können zudem neue Angriffsvarianten entstehen und zusätzlich ist gerade bei Web Applikationen der Zeitdruck zur Veröffentlichung einer solchen häufig sehr hoch.

An diesem Punkt kommen sogenannte Web Application Firewalls ins Spiel. Im Gegensatz zu regulären Firewalls haben Web Application Firewalls direkten Zugriff auf die HTTP-Anfragen (requests) und Antworten (responses) und können diese entsprechend bewerten und gegenbenenfalls blockieren oder gefährdende Inhalte filtern oder umschreiben.\\\\
%% 
\textcolor{bhtGray}{\ding{110} Definition\footnote{\url{https://owasp.org/www-community/Web_Application_Firewall} abgerufen am 30.05.2023}} A web application firewall is an application firewall for HTTP applications. It applies a set of rules to an HTTP conversation. Generally, these rules cover common attacks such as Cross-site Scripting (XSS) and SQL Injection. While proxies protect generally protect clients, WAFs protect servers. A WAF is deployed to protect a specific web application or set of web applications. A WAF can be considered a reverse proxy. WAFs may come in the form of an appliance, server plugin, or filter, and may be customized to an application to an application. The effort to perorm this customization can be significant and needs to be maintained as the application is modified.\\\\
%%
asd



\subsubsection{Anwendungsfälle}
%% Anwendungsfaelle WAF (gut beschrieben bei WAFEC2)
%% irgendwie Uebergang zu ML und WAF mit ML schaffen
%% Sammlung; payload,fuzzer,fingerprinting, bypassing

\subsection{Arten}

\subsubsection{Unterscheidung nach Position}
Grundsätzlich lassen sich solche Systeme nach ihrer Position in der Netzwerk- und Servertopologie unterscheiden. Es existieren einerseits Systeme die vor eine Anwendung geschaltet werden und Systeme die direkt in die Anwendung integriert werden. Die erste Gruppe lässt noch eine Verzweigung in weitere Unterarten, wie \emph{Reverse Proxy}, \emph{Appliance}, \emph{Plugins} für WebServer oder \emph{Passive Devices} (IDS), zu.


\subsubsection{Unterscheidung nach Abwehrmaßnahmen}

\paragraph{Regelbasierte Systeme}
Der Großteil bekannter WebApplicationFirewalls arbeitet jedoch \emph{regelbasiert}. In diesem Fall werden ein- und ausgehende Datenströme (Requests/Responses) unabhängig voneinander (zustandslos) betrachtet und einer Mustererkennung unterworfen. Häufig sind die Regeln anhand sogenannter \emph{Regular Expressions} definiert.

\paragraph{Logische Systeme}
Bei logikbasierten Abwehrmaßnahmen handelt es sich um Abwehrmaßnahmen die aufgrund von bekannten (logischen) Rückschlüssen eingeleitet werden. Einfachstes Beispiel wäre das temporäre Sperren der Loginseite bei dreimalig falschem Login. Bei Nutzung einer WAF die keine logikbasierten Auswertungen ermöglicht, müssten entsprechende Anwendungsfälle in der Anwendung selbst implementiert werden. (Oder im Fall der gerade beschriebenen Authentifizierungproblematik in einen externen Dienst ausgelagert werden.)

\subsection{Grundbegriffe allgemein}

\textbf{Bypassing:}

\textbf{Filter:}

\textbf{Fingerprinting:} Ähnlich der Abnahme und Identifizierung von Personen mit Hilfe eines individuellen Fingerabdrucks können auch Produkte wie Software anhand spezifischer Merkmale identifiziert werden. Beim \emph{Fingerprinting}

\textbf{Fuzzer:}

\textbf{Payload:}

\textbf{Request:} 

\textbf{Response:}

\section{Related Work} %umbenennen ca. 6

% Thema 1
\subsection{Evolution der Firewalls}

\subsubsection{Strikt nach Regeln}

Die praktisch einfachste Web Application Firewall wäre \emph{-beispielsweise-} eine einfache Regel innerhalb der Webserver-Konfiguration, die anhand eines bestimmten Merkmals den Datenverkehr entweder korrekt beantwortet oder \glqq\emph{abwehrt}\grqq. Wenn nicht explizit entnommen findet sich der folgende Eintrag in jeder Konfiguration des \emph{Apache httpd}-Servers und sorgt dafür dass der Zugriff auf jede Datei deren Dateiname mit \texttt{.ht} beginnt unterbunden wird:

\lstset{language=XML,
 	basicstyle=\ttfamily\color{black}\small,
 	keywordstyle=\bfseries\color{bhtBlue},
 	identifierstyle=\color{black}, 
 	commentstyle=\color{gray}\textsl
      }
%      \begin{figure}
%        \caption{Beispiel für einfache Regel}
%        \label{fig:httprule}
\begin{lstlisting}
  <Files ".ht*">
    Require all denied
  </Files>
\end{lstlisting}
%      \end{figure}

Mit steigender Komplexität der zu schützenden Anwendungen steigt auch der Bedarf an Regeln. Zu berücksichtigende Auswahlkriterien beschränken sich dann auch nicht nur auf den \glqq\emph{Dateinamen}\grqq. Attribute wie der Aufrufzeitpunkt, der Inhalt des Aufrufs, die Identität des Aufrufenden und viele andere Kriterien können zu entscheidenden Faktoren werden. Im Sinne der Entkopplung wurde diese Filterlogik häufig in entsprechende Module oder Plugins ausgelagert. Bekanntester Vertreter ist das \emph{modSecurity}-Modul für den Apache httpd-Server.

\begin{lstlisting}
  SecRule REQUEST_URI ".ht*" "deny"
\end{lstlisting}

% einfache regel => firewall
% Systeme zum verwalten der Regeln
% Standardisierte Regeln (OWASP CRS)

\subsubsection{Hybride Ansätze}

% Krueger Manaseer etc.

\subsubsection{Fortschritte in Richtung Intelligenz}

Collaborative Detection \cite{karakannas2014}

% kruegel gimenez appelt kozik testen mit ML Ansätzen

% Thema 2
\subsection{Thema 2 - ML}

%ansaetze und kombination?

% Thema 3 optional die andere seite
\subsection{hacking Wafs}


% Zusammenfassung (ca. 0,5 Seiten)
\section{Zusammenfassung}

% ggf. ditaa tabelle ueber den Zeitverlauf der verschiedenen Arbeiten nach Attack-Defend-Muster


%Kern der Arbeit (ca. 15 - 20 Seiten)
%  Probleme und die Lösungsansätze
%  Methodik/Vorgehen (diesmal ausfuehrlich)
%  Uebersicht bzw. Architektur
%  Oder aber: Der xxx Algorithmus
%  Optional: Der yyyy Algorithmus
%  Zusammenfassung (ca. 0,5 Seiten)
\chapter{Kern der Arbeit}

\anno{ca. 15-20 Seiten (10)}
Ausgehend von der historischen Entwicklung der Web Application Firewalls werden in diesem Kapitel einige der damit verbundenen Probleme angesprochen. Insbesondere die Arbeit von Carmen Torrano-Giménez leistete einen grundlegenden Beitrag für die Forschung im Bereich der IT-Sicherheit von Webanwendungen und wird immer wieder in Beiträgen und Artikeln referenziert. Der von ihr erstellte Datensatz findet sich immer wieder als Datenbasis in zahlreichen Arbeiten wieder. Zum Teil jedoch nicht ganz ohne kleinere Probleme. 

% Probleme und die Loesungsansaetze
\section{Kritik und Probleme}
\label{sec:Probleme}

Der im vorherigen Kapitel betrachtete zeitliche Rahmen, über die Entwicklung von Web Application Firewalls, umfasst mehr als zwanzig Jahre. Ein Zeitraum in dem sich zahlreiche Produkte in diesem Bereich etablierten. Um eine ungefähre Vorstellung von der Anzahl der verschiedenen Produkte zu erhalten, finden Sie im Anhang mit der Tabelle~\ref{tab:my_wafwoof} eine (nicht vollständige) Liste von derzeitigen WAF-Produkten verschiedener Hersteller. Über den gesamten Zeitraum haben sich Webtechnologien weiterentwickelt und auch die Art und Weise der Nutzung des Internets änderte sich. Mit der Verbreitung der Smartphones und Sprachassistenten wurden aus einfachen Webangeboten vielschichtige Anwendungen mit verschiedenen möglichen Clientsystemen (\emph{mobile-first-Ansatz}). Serviceorientierte Architekturen, Webservices, Microservices, Künstliche Intelligenz - alles entwickelte sich weiter. Im Bereich der IT-Sicherheit veröffentlichte das \emph{Open Web Application Security Project} 2019 erstmals eine Liste seiner zehn kritischsten Sicherheitsrisiken explizit für den Bereich der \emph{API}s. \emph{ModSecurity} ist immer noch der de-facto-Standart für regelbasierte WAFs und praktisch so gut wie der einzige verbliebene Vertreter der Web Application Firewalls aus dem OpenSource-Bereich. Freie anomaliebasierte Systeme sind so gut wie nicht existent und die privatwirtschaftlichen Akteure lassen sich kaum in die Karten schauen. \\
Mittlerweile sind bereits dreizehn Jahre seit dem Erscheinen des Datensatzes CSIC2010 vergangen und schaut man in die derzeitig vorhandene Literatur scheint auch kein Nachfolger in Sicht. \\\\

\textcolor{bhtGray}{\ding{110} Applebaum über den Datensatz CSIC2010~\cite{Applebaum2021}} Torrano-Giménez et al. provide an HTTP dataset intended to be used for the development and testing of Intrusion Detections Systems and WAFs. ... The dataset has been used extensively by other researchers ... There is still a need for a development of a new dataset as previous datasets had become outdated and did not target real systems. The dataset .. is itself now over 10 years old and targets a bespoke e-commerce system. \\\\

Dennoch ist der Datensatz weiterhin Ausgangsbasis für viele Forschungsvorhaben und praktische Projekte im Bereich der IT-Sicherheit. Im folgenden Kapitel wird daher näher auf den Datensatz CSIC 2010 eingegangen, dessen Aufbau und Verwendung erläutert und es werden mögliche Verbesserungen anhand von praktischen Beispielen vorgeschlagen.
%

\subsection{Aufbau und Format des Datensatzes CSIC2010}
\label{sec:aufbauformat}
% Aufbau /motivation von gimenez

% Aufbau
Der Datensatz besteht aus drei Teildatensätzen. Der erste Teildatensatz ist dabei für die Trainingsphase gedacht und enthält 36000 einzelne HTTP-Anfragen die den normalen bzw. erwünschten Datenverkehr abbilden. Die anderen beiden Teildatensätze sind für die Testphase gedacht. Ein Teildatensatz mit ebenfalls 36000 Anfragen an normalem Datenverkehr und ein Teildatensatz mit über 25000 Anfragen mit Gefährdungspotential (siehe \cite{csic2010}).\\

Die einzelnen Dateneinträge der drei Teildatensätze sind jeweils gleich aufgebaut und beinhalten die in Tabelle~\ref{tab:csicfields} aufgeführten Felder. Das Feld \emph{Payload} ist im Datensatz nicht explizit benannt bzw. auf den ersten Blick nicht ersichtlich, da es je nach HTTP-Methode in den Rohdaten unterschiedlich abgebildet wird. Bei einem \verb=GET=-Request wird die Payload im Query der URL übertragen (siehe \cite{rfc2626}~Abschnitt~3.2.2) und bei \verb=POST=-Requests im Request Body. Die Payload enthält im Wesentlichen die eigentlichen Daten und in den meisten Angriffsfällen genau die Anomalien die für einen Angriff verantwortlich sind. Zur Verdeutlichung zeigt Abbildung~\ref{fig:ccex} einen einzelnen Eintrag des Datensatzes \emph{CSIC2010} im Rohformat. Hier handelt es sich um einen \verb=GET=-Request mit den Parametern \emph{id},\emph{nombre}, \emph{precio}, \emph{cantidad} und \emph{B1}. Die einzelnen Dateneinträge des Datensatzes unterscheiden sich jedoch nicht in den übrigen Feldern.\\

Für eine Klassifizierung können bzw. müssen aus den Daten der einzelnen Requests entsprechende Features (\emph{Merkmale~-~individuell messbar}) abgeleitet werden. \emph{Achtung}, Felder und Features sind nicht gleichzusetzen! Torrano-Giménez hat in ihrer Doktorarbeit sowohl Features von Experten benennen lassen \footnote{siehe Anhang Tabelle~\ref{tab:tgfeatures}} als auch mittels statistischer Feature-Selection-Mechanismen \emph{extrahiert}. Bei den hier vorgeschlagenen Verbesserungen bzw. Änderungen wird sich nur auf die Datenfelder der Rohdaten oder auf die in \cite{Giménez2015} benannten \emph{Experten-Features} bezogen. 


%\lstset{language=XML,
% 	basicstyle=\ttfamily\color{black}\small,
% 	keywordstyle=\bfseries\color{bhtBlue},
% 	identifierstyle=\color{black}, 
% 	commentstyle=\color{gray}\textsl
% }
\begin{figure}[h]
  \centering
        \begin{lstlisting}[basicstyle=\footnotesize]
GET http://localhost:8080/tienda1/publico/anadir.jsp
                ?id=1&nombre=Jam%F3n+Ib%E9rico&precio=39&cantidad=41
                &B1=A%F1adir+al+carrito HTTP/1.1
User-Agent: Mozilla/5.0 (compatible; Konqueror/3.5; Linux) KHTML/3.5.8 (like Gecko)
Pragma: no-cache
Cache-control: no-cache
Accept: text/xml,application/xml,application/xhtml+xml,
                text/html;q=0.9;text/plain;q=0.8,image/png,*/*;q=0.5
Accept-Encoding: x-gzip, x-deflate, gzip, deflate
Accept-Charset: utf-8, utf-8;q=0.5, *;q=0.5
Accept-Language: en
Host: localhost:8080
Cookie: JSESSIONID=54E25FF4B7F0E4E855B112F882E9EEa5
Connection: close
\end{lstlisting}
\caption{Dateneintrag CSIC 2010 Normal Traffic Test}
\label{fig:ccex}
\end{figure}

\begin{sidewaystable}[ht]
  \centering
%  \resizebox{\textwidth}{!}{ 
  \begin{tabular}{lllll}
    \toprule
    % \textbf{Name} & \textbf{Wert} & \textbf{Anmerkungen} & \textbf{DN} & \textbf{DNT} & \textbf{DNA} \\
    \textbf{Name} & \textbf{Wert} & \multicolumn{3}{l}{\textbf{Anzahl unterscheidbarer Werte mit Fehlquote}} \\
     & & Training & Test & Anomalie \\
    \midrule
    HTTP Methode & \verb=GET= oder \verb=POST= im Anomalieteil auch \verb=PUT=  & 2 (0\%) & 2 (0\%) & 3 (0\%)\\
    URL  & & 28 (0\%) & 28 (0\%) & 1623 (0\%)\\
    \emph{Protocol} & \verb=HTTP/1.1= & 1 (0\%) & 1 (0\%) & 1 (0\%)\\
    \emph{User-Agent} & \verb=Mozilla/5.0 (compatible;Konqueror/3.5...= &  1 (0\%) & 1 (0\%) & 1 (0\%)\\
    \emph{Pragma} & \verb=no-cache=  & 1 (0\%) & 1 (0\%) & 1 (0\%)\\
    \emph{Cache-control} & \verb=no-cache=  & 1 (0\%) & 1 (0\%) & 1 (0\%)\\
    \emph{Accept} & \verb=text/xml,application/xml,...= & 1 (0\%) & 1 (0\%) & 1 (0\%)\\
    \emph{Accept-Encoding} & \verb=x-gzip,x-deflate,gzip,deflate= & 1 (0\%) & 1 (0\%) & 1 (0\%)\\
    \emph{Accept-Charset} & \verb!utf-8,utf-8;q=0.5,*;q=0.5! & 1 (0\%) & 1 (0\%) & 1 (0\%)\\
    \emph{Accept-Language} & \verb=en= & 1 (0\%) & 1 (0\%) & 1 (0\%)\\
    Host & \verb=localhost:8080= & 1 (0\%) & 1 (0\%) & 2 (0\%)\\
    Cookie & \verb=JSESSIONID= verschiedene Werte  & 36000 (0\%) & 36000 (0\%) & 25065 (0\%) \\
    Content-Type & \verb=null= oder \verb=application/x-www-from-urlencoded= & 2 (0\%) & 2 (0\%) & 2 (0\%)\\
    \emph{Connection} & \verb=close= & 1 (0\%) & 1 (0\%) & 1 (0\%)\\
    Content-Length & & 117 (0\%) & 117 (0\%) & 382 (0\%)\\
    Payload & & 19418 (0\%) & 20105 (19\%) & 14681 (5\%)\\
    \bottomrule
      % \end{tabular}}
    \end{tabular}
  \caption{Felder des CSIC2010 Datensatzes}
  \label{tab:csicfields}
\end{sidewaystable}

% Format des Datenssatzes Text -> CSV (Referenz zu Dr....) -> ARFF ?? ggf. mit Korrektur


Im Bereich der Anomalie-Requests sieht es ähnlich aus, jedoch wurde hier mit \verb=PUT= noch eine weitere HTTP-Methode angewandt~\cite{csic2010}. Des Weiteren existiert eine Unregelmäßigkeit bei den aufrufenden Hosts. Hier wird in den Anomaliedaten zusätzlich ein weiterer Port (9090) genutzt. 

\subsubsection{Zum Format}

Die drei originalen Teildatensätze befinden sich jeweils in eigenen Dateien und enthalten die Rohdaten als Aneinanderreihung der einzelnen HTTP-Requests. Die Daten sind textbasiert und umfassen nur Inhalte im lateinischen Alphabet\footnote{die originalen Dateien sind ISO-8859-1 kodiert} und können auch mit einfachen Textwerkzeugen gelesen (und bearbeitet) werden. Für eine weitere Analyse ist dieses Format jedoch eher ungeeignet.\\

Pete Scully erarbeitete für seine Doktorarbeit~\cite{Scully2016} unter anderem eine kommaseparierte Version des CSIC2010-Datensatzes zur Verwendung im  Analysewerkzeug WEKA\footnote{Waikato Environment for Knowledge Analysis}. Neben den einzelnen benannten Feldern fügte er noch die Felder \emph{Payload} und \emph{Label} hinzu. Wie in Abschnitt~\ref{sec:aufbauformat} bereits näher erläutert, handelt es sich bei der Payload um die \emph{eigentlichen} Inhalte des HTTP-Anforderungen. Das Feld \emph{Label} dient als Marker zur Unterscheidung ob es sich bei dem Dateneintrag um Trainings-, Test- oder Anomale Daten handelt, da Scully die Teildatensätze sowohl getrennt, als auch in einer großen CSV-Datei vereint zur Nutzung anbietet (siehe \cite{csiccsv2010}). Das WEKA-Tool vereinfacht eine genauere Betrachtung des Datensatzes erheblich und Tabelle~\ref{tab:csicfields} stellt nochmals die einzelnen Datenfelder übersichtlich dar.\\

Die letzten drei Spalten der Tabelle beinhalten die \emph{Anzahl unterscheidbarer Werte} und die \emph{Fehlquote}, d.h. den prozentualen Anteil an Dateneinträgen bei denen dieses Feld keinen Wert beinhaltet, für die jeweiligen Teildatensätze. Erkennbar sind einige Felder mit vorhandenem Wert in jedem Dateneintrag, der sich jedoch nie vom Wert in anderen Einträgen unterscheidet. Diese Felder sind demzufolge für eine Klassifizierung, bei der ausschließlich Daten des CSIC2010-Datensatzes genutzt werden, irrelevant. Torrano-Giménez hätte diese zum Teil weglassen können und es hätte, bei Verwendung ihrer Testdaten, nichts an ihren Ergebnissen geändert. In einer dynamischeren (und produktiven) Umgebung müssen diese Felder jedoch mit betrachtet werden\footnote{dazu mehr in Abschnitt~\ref{sec:Kategorisierung}}.\\

\textcolor{bhtGray}{\ding{110} Beispiel:} Die WAF erfasst mehrere hundert HTTP-Anfragen mit einer festen Session-ID (\verb=JSESSIONID=, Session-Cookie, o.ä.) und festem Wert im \emph{User-Agent} (~$\rightarrow$~ein Internetnutzer surft mit seinem Browser) und plötzlich erscheint eine Anfrage mit gleicher Session-ID aber anderem User-Agent. In diesem Fall sollte die WAF entsprechend reagieren, da die Möglichkeit eines Angriffes (\emph{Session Hijacking}, \emph{Man-in-the-middle}) durchaus besteht.\\
  
Zurück zum Format der Daten - das WEKA-Tool unterstützt ein eigenes Format namens \emph{\textbf{A}ttribute-\textbf{R}elation \textbf{F}ile \textbf{F}ormat} (ARFF)~\cite{arff2023}, welches die Daten ggf. mit entsprechenden Datentypen genauer beschreiben kann. Ein Vorschlag für einen Header im ARFF-Format der den erweiterten CSIC2010-Daten entsprechen könnte ist in Abbildung~\ref{fig:csicarff} dargestellt. Beispielhaft sind, hier rot eingefärbt, zwei Erweiterungen der nominalen Attribute \emph{Methode} und \emph{Protokoll} dargestellt, die den Datensatz erweitern. Bei Werten die durchaus größere, unterschiedliche Textmengen aufnehmen können (z.B. die \emph{URL} oder die \emph{Payload}) hat man die Wahl ob die Werte nominell abgespeichert werden oder der Datentyp String genutzt werden sollte. Falls im letztenn Fall der String-Wert für eine spätere Klassifizierung der genutzt werden soll, muss eine Konvertierung in einen nominellen Wert erfolgen.

\lstset{language=bash,
% 	basicstyle=\ttfamily\color{black}\small,
  keywordstyle=\bfseries\color{bhtBlue},
  morekeywords={@attribute,@relation},
% 	identifierstyle=\color{black}, 
% 	commentstyle=\color{gray}\textsl
        escapeinside={\%*}{*)}
      }
\begin{figure}[h]
  \begin{lstlisting}
@relation CSIC2010_erweitert

@attribute index numeric
@attribute method {GET,POST%*\textcolor{bhtRed}{,PUT,PATCH,DELETE}*)}
@attribute url %*\textcolor{bhtCyan}{String}*)
@attribute protocol {HTTP/1.1%*\textcolor{bhtRed}{HTTP/2}*)}
@attribute userAgent String
@attribute pragma String
@attribute cacheControl String
@attribute accept String
@attribute acceptCharset String
@attribute acceptLanguage String
@attribute host String
@attribute connection String
@attribute contentLength 
@attribute contentType String
@attribute cookie String
@attribute payload %*\textcolor{bhtCyan}{String}*)
@attribute label {norm,anom}
\end{lstlisting}
  \caption{Beispiel für einfachen arff header}
  \label{fig:csicarff}
\end{figure}
\anno{TODO Tabelle fehlplatziert, sollte erst wenn CSICÄnderungen bekannt oder Anhang}

\subsection{Alter des Datensatzes}

Der wesentlichste Kritikpunkt ist das Alter des Datensatzes. Seit der Erstellung des Datensatzes hat sich in der Welt der Webanwendungen einiges geändert. Beispielsweise haben sich mobile Geräte wie das Handy oder Tablets vermehrt als Endgeräte durchgesetzt und Anwendungen nutzen nicht mehr nur simple (\emph{HTTP-})Anfragen und Antworten. Deutlich sichtbar wird dieses beispielsweise anhand des Features \glqq\emph{Method identifier}\grqq. CSIC2010 definiert hier nur die Methoden \verb=GET= und \verb=POST= als mögliche Werte für erwünschten Datenverkehr. Explizit erscheint die \verb=PUT=-Methode im anomalen Teil. Natürlich ist der Datensatz auf die von Torrano-Giménez genutzte Anwendung zugeschnitten und dürfte auch für die meisten damaligen Web-Anwendungen funktionieren. Nach Definition kann eine WebApplicationFirewall jedoch jeden Datenverkehr über das Hypertext Transfer Protokoll absichern und dazu gehören auch die anderen HTTP-Methoden. Durch das Weglassen der restlichen Methoden könnten bei der Absicherung vieler moderner Anwendungen oder Dienste Probleme auftreten. 

\subsubsection{Das HTTP-Protokoll, REST und WebDAV}
Heute nutzten Endgeräte (wie das Mobiltelefon aber auch Server) zur Kommunikation untereinander vermehrt Dienste die ebenfalls auf dem Hypertext Transfer Protokoll basieren. Beispielsweise bieten \emph{Representational State Transfer}~(REST)-konforme Dienste einfache Möglichkeiten der Kommunikation zwischen mehreren (verteilten) Maschinen bzw. Anwendungen an. Auch solche Schnittstellen lassen sich mit Hilfe von WebApplicationFirewalls absichern.\\

% \lstset{language=XML,
% 	basicstyle=\ttfamily\color{black}\small,
% 	keywordstyle=\bfseries\color{bhtBlue},
% 	identifierstyle=\color{black}, 
% 	commentstyle=\color{gray}\textsl
%      }
\begin{figure}[h]
  \begin{lstlisting}
PUT /user/devtty HTTP/1.1
Host: localhost:8080
Content-Type: application/json

{
  "id": 2334,
  "username": devtty,
  "firstName": "Denis",
  "lastName": "Renning"
}
\end{lstlisting}
  \caption{Beispiel für einfachen REST-Request}
  \label{fig:restputexample}

\end{figure}

Eine anomalie-basierende WAF die mit dem CSIC2010-Datensatz trainiert worden wäre, würde den in Abbildung \ref{fig:restputexample} dargestellten Request ablehnen, da das Feature \glqq\emph{Method identifier}\grqq die Methode \verb=PUT= nicht erlaubt. Für die Klassifizierung bei REST-konformen Endpunkten müssten noch weitere Methoden im Datensatz aufgenommen werden.

Möglich wäre zum Beispiel auch die Absicherung eines WebDAV\footnote{Web-based Distributed Authoring and Versioning}-Endpunktes. Der Datensatz müsste dann um die entsprechenden Methoden (siehe Tabelle~\ref{tab:httpmethods} im Anhang) erweitert werden.

\subsubsection{Neue Werkzeuge}
\label{sec:neuewerkzeuge}

Grundsätzlich sollte auch bedacht werden, dass seit der Erstellung des Datensatzes von Torrano-Giménez, die genutzten Werkzeuge ebenfalls weiterentwickelt wurden und verbesserte Versionen erschienen. Zum Beispiel wurden zur Erzeugung der Anomalie-Requests die Werkzeuge \emph{Paros}\footnote{\url{https://sourceforge.net/projects/paros/} abgerufen am 29.06.2023} und \emph{w3af}\footnote{\url{https://w3af.org} abgerufen am 29.06.2023} eingesetzt. Die Entwicklung des eigentlichen Paros-Proxies wurde 2006 eingestellt, jedoch übernahm das \emph{Open Web Application Security Project} die Quellen und entwickelte einen Fork der Software unter dem Namen \emph{Zed Attack Proxy}~(ZAP) bis heute weiter. Höchstwahrscheinlich wäre der Umfang der Anomaliedaten wesentlich höher wenn der damalige Testaufbau (mit den aktuelleren Versionen der Software) wiederholt wird und der Generierungsprozess nochmals angestoßen wird.
  
\subsection{Ziel des Datensatzes}
\label{sec:zieldesdatensatzes}

Ein weiterer Kritikpunkt Applebaums am Datensatz CSIC2010, neben dem Alter, betrifft das zur Erstellung genutzte Ziel. Für CSIC2010 wurde nur eine einzelne Webanwendung genutzt, die intern vom CSIC2010-Team entwickelt wurde und rudimentäre Funktionen eines e-Commerce-Systems anbietet~\cite{csic2010}. Es existieren eine Nutzerregistrierung und diverse Funktionen aus dem Einkaufsbereich (z.B. Waren einem Einkaufskorb hinzufügen usw.). Applebaum merkt zusätzlich an, dass es sich nicht um ein real verwendetes System handelt. Damit beschränken sich die erstellten Daten nur auf den engen Bereich dieser (einen) Webanwendung. Für die Anwendung in anderen Systemem oder im Echtbetrieb müssen in jedem Fall Anpassungen erfolgen oder eigene Daten erhoben werden. Bestes Beispiel ist bereits die Referenzierung des \verb=localhost= in der URL der CSIC2010-Daten. Eine Anwendbarkeit auf einem entfernten System wäre damit nicht gegeben. Daher bedarf es für eigene Projekte in den meisten Fällen wenigsten einer Anpassung dieses Attributes. 

\subsubsection{Neue Anwendungsarchitekturen und Frameworks}
\label{sec:neueframeworks}
%% Änderungen Anwendungsarchitekturen
Eine wesentliche Änderung seit Erstellung des Datensatzes ist das Aufkommen moderner Anwendungen die durch Nutzung von Anfragen im Hintergrund wesentlich schneller auf Nutzereingaben reagieren. Gute Beispiele sind Eingabefelder mit automatischer Vervollständigung oder die Validierung der eingegebenen Werte während der Eingabe. Das explizite Auslösen einer Anfrage, z.B. über ein Formular, wurde in vielen Fällen vom Konzept der \emph{\textbf{A}synchronen Datenübertragung per \textbf{J}avascript \textbf{A}nd \textbf{X}ML} (AJAX) abgelöst. In gewisser Weise sendet nicht mehr der Endnutzer (als Person) die Anfragen an den Server sondern der Browser selbst zu vordefinierten Ereignissen. Bei den Anfragen handelt es sich immer noch um HTTP-Requests, jedoch ist deren Komplexität und Häufigkeit deutlich gestiegen (Abbildung~\ref{fig:ajax}).

\begin{figure}[h]
  %% \begin{center}
  \centering
  \includesvg[inkscapelatex=false,width=0.7\textwidth]{Ajax-vergleich}
  \caption{Das Modell einer traditionellen Web-Anwendung (links) im direkten Vergleich mit einer Ajax Web-Anwendung (rechts)~\url{https://commons.wikimedia.org/wiki/File:Ajax-vergleich.svg}}
  \label{fig:ajax}
%%  \end{center}
\end{figure}

\anno{SVG Darstellung korrekt?}

Mit der vermehrten Nutzung des AJAX-Konzeptes entwickelten sich auch die zur Oberflächengestaltung genutzten Frameworks dementsprechend weiter und verbreiteten sich schnell. Ebenso wie \emph{Applebaum} kritisierten \cite{kozik2019} den CSIC2010-Datensatz dementsprechend. Ihre Kritik galt insbesondere dem \glqq\emph{Content-Type}\grqq-Feld:\\

\textcolor{bhtGray}{\ding{110} Kozik über den Datensatz CSIC2010~\cite{kozik2019}} For instance, the original data set currently contains mainly requests that are complied with \glqq\verb=application/x-www-form-urlencoded=\grqq{} 
encoding standard. On the other hand, the modern web applications are currently also using different types of request content, e.g. \glqq\verb=application/json=\grqq, \glqq\verb=text/x-gwt-rpc=\grqq or other proprietary types.\\

In ihrer Arbeit testeten sie gegen eine GWT\footnote{Google Web Toolkit}-basierte Anwendung und erweiterten für ihre Zwecke, darauf aufbauend, den CSIC2010-Datensatz um 1500 \emph{normale} bzw. gültige GWT-Anfragen und 2000 Anfragen mit Angriffs-Absicht.

% SChönes Beispiel für Fortschritt, produktiven Einsatz und zeitliche Veränderung und wo Gimenez nicht funktioniert-- > PUT als anomalie in CSIC2010 definiert aber in REST durchaus üblich
% rest beispiel mit PUT


% Methodik und Vorgehen
\section{Methodik}
Aufbauend auf den Erläuterungen zu den angesprochendenen Problemen und auf der verübten Kritik folgen nun einige Lösungsempfehlungen. Dabei soll auf ein mögliches Endprodukt hingearbeitet werden dass erst im nächsten Kapitel konkrete Form annehmen wird. 
% Uebersicht Architektur

\section{Lösungen}

\subsection{Aus Alt mach Neu}
\label{sec:lsgalt}
Bereits in Abschnitt \ref{sec:neuewerkzeuge} wurde die erneute Erstellung des Datensatzes, mit Hilfe neuerer Versionen der Software, angesprochen. Für den damals genutzten Anwendungsserver und die Software zum Generieren der HTTP-Anfragen sind öffentlich verfügbare Nachfolger bekannt. Leider existiert jedoch keine öffentliche Version der von \cite{Giménez2015} genutzten Anwendung. Eine Wiederholung im Rahmen dieser Arbeit ist daher ausgeschlossen. Die originalen Daten können weiter verwendet werden. Gleiches gilt auch für die in Abschnitt \ref{sec:neueframeworks} erwähnte Erweiterung des Datensatzes aus \cite{kozik2019}. Die Autoren verweisen auf ein geographisches Informationssystem, liefern jedoch keinen konkreten Namen.\\

Für eine Erweiterung des Datensatzes, insbesondere in dieser Arbeit, sollten aus Gründen der Nachvollziehbarkeit nur benannte und frei verfügbare Anwendungen und Quellen in Betracht kommen. 

% Hier beschreiben wie CSIC2010 mit "neuen Mitteln" nachgestellt werden kann; warum das nicht geht (fehlende Webanwendung)... ggf. Ersatzanwendung bzw. Übernahme der Originaldatensätze mit in neuen Datensatz



\subsection{Von der Einzelanwendung zur Mehrgültigkeit}
% Hier beschreiben wie durch Verwendung von Demo und Vulnerable-Anwendung mehrgültigkeit geschaffen werden kann
% und Anwendungen nach gleichem Prinzip arbeiten (Entwickler sich ein Beispiel nehmen usw.)
Im besten Fall wird eine WAF explizit auf die zu schützende Anwendung angepasst in dem der Datenverkehr der Anwendung selbst analysiert wird. Was passiert jedoch wenn die entsprechenden Daten nicht vorliegen oder mehrere ähnliche Anwendungen mit möglichst wenigen zur Verfügung stehenden Daten abgesichert werden sollten? Bereits in der Einleitung (Abschnitt~\ref{sec:einleitungziel}) wurde behauptet dass Anwendungen auch anhand der Daten von ähnlichen Anwendungen abgesichert werden könnten. Doch was sind \glqq\emph{ähnliche}\grqq{} Anwendungen? Grundsätzlich sollte hier unterschieden werden zwischen \emph{funktionell} ähnlichen und \emph{technologisch} ähnlichen Anwendungen.\\

\subsubsection{Funktionale Ähnlichkeit}
Bei funktional ähnlichen Anwendungen handelt es sich um Anwendungen die sich in ihrer Funktion ähneln. Um das Problem der fehlenden Anwendung von Torrano-Giménez aus dem vorherigen Abschnitt \ref{sec:lsgalt}  zu lösen, könnte man beispielsweise anhand der Daten aus dem CSIC2010-Datensatz eine 1:1-Kopie der Anwendung (in ihrer Funktion) \glqq\emph{nachprogrammieren}\grqq. Diese ähnliche Kopie könnte dann genutzt werden um den Datensatz mit Ergebnissen aus den Tests mit neueren Werkzeugen zu erweitern. Zu bedenken ist jedoch dass die Kopie nur anhand der von aussen sichtbaren Funktionalität entstand und (eventuell) nicht hundertprozentig der Funktionalität des Originals entspricht. Man kann aber davon ausgehen dass das Verhalten des Zwillings in Bezug auf die Features (Tabelle~\ref{tab:tgfeatures}) größtenteils dem des Originals entspricht.\\
Für die funktionale Ähnlichkeit bedarf es aber nicht grundsätzlich einer identischen Kopie. E-Commerce-Anwendungen, nach dem von Torrano-Giménez gewählten Muster (\glqq\emph{In this web application, users can buy items using a shopping cart and register by providing some personal information}\grqq{}~\cite{csic2010}), zählen sicher zu den verbreitetsten Anwendungen im Netz und es existieren zahlreiche Implementierungen dieser (\emph{funktional-ähnlichen}) Shopping-Systeme mit entsprechend ähnlichen Funktionen. Ein gutes Beispiel ist der Anwendungsfall der \emph{Anmeldung am System}, der sich in jeder E-Commerce-Anwendung findet. Für eine Anmeldung am System werden in den meisten Fällen zwei Argumente benötigt, der Nutzername und das zugehörige Passwort, und in vielen Fällen sind die Namen der zugehörigen Request-Argumente ebenfalls ähnlich. \anno{hier fehlt was}

\subsubsection{Technologische Ähnlichkeit}
Mit \emph{technologischer Ähnlichkeit} ist der Vergleich der zugrundeliegenden Technologien (Programmiersprachen, Frameworks, usw.) gemeint. Da uns bei Proxy-Systemen wie einer WAF meistens nur der Blick von Aussen möglich ist, ist die Frage nach der \emph{technologischen Ähnlichkeit} in vielen Fällen nicht so einfach zu beantworten. Beispielsweise lässt sich anhand der originalen CSIC2010-Daten nicht feststellen, welche Datenbank die Anwendung von Torrano-Giménez verwendete oder ob überhaupt eine Datenbank im System Anwendung fand. Mit hoher Wahrscheinlichkeit handelte es sich allerdings um eine javabasierte Anwendung, da die URLs der CSIC2010-Daten sowohl den Standard-Port vieler Servletcontainer (\verb=8080=) beinhalten und der Pfadteil in vielen Fällen mit der Dateiendung \verb=*.jsp= (\emph{JavaServerPages}) endete.\\

\textcolor{bhtBlue}{\ding{110} Achtung:} Für einen Angreifer ist die technologische Basis einer Anwendung von immensem Interesse. Ist die technologische Basis erst einmal bekannt, lassen sich Schwachpunkte und Angriffswege wesentlich leichter identifizieren. Aus Sicht der IT-Sicherheit sollten Anwendungen (insbesondere im produktiven Einsatz) es daher vermeiden, Daten preiszugeben die Rückschlüsse auf eingesetzte Technologien und Werkzeuge ermöglichen.\\

Beispiele für Rückschlüsse auf Technologien
\begin{description}
  \scriptsize
  \item[http://localhost:8080/\underline{test.do}] \hfill \\
Der Pfad endet mit \verb=.do= - ein Hinweis auf eine Java-basierte Webanwendung die das \emph{Apache Struts}-Framework nutzt
  \item[http://localhost:8080/test?\underline{execution=e2s1}] \hfill \\
Hier ist ein Request-Parameter mit Namen \verb=execution= und einem Wert im Format \verb=e[\\d]s[\\d]= vorhanden, es handelt sich wahrscheinlich um eine Java-basierte Webanwendung die das \emph{Spring WebFlow}-Framework nutzt
  \item[https://devtty.de/app/detail\underline{.jsf};jsessionid=jtO8peEIzV9npU-\_ByEF-9JTq.\underline{slbs11}?\underline{dswid=3942\&dsrid=213}\&id=1041] \hfill \\
Ein Beispiel einer produktiven, javabasierten Anwendung\footnote{Domain-Name wurde geändert}. Im Request-Header wird als Server ein JBoss/EAP-7 benannt. Die Anwendung nutzt JavaServerFaces (\verb=.jsf=). Durch eine Fehlkonfiguration wird die Session-ID in der URL übermittelt. Bei JBoss/WildFly-Servern wird an die SessionID der Name des echten Hosts angehängt, es existiert also höchstwahrscheinlich ein Host mit dem internen Namen \verb=slbs11.devtty.de=. Die Parameter \verb=dswid= und \verb=dsrid= weisen auf Verwendung des Apache DeltaSpike-Frameworks hin. Im Fall das es sich bei der benutzten DeltaSpike-Bibliothek um eine etwas ältere Version handelt, existiert hier eine Schwachstelle (\href{https://cve.mitre.org/cgi-bin/cvename.cgi?name=CVE-2019-12416}{CVE-2019-12416}) die die Möglichkeit einer JavaScript-Injection bietet. 
\end{description}

Im Fall der Mehrgültigkeit, im dem Sinn dass Trainingsdaten ähnlicher Anwendungen genutzt werden sollten, ist die technologische Ähnlichkeit wichtig da in den HTTP-Anfragen die relevanten Informationen stecken. Ein Teil der Features aus Tabelle~\ref{tab:tgfeatures} stimmt bei technologisch ähnlichen Anwendungen häufig überein. Gemeinsame Schlüsselworte oder auch bestimmte Muster in den Argumenten wiederholen sich sowohl bei technologisch ähnlichen als auch bei funktional ähnlichen Anwendungen. \\

\subsubsection{Beispielanwendungen}
Bei der Entwicklung von webbasierten Anwendungen unterstützen die frei verfügbaren Demo-Anwendungen (Beispiele in Tabelle~\ref{tab:exampleapp} vieler Framework-Anbieter ihre Nutzer anhand praktischer Beispiele. Dabei enstehen in gewisser Weise auch \emph{ähnliche Anwendungen}\footnote{z.T.\emph{Copy'n'Paste}-Code} mit gleicher Technologie, ähnlichen Funktionen und (wahrscheinlich) annähernd gleichen Schwachstellen. Die Demonstratoren sind dadurch ideale Kandidaten zum Generieren von Trainingsdaten, die nicht nur für sich selbst gültige Daten generieren, sondern z.T. auch gültige Daten für ihre \glqq\emph{technologischen Geschwister}\glqq{} liefern könnten. Ein Datensatz der anhand der Beispielanwendungen der Framework-Hersteller entstand, wäre durch die \glqq\emph{Ähnlichkeit}\grqq{} zu Ziel-Anwendungen die auf der gleichen Basis entstanden, für die Trainingsphase wünschenswert.


\subsection{Unzureichende Kategorisierung im Datensatz}
\label{sec:Kategorisierung}
Ein großes Problem vieler Firewalls sind falsch-positive Meldungen. Hier löst die Firewall in der Annahme aus, dass ein Angriff vorliegt obwohl kein wirklicher Grund dafür vorliegt. Gründe dafür wurden bereits in Abschnitt~\ref{sec:rullog} (insbesondere Abbildung~\ref{fig.paranoia}) benannt. Falsch-positive sind aber kein alleiniges Problem der regelbasierten WAFs, auch trainierte Systeme können anfällig für dieses Verhalten sein. Für Betreiber und Nutzer von Anwendungen sind solche Meldungen unerwünscht. Im produktiven Einsatz einer WAF die mit dem CSIC2010-Datensatz trainiert wurde, würde bereits die Verwendung eines anderen Browsers (als den von Torrano-Giménez genutzten) zum \emph{False-Positive} führen, wenn die Firewall jede Abweichung vom (\emph{beaufsichtigt}) Erlerntem als Angriff interpretiert. Wird der Datensatz erweitert, könnten mit der Einführung weiterer Parameter Dateneinträge kategorisiert und spezifiziert werden und damit selektiv zum Einsatz kommen.\\ Beispielsweise erwähnt die Beschreibung der Anwendung in \cite{csic2010} dass Nutzer in der Anwendung registriert werden und Teile der Anwendung nur nach einem Login verfügbar sind. Ein Blick in die Daten zeigt die in Tabelle~\ref{tab:csicsecarea} dargestellte Unterteilung der Anwendung in drei Bereiche und eine entsprechende Absicherung des Zugangs zu den einzelnen Bereichen kann angenommen werden.\\ 

\begin{table}[ht]
  \centering
  \begin{tabular}{lp{7cm}}
    \toprule
    \textbf{URL-Pfad} & \textbf{verfügbar für} \\
    \midrule
    \verb=/tienda1/=\textcolor{bhtBlue}{global}/ & alle Nutzer\\
    \verb=/tienda1/=\textcolor{bhtBlue}{public}/ & nicht eingeloggte Nutzer  \\ 
    \verb=/tienda1/=\textcolor{bhtBlue}{miembros}/ & angemeldete Nutzer  \\
    \bottomrule
  \end{tabular}
  
  \caption{Sicherheitsbereiche in der CSIC2010-Beispielanwendung}
  \label{tab:csicsecarea}
\end{table}

So ersichtlich wie in dieser Anwendung ist der Zustand ob ein Nutzer eingeloggt ist jedoch nicht immer. Es sollte nicht davon ausgegangen werden dass jede Anwendung dieses ähnlich abbildet. Die Anforderung eines URL-Pfades kann durchaus unterschiedliche Antworten erzeugen und das betrifft nicht nur den Fall der Authentifizierung. Eine Anfrage an YouTube sieht beispielsweise immer gleich aus (z.B.: \url{https://youtu.be/KTc3PsW5ghQ}) kann jedoch in Abhängigkeit von verschiedenen Variablen unterschiedliche Ergebnisse liefern (\emph{Authentifizierung, Geocoding, Einstellung zur Privatsphäre, etc.}).
% Problem zu viele Falsch-Positive/Zuordnung:Framework/Status(Session) -> Anwendungsstatus (DeltaspikeClientId/SpringExecuteID)
% möglichkeit anomalien nicht auf basis der Regeln sondern auf basis von abweichungen in der Anwendungsnutzung auszuwerten

% Erweiterung Datenset um Variablen (eingeloggt/framework/id) Gimenez (anhand urlpfad publico) 
% Notwendigkeit auf gegeben das CTGimenez Datensatz auf einer Anwendung basierte und nur gegen diese getestet wurde -> Testen gegen andere Anwendung aber gleicher Art?



\subsection{Zentralisierung}
\label{sec:zentralisierungkern}

Bisher wurde der Datensatz als feste Vorgabe zum Training genutzt. In einem praktischen System wäre es jedoch von Vorteil wenn der Datensatz sich selbständig erweitert und gegebenenfalls auch von anderen, ähnlichen Systemen \glqq\emph{lernen}\grqq{} könnte. In~\cite{Manaseer2018}~erfolgte bereits der Entwurf die Daten von erkannten Angriffen zentral, über ein sogenanntes \emph{Command and Control Center} zu sammeln. Carmen Torrano-Giménez nutzte jedoch zum Trainieren ihrer WAF ausschließlich Daten von erwünschten Requests. Im Gegensatz zu Manaseer wäre es durchaus hilfreich alle Requests an eine zentrale Komponente zu senden und mit Hilfe einer genaueren Kategorisierung (siehe Abschnitt~\ref{sec:Kategorisierung}) könnten diese Daten dann auch dementsprechend ausgewertet und gegebenenfalls an die WAFs ähnlicher Anwendungen verteilt werden.


\begin{figure}[ht]
  \begin{center}
    \includegraphics[width=15cm]{waf_mana}
    \caption{Zentrale Übermittlung von Angriffsdaten nach~\cite{Manaseer2018}}
    \label{fig.wafmana}
  \end{center}
\end{figure}

% Oder: Der xxx Algorithmus

% Oder: Der yyy Algorithmus

% Zusammenfassung: ca. 0,5 Seiten
\section{Zusammenfassung}

Nach der kurzen zeitlichen Einordnung über die Entstehung anomaliebasierter Web Application Firewalls im vorherigen Kapitel lieferte dieses Kapitel einen Einblick in einen Teil der bestehenden Daten. Dazu wurde der Umfang und Inhalt des 2010 von Carmen Torrano-Giménez erstellten Datensatzes \emph{CSIC2010} näher betrachtet. Es wurden einige Modernisierungsmaßnahmen vorgeschlagen, die dem veralteten Datensatz zu etwas mehr Allgemeingültigkeit und vor allem Genauigkeit verhelfen.\\ Des Weiteren wurde ersichtlich dass der bestehende Datensatz keine Allgemeinformel für jeden Anwendungsfall darstellt. Im Gegenteil, jede \emph{praktische} Anwendung erfordert einen spezialisierten Datensatz und Gemeinsamkeiten unterschiedlicher (aber ähnlicher) Anwendungen können durchaus zu allgemeinerer Anwendbarkeit beitragen. Eine zentrale Stelle soll dabei die Daten entgegen nehmen und die Möglichkeit bieten, diese auch noch weiter zu verarbeiten. \\

Das nächste Kapitel betrachtet die praktische Umsetzung der angedachten Ziele. Hierbei wird zuerst auf einige Vorbedingungen eingegangen und im Anschluss eine bestehende Lösung dementsprechend erweitert. Die Lösungen aus diesem Kapitel dienen dabei sozusagen als Anforderungen. Im darauf folgenden Test wird ebenfalls dargestellt, wie ein neuer Datensatz generiert wird bzw. der Bestehende mit zusätzlichen Daten erweitert werden kann.


%Implementierung (max. 5 Seiten)
%  Highlight 1 der Implementierung
%  Highlight 2 der Implementierung
%  Highlight des Deployments beim Kunden
%  Zusammenfassung (ca. 0,5 Seiten)
\chapter{Implementierung}

\anno{max. 5 Seiten (7)} %Ref hat 12

% Was lesen wir in diesem Kapitel?
% Warum muss ich das als Gutachter lesen
% Wie verknüpft sich der Inhalt mit dem vorhergehenden Kapitel?
% Welche Implmentierungsentscheidungen? Welche Alternativen? Vor- und Nachteile des eigenen Ansatzes?

% Highlight 1 der Implementierung
Wie bereits in Abschnitt \ref{sec:Probleme} angemerkt sind offene Implementationen von Web Application Firewalls eher die Ausnahme und selbst Umsetzungen der in Abschnitt \ref{sec:relatedwork} erwähnten Produkte sind praktisch nicht auffindbar. In diesem Kapitel gibt es einen kurzen Überblick wie die Vorschläge aus dem vorhergehenden Kapitel dennoch praktisch umgesetzt werden können.

Eine komplette Umsetzung ohne vorherige Basis wäre zeitlich (von einer einzelnen Person) nicht umsetzbar gewesen, daher sollte ein vorhandenes Open Source Produkt entsprechend erweitert werden. Bei der Auswahl wurden zuerst die zwei (leichtgewichtigen) Lösungen \emph{OWASP ESAPIFilter}\footnote{\url{https://owasp.org/www-project-enterprise-security-api/} abgerufen am 05.07.2023} und {SpringSecurity HttpFirewall}\footnote{\url{https://docs.spring.io/spring-security/reference/servlet/exploits/firewall.html} abgerufen am 05.07.2023} betrachtet. Beide Produkte verstehen sich als Ausgangsbasis für die Entwicklung sicherheitsrelevanter HTTP-(Servlet)-Filter auf Anwendungsebene, sämtliche Funktionalität muss jedoch vom Nutzer erst implementiert werden. Bei der Nutzung wäre ein erhöhter Arbeitsaufwand absehbar. Diogo Sampaio und Jorge Bernardino verglichen 2017 verfügbare WAFs die ebenfalls als Erweiterungskandidaten in Frage kamen (siehe Tabelle \ref{tab:my_vergos}).

% Tabelle ggf. in Anhang verschieben

\begin{table}[h]
  \centering
  \begin{tabular}{lccc} 
    \toprule
    & \textbf{ModSecurity} & \textbf{WebCastellum} & \textbf{IronBee} \\ [0.5ex] 
    \midrule
    Simple filtering & Yes & Yes & Yes \\ 
    Regular expression based filtering & Yes &  & Yes \\
    Auditing & Yes &  & Yes \\
    Null byte attack prevention & Yes & Yes &  \\
    URL Encryption & Yes & Yes &  \\ [1ex] 
    Stateful Attack Detection & & Yes & Yes \\
    \bottomrule
  \end{tabular}
  \caption{Vergleich Open Source Web Application Firewalls (aus \cite{Sampaio2017}) }
  \label{tab:my_vergos}
\end{table}

\emph{ModSecurity} und \emph{IronBee}s Stärken liegen in der regelbasierten Auswertung (\emph{regular expression based filtering}). Beide werden als Reverse-Proxy vor der eigentlichen Anwendung eingesetzt und sind keine HTTP-Servlet-Filter. Dies hat den Nachteil das kein Zugriff auf die eigentlichen Anwendungsdaten besteht und nur anhand des HTTP-Datenverkehrs gefiltert werden kann. \emph{WebCastellum} scheint hier Gutes aus beiden Welten bereit zustellen, zum einen (als Servlet-Filter) direkten Zugriff auf die Anwendungsdaten anzubieten, als auch regelbasiertes Filtern \emph{out-of-the-box}. \anno{Achtung! Sampaios Tabelle ist hier fehlerhaft! Übernehmen trotz Fehler?}
Mit Hinsicht auf einen späteren produktiven Einsatz wurde daher die Software \emph{WebCastellum} als Grundlage für die Implementierung der Firewall-Funktionalität gewählt.

\section{Vorarbeiten}

Leider haben WebCastellum und der CSIC2010 Datensatz eine Gemeinsamkeit - das Alter. Die derzeit aktuelle offizielle Version 1.8.3 von WebCastellum erschien bereits im Jahr 2009 und eine Weiterentwicklung fand seit dem nicht statt. Dennoch befindet sich das Produkt weiter in Verwendung und wird wohl auch in neuen Anwendungen genutzt, lässt zumindest die aktuelle Downloadstatistik (siehe \ref{fig:downloadwc}) vermuten. Insgesamt wurde das Binärpaket von WebCastellum in der Version 1.8.3 seit 2009 mehr als 8500 mal heruntergeladen (Stand vom 05. Juli 2023) - nicht berücksichtigt sind alle Vorgängerversionen und die Downloads die über Repository-Dienste, wie z.B. Maven Central, stattfanden.

\begin{figure}[h]
  \centering
  \begin{gnuplot}[terminal=png,scale=.7]
    set datafile separator ','
    set xdata time
    set timefmt "%Y-%m"
    set xrange ["2009-01":"2023-09"]
    set format x "%b %y"
    set key autotitle columnhead
    plot '04_Implementierung/downloads.csv' using 1:2 with boxes fs solid 1.0 fc 'steelblue'
  \end{gnuplot}
  \caption{Download-Statistik SourceForge WebCastellum 1.8.3 binary}
  \label{fig:downloadwc}
\end{figure}

Vor den eigentlichen Arbeiten waren also noch ein paar kleinere Vorarbeiten notwendig. In einem ersten Schritt wurde der Quellcode aus dem SourceForge-Repository heruntergeladen und in ein eigenes git-Repository\footnote{\url{https://github.com/devtty/webcastellum/} abgerufen am 05.07.2023} übertragen. 

\subsection{Version 1.8.4. - Review des Quellcode und IT-Sicherheit}
Während die Binär-Distribution mit der Versionsnummer 1.8.3 verteilt wird, sind die Quellen bereits mit der nächsten Minor-Version versehen. Diese lassen sich auch fast problemlos kompilieren und zu einem verwendungsfähigen Paket zusammenbauen. Das Projekt hat nur sehr wenig Abhängigkeiten zu weiteren Bibliotheken und diese (laut durchgeführtem OWASP dependency-check) keine bekannten Schwachstellen. Etwas problematisch ist die Abhängigkeit zu älteren Bibliotheken (\emph{javax.jms} und \emph{javax.mail}), da diese nicht mehr (direkt) über den Standard-Build-Mechanismus verfügbar sind, sich aber über Umwege auftreiben lassen.

Im weiteren Verlauf wurden die Quellen noch an Githubs eigenes CI/CD-System angebunden und automatisiert einer Codeanalyse auf der Sonarcloud-Plattform \footnote{\url{https://sonarcloud.io/project/overview?id=devtty_webcastellum} abgerufen am 05.07.2023} unterzogen. Insgesamt fielen dabei im Quelltext sechs direkte Schwachstellen und 58 potentielle Schwachstellen (siehe \ref{fig:my_sonar1}) auf. Bei den direkten Schwachstellen handelt es sich um:

\begin{itemize}
    \item 3 x Verwendung schwacher Kryptographie (hier AES; Cryptographic Failures / Security Misconfiguration )
    \item 2 x Ausgabe der Session-ID in Log-Dateien (Insecure Design/Broken Authentication)
    \item 1 x XML External Entity Injection Attack (Security Misconfiguration / XML External Entities (XXE))
\end{itemize}

Wobei die XXE-Attacke von Sonar als Blocker eingestuft wird. Bei den 58 Security Hotspots handelt es sich meistens um direkte Ausgaben im Fehlerfall (\verb=printStackTrace()=) mit der Bitte zur Überprüfung das diese im Produktivbetrieb nicht ausgegeben werden. 

Inklusive Code Smells bemängelte Sonar den Quellcode in insgesamt 2657 Fällen und berechnet die angehäuften technischen Schulden auf 52 Personen-Tage Nacharbeit. 

\begin{figure}
    \centering
    \includegraphics[width=\textwidth]{first_sonar2.jpg}
    \caption{SonarCloud erste Sichtung}
    \label{fig:my_sonar1}
\end{figure}

\begin{figure}
    \centering
    \includegraphics[width=\textwidth]{first_sonar_sql.jpg}
    \caption{SonarCloud SQL-Injection}
    \label{fig:my_sonar2}
\end{figure}



\subsection{ToDos, Unit-Tests,  etc.}

Interessant war ebenfalls die Suche nach \verb=TODO=-Markierungen im Quelltext. Insgesamt 243 mal traten entsprechende Kommentare im Quelltext auf und markierten Stellen an denen die Entwickler Probleme und Verbesserungen vermuteten bzw. Aufgaben für die nächste Version notierten. Diese Notizen reichten von einfachen Aufgabenstellungen, wie \glqq\emph{add proper logging}\grqq, über Refactoring-Ideen bis hin zu Anmerkungen deren Umsetzung die Funktionsweise des Filters beeinflussen würden. \anno{Zitierung aus eigener unveröffentlichter notwendig?}

Einige der Todo-Anmerkungen ließen auf eine geplante Überführung des Quellcodes von Java in der Version 1.4 auf Java 5 schließen. Häufig fanden sich auch auskommentierte dementsprechende Fragmente und, insbesondere bei der Verwendung listenähnlicher Objekte, fallen die in Java 5 eingeführten \emph{Generics} (s. Abbildung \ref{fig:my_l2} auf.

\begin{figure}[h]
  \begin{small}
    \begin{lstlisting}[language=java]
    // TODO: in Utility-Klasse packen (ServerUtils)
    private static void concatenateParameterMaps(
      final Map/*<String,String[]>*/ parameters, 
      final Map/*<String,String[]>*/ parameterMapToAdd) {
    \end{lstlisting}
  \end{small}
  \caption{Auszug aus WebCastellumFilter als Beispiel für ToDos und auskommentierte Generics}
  \label{fig:my_l2}
\end{figure}
  
Dabei sollte berücksichtigt werden, dass diese Kommentare aus dem Jahr 2009 stammen und auch die Programmiersprache Java in den letzten Jahren deutliche Fortschritte gemacht hat. Die Sprache erlaubt mittlerweile auch Konstrukte wie Lambda-Expressions (Java 8) und Typinferenz (Java 10). Ein Refactoring des Quelltextes diesbezüglich wäre sicher lohnenswert.

Tests, im Sinne von Unit-Tests z.B. im Rahmen eines etablierten Testframeworks, lassen sich in den Quellen der Version 1.8.3 nicht finden. Für eine Software, die zur Sicherheit von Anwendungen beitragen sollte, eher ungewöhnlich. Stattdessen existieren in einigen Klassen \verb=main=-Methoden mit entsprechender Absicht. Die Klasse \verb=LargeFormPostRequestTester= dient beispielsweise ausschließlich Testzwecken. \verb=ResponseUtils= behinhaltet gleich fünf auskommentierte \verb=main=-Methoden zum Testen verschiedener Funktionalitäten inklusive der Anmerkung \emph{for local testing only}. 


\subsection{Fehlerbereinigung}

Die minimale (persönliche) Anforderung an das Ziel der Vorarbeiten war den seit 2009 nicht bearbeiteten Quelltext auf eine fehlerfreie Basis anzuheben, ein (kompilierbares) nutzbares Endprodukt zu erhalten und mindestens einen Anwendungsfall eines Angriffs komplett durchtesten zu können.

Umgesetzt wurde dann ein leichtes Refactoring und die Erstellung von 222 Unit-Tests\footnote{\url{https://github.com/devtty/webcastellum/actions/runs/5441412243/jobs/9895332455\#step:8:437} aufgerufen am 07.07.2023}. Dabei wurden:

\begin{itemize}
    \item 33 Bugs beseitigt
    \item 6 Sicherheitslücken beseitigt
    \item 54 Security-Hotspots überprüft
    \item über 300 Code-Smells beseitigt
    \item und die Testabdeckung des Quelltextes von 0 auf 18,8 Prozent angehoben.
\end{itemize}

Zwei Testklassen (\verb=WebFilterDroneTest= und \verb=WebFilterHttpClientTest=) mit insgesamt 14 Tests spielen dabei den Anwendungsfall eines Angriffs per SQL-Injection auf eine kleine Beispiel-Anwendung durch und überprüfen die korrekte Funktionalität des Filters. Mit Hilfe der Arquillian-Erweiterungen \emph{Drone} und \emph{Graphene} wird im Fall der ersten Testklasse ein Browser instantiiert und der Angriff automatisiert über diesen durchgeführt. Im zweiten Fall geschieht dieses, mit Hilfe eines leichtgewichtigen Clients, direkt über das HTTP-Protokoll.

\begin{figure}[h]
    \centering
    \includegraphics[width=\textwidth]{scfinalc.png}
    \caption{Ergebnis des Bugfixings (Juni 2023)}
    \label{fig:my_sonarf}
\end{figure}

Mit der Möglichkeit der Überwachung der qualitativen Metriken der Anwendung und der erweiterten Testabdeckung können die Arbeiten zur Erweiterung der Web Application Firewall beginnen. Hierzu zählen insbesondere der Datenaustausch zwischen verschiedenen Instanzen über eine zentrale Komponente und die (hoffentlich) effektivere Angriffserkennung.

% Highlight 2 der Implementierung
\section{Zentralisierung}
Im Ursprungszustand handelt es sich bei WebCastellum um einen relativ leichtgewichtigen Servlet-Filter der sämtliche Abwehrmaßnahmen direkt verarbeitet. Abgesehen von den Interaktionen mit der eigentlichen Anwendung existieren nur wenige Berührungspunkte mit externen Systemen. Die zwei wichtigsten Fälle sind dabei die \emph{Konfiguration} der Anwendung und die \emph{Benachrichtung} des Betreibers über den Zustand (Funktion, Angriffe, usw.) der Web Application Firewall. Beide Fälle benutzen derzeit textuelle Schnittstellen. Die Konfiguration erfolgt über \emph{manuelles} Editieren des Bereitstellungsdeskriptors (\emph{web.xml}) und die Bereitstellung des Status über Logdateien. Um den Datenaustausch mehrerer Instanzen der WAF nach dem Muster aus Abschnitt~\ref{sec:zentralisierungkern} zu ermöglichen, müssten zusätzliche (maschinell lesbare) Schnittstellen geschaffen werden. Diese Schnittstellen sorgen dann für den Datenaustausch mit einer zentralen Schnittstelle (Abbildung~\ref{fig:my_verbund}).

\begin{figure}[h]
    \centering
    \includegraphics[width=8cm]{central.png}
    \caption{Arbeit der WAFs im Verbund}
    \label{fig:my_verbund}
\end{figure}

Glücklicherweise haben die Entwickler der WebCastellum-Software entsprechende Schnittstellen (nun im Sinne der objektorientierten Programmierung) mitgeliefert. Das System bietet mit dem Interface \verb=org.webcastellum.ConfigurationLoader= sozusagen bereits eine technische Blaupause für die Implementierung einer alternativen Konfigurationsmöglichkeit. Tatsächlich finden sich in der Software bereits zwei Implementierungen des \emph{Interfaces}:

\begin{description}
  \small
\item[DefaultConfigurationLoader] \hfill \\
  lädt die Konfiguration, wie zuvor beschrieben, aus Parametern im Bereitstellungsdeskriptor (\emph{web.xml})
\item[PropertiesFileConfigurationLoader] \hfill \\
  lädt die Konfiguration aus einer benannten Datei im \emph{properties}-Format mit der Möglichkeit bei fehlenden Werten auf die Daten der \emph{web.xml} zurückzufallen.
\end{description}

Die Klasse \verb=org.webcastellum.ConfigurationManager= implementiert als sogenannter \emph{Service-Locator}\footnote{s. \url{https://martinfowler.com/articles/injection.html} abgerufen am 14.08.2023} eine einfache Möglichkeit zwischen den beiden Implementationen zu wechseln.\\
Eine zusätzliche Implementation der Schnittstelle, wie der in Abbildung~\ref{fig.impkonfig} dargestellte \verb=RemoteConfigurationLoader=, könnte sich die Konfigurationsparameter über beliebige Quellen, zum Beispiel in Form eines REST-API-Clients/MicroServices, \glqq\emph{besorgen}\grqq. 

\begin{figure}[h]
  \begin{center}
    \includegraphics[width=14cm]{configclasses}
    \caption{Implementierung Konfiguration}
    \label{fig.impkonfig}
  \end{center}
\end{figure}

Wenn die Daten zur Konfiguration der Web Application Firewall von einem zentralen Dienst bezogen werden, muss dieser seinen \emph{Kunden} auch identifizieren können. Beispielsweise über einen eindeutigen, mitgelieferten Parameter (z.B. eine UUID - Universally Unique Identifier). Im echten Produktiveinsatz sollte eine genauere Absicherung des Abrufs von Konfigurationsparametern bedacht werden, da diese auch sicherheitsrelevant sind.\\

\anno{ggf. Bsp. Quelltext f. REST-Aufruf einfügen?}
Die Schnittstelle berücksichtigt jedoch nur den Abruf von Konfigurationen, mehr ist für eine einzelne Instanz der Firewall auch nicht notwendig. Im Verbund sollten Konfigurationsparameter anpassbar sein und die Möglichkeit bestehen diese auch zu speichern. 



\begin{figure}[h]
  \begin{center}
    \includegraphics[width=12cm]{classp}
    \caption{Implementierung Nachrichtenversand}
    \label{fig.impversand}
  \end{center}
\end{figure}

\section{Lernmodus}



\section{ML-Fähigkeit}

Die Implementierung der ML-Fähigkeit erfolgt nach dem Vorschlag aus \cite{kozik2014}.

Aus den Testdaten werden dafür nur die Attribute der URL, der HTTP-Methode und die Payload (die Requestparameter) genutzt. 

Klassifizierung anhand der J48-Implementierung eines Entscheidungsbaums nach dem C4.5-Algorithmus

\begin{figure}[h]
    \centering
    \includegraphics[width=9cm]{webcastellumcentral.png}
    \caption{Komponentendiagramm für zentrale Klassifizierung}
    \label{fig:my_future}
\end{figure}


% Highlight des Deployments beim Kunden

% Zusammenfassung: ca. 0,5 Seiten
\section{Zusammenfassung}

% Was haben wir in diesem Kapitel gelernt?
% Wie passt das zur Zielstellung der Arbeit?
% Wie passt das zum nächsten Kapitel?

%Evaluierung (wenn Sie ein System bauen, ca. 10 Seiten)
%  Aufbau der Messumgebung (1-2 Seiten)
%  Ergebnisse und Beobachtungen (3-4 Seiten)
%  Diskussion und Bewertung (3-4 Seiten)
%  Zusammenfassung (ca. 0,5 Seiten)
\chapter{Evaluierung}

\anno{ca. 10 Seiten (4)} %Ref. hat 5

Nachdem im letzten Kapitel hauptsächlich Aspekte der praktischen Umsetzung für das WAF-System erläutert wurden, widmet sich dieses Kapitel der Überprüfung der gemachten Annahmen aus der Zielbeschreibung der Arbeit. Insbesondere die Frage nach möglichst aktuellen bzw. neuen Daten und der damit verbundenen Qualität soll damit beantwortet werden. Dazu werden zuerst die benutzten Werkzeuge und der Testaufbau beschrieben. 

\section{Werkzeuge}
Für den Testaufbau werden einige Grundkomponenten benötigt. 

\subsection{Die Laufzeit- und Entwicklungsumgebung}
Für den Betrieb der in dieser Arbeit verwendeten Werkzeuge, wie z.B. Anwendungsserver oder Buildtools, ist eine Java-Laufzeitumgebung notwendig. In den lokal ausgeführten Fällen wurde dafür das \emph{Java Development Kit} der Firma Oracle in der Version 17.0.1 verwendet. Die im CI/CD-Build-Cycle ausgeführten Tests und Werkzeuge nutzen hingegen das Eclipse Temurin JDK in der jeweils aktuellsten Version 11.x


\subsection{OWASP\textregistered  Zed Attack Proxy}
\label{ref:zap}

\emph{The world's most widely used web app scanner}\footnote{https://zaproxy.org}\\

Der Zed Attack Proxy (ZAP) wird genutzt um Web Anwendungen auf Sicherheitslücken zu untersuchen. Als sogenannter Proxy befindet sich diese Anwendung zwischen dem Endnutzer bzw. Browser und der zu testenden Anwendung. ZAP bietet dabei zwei verschiedene Anwendungsmodi an. Im passiven Modus \emph{begleitet} ZAP den Nutzer auf seinem Weg durch eine Anwendung und registriert sämtliche IT-sicherheitstechnischen Auffälligkeiten. Der Nutzer klickt sich durch seine Anwendung, ggf. auf einem vorher definierten Weg,  und erhält am Ende eine Übersicht möglicher Angriffswege. Des Weiteren kann dieses Werkzeug zur Aufzeichnung der HTTP-Anfragen und -Antworten genutzt werden\\ Im aktiven Modus übernimmt ZAP selbst die Rolle des Angreifers und versucht Sicherheitslücken automatisch zu finden. \\

Es existieren zahlreiche weitere Tools mit ähnlicher Funktionalität, z.T. auch mit Spezialisierung auf das Testen von Web Application Firewalls. Nennenswerte Beispiele sind \emph{BurpSuite}, \emph{WAFNinja}, \emph{gotestwaf} und \emph{w3af}.  

\subsection{Web Server}

Zur Auslieferung der verwendeten Webapplikationen wird ein Web Server benötigt. Bedingt aus den Anforderungen der zu testenden Webanwendungen und Web Application Firewall muss dieser die Java Servlet Spezifikation implementieren. Bei der Verwendung eines lokalen Web Servers oder in den automatisierten Tests dieser Arbeit wird dabei der Apache Tomcat-Server in der Version 8.5.78 genutzt. Die Ausführung von Anwendungen die auf dem \emph{Spring}-Framework basieren benötigen nicht zwingend einen dedizierten Anwendungsserver, da diese ein integriertes System mit sich bringen (Undertow 2.2.18).

\subsection{Die Web Application Firewall}
Bei WebCastellum handelt es sich um eine Web Application Firewall die zum Schutz von Webanwendungen vor allgemeineren Angriffsszenarien entwickelt wurde. Implementiert wurde WebCastellum als ServletFilter, d.h. im Gegensatz zu WAF-Lösungen die als Proxy vor der Anwendung filtern, hätte WebCastellum vollen Zugriff auf alle Informationen die auch der Webanwendung bekannt sind. Erwähnenswert wären hier zum Beispiel der Login- oder Sessionstatus. 


\subsection{Ziel - Anwendungen}

\ref{sec:zieldesdatensatzes}

\subsubsection{PrimeFacesShowcase}

Beim PrimeFaces Showcase\footnote{Download unter https://www.primefaces.org/downloads} handelt es sich um eine Web-Anwendung zur Demonstration der PrimeFaces-Komponenten-Bibliothek. Dabei handelt es sich um eine Sammlung an Komponenten entsprechend der JSF-Spezifikation (JavaServerFaces). Neben der einfachen Darstellung der Komponenten liefert der Showcase auch Quelltext-Beispiele für deren Verwendung. Geht man davon aus dass eine Mehrheit der Entwickler diese Beispiele für eigene Implementierungen nutzt, könnte man schlussfolgern dass zahlreiche Anwendungen sicherheitstechnisch ähnlich reagieren wie der Showcase und dieser stellvertretend hier für den Nachweis genutzt werden kann.\anno{zu kompliziert?} Des weiteren bietet der Showcase die Möglichkeit \emph{alle} Komponenten des Frameworks zu testen.

\subsubsection{PetStore}

\subsubsection{CargoTracker}

\subsubsection{Spring PetClinic}



\subsubsection{OWASP\textregistered WebGoat}

In Form eines IT-Sicherheitstutorials wurde mit dem WebGoat-Projekt eine \emph{absichtlich, unsichere Anwendung} geschaffen, die es \emph{interessierten Entwicklern erlaubt häufig vorkommende Schwachstellen in Java-basierten Anwendungen} zu testen \cite{owaspgoat}. Da es sich um ein Tutorial handelt sind alles Schwachstellen auch explizit benannt. Ein guter Scanner sollte in diesem Fall auch alle Schwachpunkte auffinden können. 


\section{Aufbau der Testumgebung}
% Aufbau der Messumgebung (1-2 Seiten)
%% Server/Betriebssystem
%% Datensätze
%% Anfragen
%% Systeme/Ansätze gegen die Sie sich vergleichen
%% Wie messen Sie? Methodik und Maßeinheiten?
%% Ist die Messung signifikant?
%% Hypothesen? Was erwarten Sie?

Sämtliche Tests in dieser Arbeit werden nach dem Muster aus Abbildung \ref{fig.pattern} aufgebaut. Ein \glqq Angreifer\grqq{} versucht einen Angriff auf das \glqq Opfer\grqq. Zum Vergleich folgt auf einen Test ohne \glqq Verteidiger\grqq{} immer ein gleicher Test mit Verteidiger.

\begin{figure}[bht]
  \begin{center}
    \includegraphics[width=10cm]{pattern}
    \caption{Muster für Tests}
    \label{fig.pattern}
  \end{center}
\end{figure}

\subsection{Attacker}
Angreifer können (handgeschriebene) Tests oder automatisierte Tools sein. Konkret werden in dieser Arbeit folgende Ausprägungen angewandt.

\subsubsection{Der manuelle Angriff}

Ein manueller Angriff ist ein exakt definierter Angriff der manuell ausgeführt wird. Der Ablauf ist dabei exakt in einer (Test-)Spezifikation beschrieben. Häufig existieren diese Tests in Form von Checklisten die vom einem Tester (Mensch) ab zuhaken sind. Mit Werkzeugen wie Selenium existieren auch Möglichkeiten diese Art von Test zu automatisieren. Der hier verwendete Test ruft im Grunde nur ein Formular auf und trägt in ein Textfeld einen mit einer SQl-Injektion versehenen bzw. manipulierten Wert ein. Im Anschluß wird das Formular abgesendet und der Rückgabewert kontrolliert.

\begin{neu}
manuellen Angriff beispielhaft erläutern; ggf. nochmal hinweis auf GrapheneTest (automatisierte Angriff); Hinweis/Übergang zu verschiedenen Beispiel-Datasets bzw. ref aus WAf-a-mole bzgl. verfügbarkeit
\end{neu}


\subsubsection{Der passive Angriff}
Im passiven Angriff wird der bereits aus Abschnitt \ref{ref:zap} bekannte \emph{Zed Attack Proxy} zwischen den Browser und der Webanwendung geschaltet. Im Grunde klickt sich nun ein Anwender durch die Anwendung und der Proxy ließt den Datenverkehr mit. Dabei werden die entdeckten Sicherheitslücken markiert. Um vergleichbare Ergebnisse zwischen den beiden Testläufen (mit und ohne aktiver Web Application Firewall) zu erhalten ist es zwingend erforderlich das beide Testläufe im Ablauf absolut gleich aufgebaut sind. Hierbei hilft beispielsweise die Definition des Testszenarios mittels Ablaufplan. Der Ablaufplan hilft auch bei einer späteren Automatisierung dieses Testvorgehens.

\subsubsection{Der aktive Angriff}
Beim aktiven Angriff übernimmt \emph{ZAP} die volle Kontrolle und sucht (möglichst) selbständig nach potentiellen Schwachstellen in der zugewiesenen Webanwendung. Eine \emph{kleine} Grundkonfiguration ist trotzdem notwendig, beispielsweise um Passworte zu hinterlegen falls ZAP - als Angreifer - auf eine Loginseite trifft. \\\\
\textcolor{bhtGray}{\ding{110} Hinweis} Ist man nicht Eigentümer der Ziel-Webanwendung sollte OWASP ZAPs aktiver Scanmodus nicht genutzt werden!\\


\subsection{Defender}

Hierbei handelt es sich um die zu testende Instanz der WebApplicationFirewall. Gegebenenfalls kann hier natürlich die Konfiguration - je nach Testfall - variieren. Da jedoch eine zentrale Installation im Fokus steht sollte in jedem Fall diese zentrale Instanz getestet werden.
% 3 Fälle -> ohne, defaultconfig, von zentrale gelieferte Konfiguration

\subsection{Victim}

Bei den \glqq\emph{Opfern}\grqq  im Test sollte es sich um zu schützende Anwendungen handeln. Vornehmlich sollte es sich um ein möglichst breit gefächertes Angebot an Anwendungen handeln. Aus zeitlichen Gründen werden in dieser Arbeit jedoch nur die im Grundlagen-Absatz erwähnten Anwendungen, \emph{Primefaces Showcase} und \emph{WebGoat} genutzt. Grundsätzlich könnten aber auch andere Anwendungen für den Testaufbau genutzt werden.\\ 

\subsection{TestMatrix}
Grundsätzlich besteht Bedarf möglichst viele Szenarien im Test abzudecken. Hierfür wird die in Tabelle \ref{tab:testplan} als Grundgerüst verwendet. Möglichst alle Angreifer sollten jeweils jedes Opfer angreifen. Mit steigender Anzahl an Opfern bzw. Tätern würde sich die Matrix entsprechend erweitern.

\anno{Tabelle müssen aktualisiert werden}
\begin{table}[h]
    \centering
    \begin{tabular}{cccc} 
      \toprule
    \textbf{Testlauf} & \textbf{Angreifer} & \textbf{Verteidiger} & \textbf{Opfer} \\ 
     \midrule
     1 & manuell & - & PF showcase\\
     2 & manuell & WebCastellum & PF showcase \\
     3 & ZAP (passiv) & - & PF showcase\\
     4 & ZAP (passiv) & WebCastellum & PF showcase \\
     5 & ZAP (aktiv) & - & PF showcase\\
     6 & ZAP (aktiv) & WebCastellum & PF showcase \\
     7 & manuell & - & WebGoat \\ 
    8 & manuell & WebCastellum & WebGoat \\
    9 & ZAP (passiv) & - & WebGoat \\ 
    10 & ZAP (passiv) & WebCastellum & WebGoat \\
    11 & ZAP (aktiv) & - & WebGoat \\ 
    12 & ZAP (aktiv) & WebCastellum & WebGoat \\
   \bottomrule
    \end{tabular}
    \caption{Testplan}
    \label{tab:testplan}
\end{table}

% Ergebnisse und Beobachtungen (3-4 Seiten)
\section{Ergebnisse und Beobachtungen}

% Beschreibung der Ergebnisse
% Diagramme
% Darstellen von Zusammenhängen

\anno{Tabellenwert! webgoat is spring ML lernen}

%lohnt es sich Training mit altem und neuen Datensatz zu testen? Zeit?

Die Ergebnisse aus den Tests (nach der spezifizierten Testmatrix) zeigen eine eindeutige Verbesserung durch Nutzung der Web Application Firewall. 

\begin{table}[h]
    \centering
    \begin{tabular}{cccccc} 
      \toprule
    \textbf{Testlauf} & \textbf{Angreifer} & \textbf{Verteidiger} & \textbf{Opfer} & \textbf{Schwachstellen} & \textbf{Verbesserung(in \%)} \\ 
     \midrule
     1 & manuell & - & PF showcase & 3 &\\
     2 & manuell & WebCastellum & PF showcase & 1 & 66\\
     3 & ZAP (passiv) & - & PF showcase & 16 &\\
     4 & ZAP (passiv) & WebCastellum & PF showcase & 2 & 87.5 \\
     5 & ZAP (aktiv) & - & PF showcase & 56 & \\
     6 & ZAP (aktiv) & WebCastellum & PF showcase & 20 & 64.3\\
     7 & manuell & - & WebGoat & 10 & \\ 
    8 & manuell & WebCastellum & WebGoat & 3 & 70 \\
    9 & ZAP (passiv) & - & WebGoat & 25 & \\ 
    10 & ZAP (passiv) & WebCastellum & WebGoat & 12 & 50\\
    11 & ZAP (aktiv) & - & WebGoat & 34 & \\ 
    12 & ZAP (aktiv) & WebCastellum & WebGoat & 14 & 59 \\
   \bottomrule
    \end{tabular}
    \caption{Ergebnisse}
    \label{tab:tes1tergebnisse}
  \end{table}

  Tabelle \ref{tab:tes1tergebnisse} gibt einen Überblick über die allgemeine Frage ob Angriffe verhindert werden können. Grundsätzlich verbessert der Einsatz einer WAF die IT-Sicherheit signifikant. Die zweite Tabelle verdeutlicht die Ergebnisse nochmals anhand der einzelnen Kategorien.

\begin{table}[h]
    \centering
    \begin{tabular}{cccccc} 
      \toprule
    \textbf{Testlauf} & \textbf{SQLi} & \textbf{CSRF} & \textbf{XMLi} & \textbf{sonstige} & \textbf{gesamt} \\ 
     \midrule
     1 & 2 & 1 &  &  & 3\\
     2 &   & 1 &  &  & 1\\
     3 & 6 & 5 &  & 5 & 16\\
     4 &   & 1 &  & 1  & 2 \\
     5 & 22 & 9 & 7 & 8& 56\\
     6 &  & 7 & 5 & 8 & 20 \\
     7 & 7 & 3 &  &  &  10  \\ 
    8 &  & 3 &  & & 3  \\
    9 & 11 & 5 &  & 9 & 25 \\ 
    10 &   & 5 &  & 7  & 12 \\
    11 & 17 & 8 &  & 9 & 34  \\ 
    12 &  & 8 &  & 6 & 14 \\
   \bottomrule
    \end{tabular}
    \caption{Ergebnisse}
    \label{tab:tes2tergebnisse}
\end{table}

Betrachtet man Tabelle \ref{tab:tes2tergebnisse} fällt auf das die Wirksamkeit der Firewall insbesondere im Fall von SQL-Injections (SQLi) gegeben ist. Nach Einsatz der Firewall wurden alle vorher erkannten Schwachstellen geschlossen bzw. eine Ausnutzung der Schwachstelle verhindert. Beide Anwendungen beinhalten eine oder mehrere Schwachpunkte im Bereich Cross-Site-Request-Forgery (CSRF). Obwohl die WAF-Software einen derartigen Schutzmechanismus beinhaltet scheint dieser nur im \emph{Primefaces Showcase} zum Teil zu wirken. Für die \emph{WebGoat}-Anwendung scheint dieser jedoch wirkungslos zu sein.


% Diskussion und Bewertung (3-4 Seiten)
\section{Bewertung}

% Wurden Sie überrascht?
% Stimmten die Hypothesen?
% Sind sie besser, anders als das andere System?
% wichtigster Erkenntnisgewinn?
% Anwendbarkeit Szenario?


% Zusammenfassung: ca. 0,5 Seiten


\section{Zusammenfassung}

% Was haben wir in diesem Kapitel gelernt?
% Wie passt das zur Zielstellung der Arbeit?
% Wie passt das zum nächsten Kapitel?

%Zusammenfassung und Ausblick (5 Seiten)
%  Zusammenfassung
%  Ausblick
\chapter{Zusammenfassung und Ausblick}

\anno{5 Seiten} %Ref. 3

% TODO Zusammenfassung
% TODO Lessons learned

Das der Einsatz einer Web Application Firewall ein Sicherheitsgewinn für jegliche Anwendungen darstellt sollte nicht überraschen. Erfreulich ist jedoch dass eine Software wie WebCastellum auch Jahre nach der eigentlichen Einstellung ihrer Entwicklung und mit einer hohen Last an technischen Schulden auf einen aktuelleren Stand gebracht werden konnte. Im Rahmen dieser Arbeit 

\section{Lessons Learned}
Eine wesentliche Schwierigkeit bei der praktischen Umsetzung eines Projektes auf Basis der Programmiersprache Java mit Techniken aus dem Bereich des \emph{Maschinellen Lernens} ist, dass dieser Bereich sehr \glqq\emph{Python}\grqq-lastig daher kommt. Die meisten Beispiele, Bibliotheken und Tutorials nutzen Python als \emph{Sprache der Wahl}. Der häufig vertretene Standpunkt - man sollte die Sprache wählen, die sich für das jeweilige Projekt am besten eignet - steht für ein \emph{Migrationsprojekt} wie WebCastellum leider nicht zur Verfügung. Zwar ähneln sich beide Sprachen in ihren Paradigmen, vielen Konzepten und in ihrer Anwendung, aber allein die Wahl einer geeigneten Bibliothek mit Unterstützung des maschinellen Lernens ist in diesem Fall doch etwas schwieriger. Umso erfreulicher ist die Tatsache dass mit dem Artikel \glqq\emph{Java \& KI - Das Beste zweier Welten}\grqq{} in der aktuellen Ausgabe des \emph{Javamagazin}s (10/23) die neue Serie \glqq\emph{Künstliche Intelligenz in der Praxis}\grqq{} startet. Offen bleibt dabei ob der Inhalt dieser Arbeit sich eventuell etwas \emph{unterschieden} hätte, wäre diese Serie etwas früher erschienen.\\

Die zweite Schwierigkeit bei der Arbeit war der enorme \emph{Leistungshunger} der Anwendung bei der Erstellung der Entscheidungsbäume. Dass der Rechen- und Speicherbedarf bei Anwendungen dieser Art etwas höher ausfällt war zu erwarten, dass der mir zu Verfügung stehende Rechner bei einem Entscheidungsbaum mit gerade mal zwei bis drei Attributen (bezogen auf den CSIC2010-Datensatz) bereits an seine Grenzen geriet, eher überraschend. Für einen praktischen Einsatz der Software sollten in diesem Bereich dringend weitere Verbesserungen erfolgen.\\

% https://www.dev-insider.de/wie-angreifbar-ist-webgoat-wirklich-a-7a385bc1e48bb04b7fdc9034029555e1/



% Ausblick
\section{Ausblick}

Am Ende dieser Arbeit soll es noch zu einem kleinen Ausblick auf aktuelle Entwicklungen für den Bereich der Web Application Firewalls und auf das Projekt \glqq\emph{WebCastellum}\grqq speziell geben.

Der größte Teil der verwendeten Quellcodes befindet sich derzeit in einem persönlichen aber öffentlich verfügbaren GitHub-Repository unter \url{https://github.com/devtty/webcastellum}. Einzelne Funktionalitäten und Arbeitsschritte werden dort in sogenannten \emph{Feature-Branches} untergebracht, die ggf. in den Hauptzweig des Projektes mit eingebracht werden. Bei dem Repository handelt es sich um lebendes Repository, d.h. Änderungen sind hier jederzeit (auch nach Erscheinen der Arbeit) möglich. Eine zukünftige Weiterentwicklung und Verbesserung des Projektes und insbesondere der vorgestellten Konzepte ist durchaus wünschenswert und hängt neben der persönlichen Motivation auch von der Mitarbeit interessierter Personen ab und ob sich noch weitere \emph{Interessierte} finden. Unabhängig von der Software sollte das Interesse zur Absicherung von Software jeglicher Art gegen Angriffe von Aussen nicht nur proprietären Anbietern überlassen werden. Die größte Priorität aus praktischer Sicht für diesen Zweig von WebCastellum wird weiterhin die Verbesserung der Stabilität und Performance haben. Die neue, nicht-regelbasierte Funktionalität gehört aber in jedem Fall mit dazu.


\begin{neu}
  Nacharbeit Überführung der Evaluation als arquillian test?!
  Bereitstellung als (neuere) WebCastellum Version und Bereitstellung des zentralen Sammelpunktes.
\end{neu}



\begin{appendix}
  %%
%% Beuth Hochschule für Technik --  Abschlussarbeit
%%
%% Anhang
%%
%%%%%%%%%%%%%%%%%%%%%%%%%%%%%%%%%%%%%%%%%%%%%%%%%%%%%%%%%%%%%%%%%%%%%


\chapter{Angehängtes: Die Dateien des Pakets}

\subsection*{Tabellen}

\begin{table}[h]
  \centering
  \resizebox{\textwidth}{!}{
    \begin{tabular}{|c | c | c | c |} 
 \hline
      \textbf{WAF Name} & \textbf{Hersteller} & \textbf{WAF Name} & \textbf{Hersteller} \\
      \hline
      ACE XML Gateway & Cisco & FirePass & F5 Networks\\
      Airlock & Phion/Ergon & FortiWeb & Fortinet\\
      Alert Logic & Alert Logic & GoDaddy Website Protection & GoDaddy\\
      AliYunDun & Alibaba Cloud Computing & Huawei Cloud Firewall & Huawei\\
      AnYu & AnYu Technologies & HyperGuard & Art of Defense\\
      AppWall & Radware & Imunify360 & CloudLinux\\
      Armor Defense & Armor & Incapsula & Imperva Inc.\\
      ArvanCloud & ArvanCloud & ISA Server & Microsoft\\
      ASP.NET Generic & Microsoft & Janusec Application Gateway & Janusec\\
      ASPA Firewall & ASPA Engineering Co. & Kona SiteDefender & Akamai\\
      Astra & Czar Securities & KS-WAF & KnownSec\\
      AWS Elastic Load Balancer & Amazon & LimeLight CDN & LimeLight\\
      AzionCDN & AzionCDN & Oracle Cloud & Oracle\\
      Azure Front Door & Microsoft & ModSecurity & SpiderLabs\\
      Barracuda & Barracuda Networks & NetContinuum & Barracuda Networks\\
      Bekchy & Faydata Technologies Inc. & NetScaler AppFirewall & Citrix Systems\\
      Beluga CDN & Beluga & NullDDoS Protection & NullDDoS\\
      BinarySec & BinarySec & PentaWAF  & Global Network Services\\
      BitNinja & BitNinja &   PT Application Firewall & Positive Technologies\\
      BlockDoS & BlockDoS &   Profense  & ArmorLogic\\
      Bluedon & Bluedon IST &   Qcloud   & Tencent Cloud\\
      CacheWall & Varnish &   RequestValidationMode   & Microsoft\\
      CacheFly CDN & CacheFly &   Sabre Firewall     & Sabre\\
      Comodo cWatch & Comodo CyberSecurity &   eEye SecureIIS & BeyondTrust\\
      CdnNS Application Gateway & CdnNs/WdidcNet &   SecuPress WP Security  & SecuPress\\
      ChinaCache Load Balancer & ChinaCache &   SecureSphere  & Imperva Inc.\\
      Chuang Yu Shield & Yunaq &   SEnginx       & Neusoft\\
      Cloudbric & Penta Security &   SiteGuard  & Sakura Inc.\\
      Cloudflare & Cloudflare Inc. &   SonicWall   & Dell\\
      Cloudfloor & Cloudfloor DNS &   UTM Web Protection              & Sophos\\
      Cloudfront & Amazon &   SquidProxy IDS   & SquidProxy\\
      CrawlProtect & Jean-Denis Brun &   Tencent Cloud Firewall   & Tencent Technologies\\
      DataPower & IBM &   Teros   & Citrix Systems\\
      DenyALL & Rohde+Schwarz CyberSecurity &  Trafficshield  & F5 Networks\\
      Distil & Distil Networks &   Varnish        & OWASP\\
      DotDefender & Applicure Technologies &   WebSEAL  & IBM\\
      DynamicWeb Injection Check & DynamicWeb &   XLabs Security WAF   & XLabs\\
      BIG-IP AppSec Manager & F5 Networks &   ZScaler & Accenture\\
      BIG-IP AP Manager & F5 Networks & & \\
      Fastly & Fastly & & \\
      \hline
      
      
\end{tabular}}
\caption{Auszug WAFw00f Plugins(aus https://github.com/EnableSecurity/wafw00f)}
\medskip
\small
WAFw00f ist ein Werkzeug zur Identifikation von WAFs und die Liste der Plugins gleichzeitig eine Liste der WAF-Produkte die WAFw00f erkennt.
\label{tab:my_wafwoof}
\end{table}

\begin{table}[h]
  \centering
%  \resizebox{\textwidth}{!}{
  \begin{tabular}{|l | p{7cm} |}
    \hline
    \textbf{Name} & \textbf{Beschreibung} \\
    \hline
    gotestwaf & tool for API and OWASP attack simulation \\
    BurpSuite & vulnerability scanning, penetration testing, and web app security platform\\
    identYwaf & identification tool that can recognize web protection \\
    lightbulb-framework & framework for auditing web application firewalls \\
    OWASP Zed Attack Proxy & the world's most widely used web app scanner \\
    sqlmap & testing tool that automates SQL injection\\
    w3af & Web Application Attack and Audit Framework\\
    weka & Waikato Environment for Knowledge Analysis - collection of ML algorithms for data mining tasks\\
    waf-brain & WAF - the Machine-Learning-Deep-Learning-Way\\
    waf-bypass & open source tool to analyze the security of an WAF\\
    WAFNinja &  WAFNinja is a tool which contains two functions to attack Web Application Firewalls. \\
    wafw00f & allows one to identify and fingerprint Web Application Firewall products\\
    \hline
    % \end{tabular}}    
  \end{tabular}
  
  \caption{Übersicht Werkzeuge}
  \medskip
  \small
  (Die Beschreibung der Werkzeuge wurde den jeweiligen Internetpräsenzen entnommen)
  \label{tab:my_tools}
\end{table}

\begin{table}[h]
  \centering
%  \resizebox{\textwidth}{!}{ 
  \begin{tabular}{|l | l | l |}
    \hline
    \textbf{Name} & \textbf{Wert} & \textbf{Anmerkungen} \\
    \hline
    HTTP Methode & \verb=GET= oder \verb=POST= im Anomalieteil auch \verb=PUT= & \\
    URL & & \\
    Protocol & \verb=HTTP/1.1= & HTTP/2 \\
    User-Agent & \verb=Mozilla/5.0 (compatible;Konqueror/3.5...= & \\
    Pragma & \verb=no-cache= & \\
    Cache-control & \verb=no-cache= & \\
    Accept & \verb=text/xml,application/xml,...= &\\
    Accept-Encoding & \verb=x-gzip,x-deflate,gzip,deflate= & \\
    Accept-Charset & \verb!utf-8,utf-8;q=0.5,*;q=0.5! & \\
    Accept-Language & \verb=en= & \\
    Host & \verb=localhost:8080= & \\
    Cookie & \verb=JSESSIONID= verschiedene Werte & \\
    Content-Type & \verb=null= oder \verb=application/x-www-from-urlencoded= & \\
    Connection & \verb=close= & \\
    Content-Length & & \\
    Payload & & \\
    \hline
      % \end{tabular}}
    \end{tabular}
  \caption{Felder des CSIC2010 Datensatzes}
  \label{tab:csicfields}
\end{table}

\begin{table}[h]
  \centering
  \begin{tabular}{|l|}
    \hline
    JavaVulnerableLab \\
    Marathon\\
    VulnarableSpring \\
    OWASP Mutillidae II \\
    OWASP WebGoat \\
    \hline
  \end{tabular}
  \caption{Übersicht Absichtlich angreifbare Anwendungen}
  \label{tab:vulnapp}
\end{table}
\end{appendix}
%%%%%%%%%%%%%%%%%%%%%%%%%%%%%%%%%%%%%%%%%%%%%%%%%%%%%%%%%%%%%%%
%% Literaturverzeichnis

\clearpage\newpage
\addcontentsline{toc}{chapter}{Literatur- und Quellenverzeichnis}
\bibliographystyle{myapalike}
\bibliography{bhtThesis}

\end{document}
