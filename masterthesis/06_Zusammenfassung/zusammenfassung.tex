\chapter{Zusammenfassung und Ausblick}

\anno{5 Seiten (2)} %Ref. 3

% TODO Zusammenfassung
% TODO Lessons learned

Das der Einsatz einer Web Application Firewall ein Sicherheitsgewinn für jegliche Anwendungen darstellt sollte nicht überraschen. Erfreulich ist jedoch dass eine Software wie WebCastellum auch Jahre nach der eigentlichen Einstellung ihrer Entwicklung und mit einer hohen Last an technischen Schulden auf einen aktuelleren Stand gebracht werden konnte und die erweiterten Funktionen, die im Rahmen dieser Arbeit vorgeschlagen wurden auch mit eingebracht werden konnten.

\section{Lessons Learned - Schwierigkeiten am Rande}
Eine wesentliche Schwierigkeit bei der praktischen Umsetzung eines Projektes auf Basis der Programmiersprache Java mit Techniken aus dem Bereich des \emph{Maschinellen Lernens} ist, dass dieser Bereich sehr \glqq\emph{Python}\grqq-lastig daher kommt. Die meisten Beispiele, Bibliotheken und Tutorials nutzen Python als \emph{Sprache der Wahl}. Der häufig vertretene Standpunkt - man sollte die Sprache wählen, die sich für das jeweilige Projekt am besten eignet - steht für ein \emph{Migrationsprojekt} wie WebCastellum leider nicht zur Verfügung. Zwar ähneln sich beide Sprachen in ihren Paradigmen, vielen Konzepten und in ihrer Anwendung, aber allein die Wahl einer geeigneten Bibliothek mit Unterstützung des maschinellen Lernens ist in diesem Fall doch etwas schwieriger. Umso erfreulicher ist die Tatsache dass mit dem Artikel \glqq\emph{Java \& KI - Das Beste zweier Welten}\grqq{} in der aktuellen Ausgabe des \emph{Javamagazin}s (10/23) die neue Serie \glqq\emph{Künstliche Intelligenz in der Praxis}\grqq{} startet. Offen bleibt dabei ob der Inhalt dieser Arbeit sich eventuell etwas \emph{unterschieden} hätte, wäre diese Serie etwas früher erschienen.\\

Die zweite Schwierigkeit bei der Arbeit war der enorme \emph{Leistungshunger} der Anwendung bei der Erstellung der Entscheidungsbäume. Dass der Rechen- und Speicherbedarf bei Anwendungen dieser Art etwas höher ausfällt war zu erwarten, dass die zur Verfügung stehenden Rechner\footnote{Apple MacBook M2 8GB RAM und Lenovo ThinkPad X230 i5 8GB} bei einem Entscheidungsbaum mit gerade mal zwei bis drei Attributen (bezogen auf den CSIC2010-Datensatz) bereits an ihre Grenzen geraten, eher überraschend. Für einen praktischen Einsatz der Software sollten in diesem Bereich dringend weitere Verbesserungen erfolgen.\\



% Ausblick
\section{Was bringt die Zukunft}

Am Ende dieser Arbeit soll es noch zu einem kleinen Ausblick auf aktuelle Entwicklungen und Randthemen für den Bereich der Web Application Firewalls und auf das Projekt \glqq\emph{WebCastellum}\grqq geben.\\

% wekalight

Ein weiteres Feld für zukünftige Entwicklungen könnte auch von der aus anderen Bereichen bekannten \emph{Miniaturisierung} abgeleitet werden. Wie auch technische Geräte unter Beibehaltung ihrer Funktion\footnote{s. \url{https://de.wikipedia.org/wiki/Miniaturisierung} abgerufen am 24.08.2023} immer kleiner werden ist ein ähnlicher Trend auch bei Softwaresystemen oder deren Plattformen zu beobachten. Monolithisch gebaute Anwendungen werden in Microservices aufgeteilt und auch die Hostsysteme werden immer kleiner. Teilweise werden Anwendungen auf Mobiltelefon betrieben, für die vor einigen Jahren noch Großrechner nicht leistungsfähig genug waren. Web Application Firewalls sollen vielleicht nicht mehr ganze Anwendungen absichern sondern nur noch einzelne verteilte Dienste und die WAF ist dabei selbst nur ein Dienst der sich selbst verteilen muss (s. \url{https://github.com/projectcalico/calico}). Selbst die in dieser Arbeit verwendete WEKA-Bibliothek hat mit \emph{Tiny WEKA} (Paket im Maven Central Repository abrufbar) einen kleineren Abkömmling mit etwas weniger Funktionalität im Gepäck der auch auf mobilen Endgeräten lauffähig sein sollte.\\

% chatgpt
Aus persönlichem Interesse wäre, neben der weiteren Entwicklung der Firewall, die Erweiterung der Features für die ML-Funktion ein Punkt für weitere Arbeiten. Außer den Daten aus den HTTP-Anfragen und Antworten existieren in vielen Anwendungen noch weitere Variablen die für die \emph{Überwachung} einer Anwendung in sicherheitstechnischen Aspekten von Bedeutung wären. Beispielsweise speichern JSF-Anwendungen den Status der Oberflächenkomponenten in einer Session (\verb=ViewState=) ab. Der direkte Zugriff auf die GUI der Webanwendung bzw. Teile der Oberfläche könnten ebenfalls gute Unterscheidungsmerkmale zur Klassifizierung abgeben.\\

% https://www.dev-insider.de/wie-angreifbar-ist-webgoat-wirklich-a-7a385bc1e48bb04b7fdc9034029555e1/
Ein weiterer zu betrachtender Aspekt ist eher nicht-technischer Natur. Nicht jeder theoretisch mögliche Angriff auf Web-Anwendungen ist auch praktisch durchführbar, da in vielen Fällen die für einen Angriff erforderlichen Vorraussetzungen nicht erfüllt werden. Für die aus Abschnitt~\ref{sec:absangan} bekannte Anwendung WebGoat wurde beispielsweise festgestellt, dass \glqq\emph{78 Prozent der gemeldeten CVEs tatsächlich nicht ausnutzbar waren}\grqq{} (s.~\cite{lueck23}). Hier gilt es sich nicht nur auf technische Werkzeuge und deren Ergebnisse zu verlassen.\\

Der größte Teil der verwendeten Quellcodes befindet sich derzeit in einem persönlichen aber öffentlich verfügbaren GitHub-Repository unter \url{https://github.com/devtty/webcastellum}. Einzelne Funktionalitäten und Arbeitsschritte werden dort in sogenannten \emph{Feature-Branches} untergebracht, die ggf. in den Hauptzweig des Projektes mit eingebracht werden. Bei dem Repository handelt es sich um lebendes Repository, d.h. Änderungen sind hier jederzeit (auch nach Erscheinen der Arbeit) möglich. Eine zukünftige Weiterentwicklung und Verbesserung des Projektes und insbesondere der vorgestellten Konzepte ist durchaus wünschenswert und hängt neben der persönlichen Motivation auch von der Mitarbeit interessierter Personen ab und ob sich noch weitere \emph{Interessierte} finden. Unabhängig von der Software sollte das Interesse zur Absicherung von Software jeglicher Art gegen Angriffe von Aussen nicht nur proprietären Anbietern überlassen werden. Die größte Priorität aus praktischer Sicht für diesen Zweig von WebCastellum wird weiterhin die Verbesserung der Stabilität und Performance haben. Die neue, nicht-regelbasierte Funktionalität gehört aber in jedem Fall mit dazu.
