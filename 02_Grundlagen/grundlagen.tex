\chapter{Grundlagen}

\anno{bitte nicht mehr als 8 Seiten}

% Grundbegriffe 2 Seiten
\section{Grundbegriffe}
\subsection{WAF allgemein}
\subsubsection{Anwendungsfälle}
%% Anwendungsfaelle WAF (gut beschrieben bei WAFEC2)
%% irgendwie Uebergang zu ML und WAF mit ML schaffen
%% Sammlung; payload,fuzzer,fingerprinting, bypassing

\subsection{Arten}

\subsection{Grundbegriffe allgemein}
\textbf{Bypassing:}

\textbf{Fingerprinting:} Ähnlich der Abnahme und Identifizierung von Personen mit Hilfe eines individuellen Fingerabdrucks können auch Produkte wie Software anhand spezifischer Merkmale identifiziert werden. Beim \emph{Fingerprinting}

\textbf{Fuzzer:}

\textbf{Payload:}

\section{Related Work} %umbenennen ca. 6

% Thema 1
\subsection{Evolution der Firewalls}

\subsubsection{Strikt nach Regeln}

\subsubsection{Hybride Ansätze}

% Krueger Manaseer etc.

\subsubsection{Fortschritte in Richtung Intelligenz}

% gimenez appelt kozik testen mit ML Ansätzen

% Thema 2
\subsection{Thema 2 - ML}

%ansaetze und kombination?

% Thema 3 optional die andere seite
\subsection{hacking Wafs}


% Zusammenfassung (ca. 0,5 Seiten)
\section{Zusammenfassung}

%ggf. ditaa tabelle ueber den Zeitverlauf der verschiedenen Arbeiten nach Attack-Defend-Muster