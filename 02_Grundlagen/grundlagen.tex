\chapter{Grundlagen}

\anno{bitte nicht mehr als 8 Seiten}

% Grundbegriffe 2 Seiten
\section{Grundbegriffe}
\subsection{WAF allgemein}
m die Sicherheit einer Anwendung zu gew ̈ahrleisten sind viele verschiedene Schrit-te  notwendig,  so  sollten  nat ̈urlich  bereits  in  der  Anwendung  selbst  Ein-  und  Ausga-bem ̈oglichkeiten, z.B. durch Validierung, Encoding, usw. ̈uberpr ̈uft und gegebenenfallseingeschr ̈ankt werden. Dabei l ̈asst sich nicht jeder m ̈ogliche Angriffsfall vorhersehen odereine eigene Implementierung ist zu aufwendig und teuer, weil h ̈aufig nicht ausreichendpersonelle Ressourcen  oder  Bugdet vorhanden  sind um eine  Web  Applikation auf  allem ̈oglichen  Sicherheitsl ̈ucken  zu  pr ̈ufen.  Im  Laufe  des  Betriebes  einer  Web  Anwendun-gen k ̈onnen zudem neue Angriffsvarianten entstehen und zus ̈atzlich ist gerade bei WebApplikationen der Zeitdruck zur Ver ̈offentlichung einer solchen h ̈aufig sehr hoch.An diesem Punkt kommen sogenannte Web Application Firewalls ins Spiel. Im Ge-gensatz  zu  regul ̈aren  Firewalls  haben  Web  Application  Firewalls  direkten  Zugriff  aufdie  HTTP-Anfragen  (requests)  und  Antworten  (responses)  und  k ̈onnen  diese  entspre-chend  bewerten  und  gegenbenenfalls  blockieren  oder  gef ̈ahrdende  Inhalte  filtern  oderumschreiben.Grunds ̈atzlich lassen sich solche Systeme nach ihrer Position in der Netzwerk- und Ser-vertopologie  unterscheiden.  Es  existieren  einerseits  Systeme  die  vor  eine  Anwendunggeschaltet werden und Systeme die direkt in die Anwendung integriert werden. Die ersteGruppe l ̈asst noch eine Verzweigung in weitere Unterarten, wieReverse Proxy,Applian-ce,Pluginsf ̈ur WebServer oderPassive Devices(IDS), zu.

\subsubsection{Anwendungsfälle}
%% Anwendungsfaelle WAF (gut beschrieben bei WAFEC2)
%% irgendwie Uebergang zu ML und WAF mit ML schaffen
%% Sammlung; payload,fuzzer,fingerprinting, bypassing

\subsection{Arten}

\subsubsection{Unterscheidung nach Position}

\subsubsection{Unterscheidung nach Abwehrmaßnahmen}

\subsection{Grundbegriffe allgemein}
\textbf{Bypassing:}

\textbf{Fingerprinting:} Ähnlich der Abnahme und Identifizierung von Personen mit Hilfe eines individuellen Fingerabdrucks können auch Produkte wie Software anhand spezifischer Merkmale identifiziert werden. Beim \emph{Fingerprinting}

\textbf{Fuzzer:}

\textbf{Payload:}

\section{Related Work} %umbenennen ca. 6

% Thema 1
\subsection{Evolution der Firewalls}


\subsubsection{Strikt nach Regeln}

\subsubsection{Hybride Ansätze}

% Krueger Manaseer etc.

\subsubsection{Fortschritte in Richtung Intelligenz}

% gimenez appelt kozik testen mit ML Ansätzen

% Thema 2
\subsection{Thema 2 - ML}

%ansaetze und kombination?

% Thema 3 optional die andere seite
\subsection{hacking Wafs}


% Zusammenfassung (ca. 0,5 Seiten)
\section{Zusammenfassung}

%ggf. ditaa tabelle ueber den Zeitverlauf der verschiedenen Arbeiten nach Attack-Defend-Muster