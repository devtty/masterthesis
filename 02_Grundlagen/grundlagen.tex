\chapter{Grundlagen}

\anno{bitte nicht mehr als 8 Seiten (8)}
%evt. kann hier aus den Technischen Grundlagen (Overleaf WP_Thesis) noch eingefuegt werden

% Grundbegriffe 2 Seiten

Die erste Hälfte des Kapitels beinhaltet einige Grundlagen und Erklärungen um genauer zu verstehen wie Angriffe auf eine Web Anwendung ablaufen und wie damit verbundene Sicherheitsrisiken präventiv entdeckt werden können. Die zweite Hälfte gibt einen Einblick zum Thema Softwaretests mit Bezug auf die für diese Arbeit notwendigen Abläufe und Werkzeuge. 

\section{Grundbegriffe}

\subsection{Allgemein}

\textbf{API:} \emph{Application Programming Interface} oder auch Anwendungsprogrammierschnittstelle, Komponente einer Software die die Anbindung weiterer externer Anwendungen ermöglicht

\textbf{Bypassing:}

\textbf{Dictionary:} ein Wörterbuch, in der IT-Sicherheit eine Sammlung von Zeichenketten, z.B. eine Liste der meist verwendeten Passworte

\textbf{Entscheidungsbaum:} überwachter Lernalgorithmus für Klassifizierungsaufgaben; anhand eines Entscheidungsbaumes könnte Datenverkehr beispielsweise in erwünscht oder nicht erwünscht eingeteilt werden

\textbf{Feature:} Im Bereich \emph{Machine Learning} repräsentiert ein Feature eine Eigenschaft anhand derer einzelne Dateneinträge klassifiziert werden können. Übertragen auf den Menschen - zum Beispiel - die Augenfarbe. Diese Merkmale sollten individuell messbar sein und können auch kombiniert werden, z.B. alle Menschen, über 170 Zentimeter Körpergröße und mit blauen Augen. 

\textbf{Filter:}

\textbf{Fingerprinting:} Ähnlich der Abnahme und Identifizierung von Personen mit Hilfe eines individuellen Fingerabdrucks können auch Produkte wie Software anhand spezifischer Merkmale identifiziert werden. Beim \emph{Fingerprinting}

\textbf{Fuzzer:} Generator für Zufallsdaten

\textbf{Klassifizierung:} Einteilung in Gruppen durch Abgrenzung anhand von Kriterien

\textbf{Maschinelles Lernen:} Unterschieden zwischen 

\textbf{OpenSource:}

\textbf{Payload:} Allgemein steht der Begriff für Nutzdaten im Datenverkehr; IT-sicherheitstechnisch bedeutsam da Sicherheitslücken häufig Fehler in der Überprüfung der Nutzdaten ausnutzen, daher zum Teil auch Synonym für Angriffsursache

\textbf{Reverse Proxy:} Server zwischen Client und dem eigentlichen Server/Servern 

\textbf{Regulärer Ausdruck:} Eine Zeichenkette zur Beschreibung einer Menge von Zeichenketten mit Hilfe einer vordefinierten Syntax. Ein regulärer Ausdruck kann beispielsweise genutzt werden um bestimmte Muster in einem Text zu erkennen oder einen Text mit einem Muster zu vergleichen.

\textbf{Request:} 

\textbf{Response:}

\textbf{Überwachtes Lernen:} 

\subsection{Arten von Angriffen}

Die bekannteste Referenz für das Thema \glqq Web Anwendungen und IT-Sicherheit\grqq  ist sicherlich das \emph{Open Web Application Security Project\textsuperscript{\textregistered}}, eine Non-Profit-Organisation mit dem Ziel, die Sicherheit von Anwendungen und Diensten im World Wide Web zu verbessern\cite{wpowasp}. Die OWASP Foundation lädt regelmäßig zu Konferenzen ein und erstellt Leitfäden, Sicherheitsempfehlungen und Werkzeuge zum Thema Sicherheit im Web.\\
Bekanntestes Produkt ist eine jährlich herausgegebene, kuratierte Liste der am häufigsten vorkommenden Arten von Angriffen auf Anwendungen im Netz, die so genannten \emph{OWASP Top Ten}\cite{owasp10}

\subsubsection{Kategorien von Angriffszielen}

Die OWASP Top 10 sind im Umkreis von IT-Sicherheits-Experten, Softwarearchitekturen und Entwicklern bekannt und geben einen guten Überblick über die häufigsten Angriffsmuster auf Webanwendungen. Web Application Firewalls können insbesondere bei der Vermeidung von Angriffen in den Kategorien \emph{Injection} und \emph{Server-Side Request Forgery} nützlich sein - je nach Konfiguration aber auch in den (meisten) anderen Kategorien.  

Mit der vermehrten Nutzung webbasierter Schnittstellen, z.B. durch Microservice-Architekturen, Single Page Applications oder dem Internet-of-Things, sah sich die OWASP Foundation gezwungen für den Bereich API-Security eine eigene Top 10 Liste zu erstellen, die OWASP API Security Top 10\cite{owaspapi10}. Da \emph{Application Programming Interfaces} (API) in vielen Fällen auch für die allgemeine Nutzung im Internet freigegeben sind, sollte der Einsatz von WAFs zum Schutz dieser Fälle daher mit betrachtet werden. 

\begin{table}[ht]
    \centering
    \begin{tabular}{|c|}
    \hline
         Broken Object Level Authorization   \\
         Broken User Authentication  \\
         Excessive Data Exposure  \\
         Lack of Resources and Rate Limiting  \\
         Broken Function Level Authorization  \\
         Mass Assignment  \\
         Security Misconfiguration  \\
         Injection  \\
         Improper Assets Management  \\
         Insufficient Logging and Monitoring  \\
         \hline
    \end{tabular}
    \caption{OWASP API Top 10}
    \label{tab:owaspapitop10}
\end{table}

\textcolor{bhtGray}{\ding{110} Beispiel MyHundai Hack\footnote{https://threadreaderapp.com/thread/1597695281881296897.html}} Viele Fahrzeughersteller bieten mittlerweile Handy-Apps zum Abrufen von Informationen über Fahrzeuge und (teilweise) deren Steuerung an. Der Hersteller Hyundai lässt seine App dazu über den Endpunkt \texttt{api.telematics.hyundaiusa.com} kommunizieren. Eine Beschreibung der API findet sich unter \emph{https://hacksore.github.io/bluelinky-docs/docs/api-reference}. Im beschriebenen Beispiel nutzten die Angreifer eine manipulierte Email-Adresse und verschafften sich Zugang zum Fahrzeug des Opfers (einer der Angreifer), inklusive der Möglichkeit das Fahrzeug aus der Ferne zu starten. \\\\

Die MITRE Corporation sammelt in einer Datenbank\footnote{https://cve.mitre.org} öffentlich bekannt gewordene Schwachstellen und Risiken. Wenn eine Gefahr identifiziert wurde wird dieser Gefahr von einer sogenannten \emph{CVE Numbering Authority} eine eindeutige Referenznummer zugewiesen. Eine CVD-Referenznummer beginnt dabei immer mit dem Präfix \glqq\texttt{CVD-}\grqq gefolgt von der Jahreszahl der Veröffentlichung und einer fortlaufenden Nummer. Sind dann nähere Informationen oder Lösungen bekannt werden diese in weiteren Datenbanken veröffentlicht. Ein Beispiel dafür wäre die U.S National Vulnerability Database (NVD)\footnote{https://nvd.nist.gov}.\\\\

\textcolor{bhtGray}{\ding{110} Beispiel log4j2\footnote{https://nvd.nist.gov/vuln/detail/CVE-2021-44228}} Eine der größeren Sicherheitslücken der letzten Zeit betraf das Logging-Framework \emph{Apache Log4j2}. Durch Manipulation der Lognachrichten konnten hier ganze Rechner gekapert werden. Eine Möglichkeit beschrieb Stefano Chierici in seinem Blog\cite{chierici2021}. Er nutzte einen HTTP-Request zum Setzen eines Cookies zur Übernahme eines Rechners.\\\\

\anno{beide Schwachstellen später nochmal referenzieren mit Hinweis das beide durch vorschalten einer WAF u.U. schneller eingedämmt wären -> 0day}

\subsection{Der Tag Null}

Wird eine Schwachstelle aktiv ausgenutzt ohne das der Angriff entdeckt wird, spricht man von einem \glqq\emph{Zero-Day-Exploit}\grqq. Anwendungen sind in diesem Fall besonders gefährdet da der Hersteller  noch keine Zeit hatte, den Nutzer zu schützen\cite{bsi0day}. 

\subsection{WAF allgemein}
Um die Sicherheit einer Anwendung zu gewährleisten sind viele verschiedene Schritte notwendig, so sollten natürlich bereits in der Anwendung selbst Ein- und Ausgabemöglichkeiten, z.B. durch Validierung, Encoding, usw. überprüft und gegebenenfalls eingeschränkt werden. Dabei lässt sich nicht jeder mögliche Angriffsfall vorhersehen oder eine eigene Implementierung ist zu aufwendig und teuer, weil häufig nicht ausreichend personelle Ressourcen oder Budget vorhanden sind um eine Web Applikation auf alle möglichen Sicherheitslücken zu prüfen. Im Laufe des Betriebes einer Web Anwendungen können zudem neue Angriffsvarianten entstehen und zusätzlich ist gerade bei Web Applikationen der Zeitdruck zur Veröffentlichung einer solchen häufig sehr hoch.\\

An diesem Punkt kommen sogenannte Web Application Firewalls ins Spiel. Im Gegensatz zu regulären Firewalls haben Web Application Firewalls direkten Zugriff auf die HTTP-Anfragen (requests) und Antworten (responses) und können diese entsprechend bewerten und gegebenenfalls blockieren oder gefährdende Inhalte filtern oder umschreiben.\\

%% 
\textcolor{bhtGray}{\ding{110} Definition\footnote{\url{https://owasp.org/www-community/Web_Application_Firewall} abgerufen am 30.05.2023}} A web application firewall is an application firewall for HTTP applications. It applies a set of rules to an HTTP conversation. Generally, these rules cover common attacks such as Cross-site Scripting (XSS) and SQL Injection. While proxies protect generally protect clients, WAFs protect servers. A WAF is deployed to protect a specific web application or set of web applications. A WAF can be considered a reverse proxy. WAFs may come in the form of an appliance, server plugin, or filter, and may be customized to an application to an application. The effort to perorm this customization can be significant and needs to be maintained as the application is modified.

%%

\subsubsection{Anwendungsfälle}
%% Anwendungsfaelle WAF (gut beschrieben bei WAFEC2)
%% irgendwie Uebergang zu ML und WAF mit ML schaffen
%% Sammlung; payload,fuzzer,fingerprinting, bypassing

\subsection{Arten}

\subsubsection{Unterscheidung nach Position}
Grundsätzlich lassen sich solche Systeme nach ihrer Position in der Netzwerk- und Servertopologie unterscheiden. Es existieren einerseits Systeme die vor eine Anwendung geschaltet werden und Systeme die direkt in die Anwendung integriert werden. Die erste Gruppe lässt noch eine Verzweigung in weitere Unterarten, wie \emph{Reverse Proxy}, \emph{Appliance}, \emph{Plugins} für WebServer oder \emph{Passive Devices} (IDS), zu.


\subsubsection{Unterscheidung nach Abwehrmaßnahmen}

\paragraph{Regelbasierte Systeme}
Der Großteil bekannter WebApplicationFirewalls arbeitet jedoch \emph{regelbasiert}. In diesem Fall werden ein- und ausgehende Datenströme (Requests/Responses) unabhängig voneinander (zustandslos) betrachtet und einer Mustererkennung unterworfen. Häufig sind die Regeln anhand sogenannter \emph{Regular Expressions} definiert.

\paragraph{Logische Systeme}
Bei logikbasierten Abwehrmaßnahmen handelt es sich um Abwehrmaßnahmen die aufgrund von bekannten (logischen) Rückschlüssen eingeleitet werden. Einfachstes Beispiel wäre das temporäre Sperren der Loginseite bei dreimalig falschem Login. Bei Nutzung einer WAF die keine logikbasierten Auswertungen ermöglicht, müssten entsprechende Anwendungsfälle in der Anwendung selbst implementiert werden. (Oder im Fall der gerade beschriebenen Authentifizierungproblematik in einen externen Dienst ausgelagert werden.)


\section{Related Work} %umbenennen ca. 6
\label{sec:relatedwork}

% Thema 1
\subsection{Evolution der Firewalls}

\subsubsection{Strikt nach Regeln}

Die praktisch einfachste Web Application Firewall wäre \emph{-beispielsweise-} eine einfache Regel innerhalb der Webserver-Konfiguration, die anhand eines bestimmten Merkmals den Datenverkehr entweder korrekt beantwortet oder \glqq\emph{abwehrt}\grqq. Wenn nicht explizit entnommen findet sich der folgende Eintrag in jeder Konfiguration des \emph{Apache httpd}-Servers und sorgt dafür dass der Zugriff auf jede Datei deren Dateiname mit \texttt{.ht} beginnt unterbunden wird:

\lstset{language=XML,
 	basicstyle=\ttfamily\color{black}\small,
 	keywordstyle=\bfseries\color{bhtBlue},
 	identifierstyle=\color{black}, 
 	commentstyle=\color{gray}\textsl
      }
%      \begin{figure}
%        \caption{Beispiel für einfache Regel}
%        \label{fig:httprule}
\begin{lstlisting}
  <Files ".ht*">
    Require all denied
  </Files>
\end{lstlisting}
%      \end{figure}

Mit steigender Komplexität der zu schützenden Anwendungen steigt auch der Bedarf an Regeln. Zu berücksichtigende Auswahlkriterien beschränken sich dann auch nicht nur auf den \glqq\emph{Dateinamen}\grqq. Attribute wie der Aufrufzeitpunkt, der Inhalt des Aufrufs, die Identität des Aufrufenden und viele andere Kriterien können zu entscheidenden Faktoren werden. Im Sinne der Entkopplung wurde diese Filterlogik häufig in entsprechende Module oder Plugins ausgelagert. Aufgrund des hohen Verbreitungsgrades des \emph{Apache httpd}-Servers wurde dessen \emph{modSecurity}-Modul zu einem der bekanntesten Vertreter einer Web Application Firewall. Dabei handelt es sich um eine frei verfügbare und als \emph{Open Source} entwickelte Softwarelösung, die eine eigene Sprache zum Erstellen der WAF-Regeln mit sich bringt. Der obige Aufruf zum Verhindern des Zugriffs auf Dateien die mit \texttt{.ht} beginnen lautet:

\lstset{language=bash,
 	basicstyle=\ttfamily\color{black}\small,
 	keywordstyle=\bfseries\color{bhtBlue},
 	identifierstyle=\color{black}, 
 	commentstyle=\color{gray}\textsl
      }
\begin{lstlisting}
  #SYNTAX: SecRule VARIABLES OPERATOR [ACTIONS]        
  SecRule REQUEST_URI ".ht*" "deny"
\end{lstlisting}
% end figure

Bei dieser Sicherheitsregel \verb=SecRule= wird die aufgerufene Adresse \verb=VARIABLES= auf Übereinstimmung mit einem regulären Ausdruck \verb=OPERATOR= geprüft und bei Übereinstimmung die Ausführung verweigert \verb=ACTIONS=.

\subsubsection{Allgemeine Regeln}
Mit der Zeit entwickelten sich immer mehr und immer komplexere Regeln. Regeln wurden auf spezielle Anwendungen, Anwendungsfälle oder Angriffsszenarien zugeschnitten und gesammelt. Es entstanden Sammlungen die unter Administratoren, Entwicklern und anderen IT-Spezialisten ausgetauscht wurden.
Im Oktober 2006 veröffentlichte das \emph{Open Web Application Security Project} erstmals eine allgemein verfügbare Sammlung von generischen Regeln für das modSecurity-Modul mit dem Ziel ein Art Grundschutz zur Absicherung von Webanwendungen anzubieten. Die als \emph{OWASP Core Rule Set} bekannte Sammlung wird seitdem stetig erweitert und weiter entwickelt. Wirft man heute einen Blick in die verfügbaren Regeldateien des Projekts \footnote{\url{https://github.com/coreruleset/coreruleset/tree/v4.0/dev/rules} abgerufen am 01.06.2023} findet man sowohl allgemeine Regeln mit Bezug auf die Angriffsmuster aus den OWASP Top10~\cite{owasp10}, als auch Regeln die sehr speziell auf bestimmte Anwendungen oder Sicherheitsvorfälle ausgelegt sind.

Obwohl die bereitgestellten Regeln des Core Rule Sets eine gewisse Basissicherheit bieten können, sind diese nur als Ausgangsbasis zur (produktiven) Absicherung von Anwendungen zu sehen. Erstens wird nicht jeder Fall berücksichtigt und zweitens erzeugen die Regeln unter Umständen auch Fehlmeldungen (Falsch-Positive).

\subsubsection{Von Regeln zur Logik}
Die Einträge in der Serverkonfiguration und die ersten Regeln im modSecurity-Modul erlaubten es die Anfragen an den Server entweder positiv oder negativ zu beantworten. Im Grunde könnte man die Funktionsweise zum damaligen Zeitpunkt mit einer \emph{Blacklist} bzw. \emph{Whitelist} für HTTP-Anfragen vergleichen. Einzelne Regeln entschieden über die Beantwortung einer Anfrage im positiven oder negativem Sinn. Mit der Weiterentwicklung des Moduls und der zweiten Version des Core Rule Sets wurden mit dem \emph{Scoring} und dem \emph{Paranoia-Level} zwei neue Funktionalitäten umgesetzt, die die Konfiguration für den Endanwender wesentlich vereinfachen sollten und auch einfache logische Entscheidungen ermöglichen sollten.\\

\textcolor{bhtGray}{\ding{110} Anmerkung} Natürlich erlauben \emph{Reguläre Ausdrücke} ebenfalls eine gewisse Logik und somit auch die darauf basierenden Regeln.\\

Beim \emph{Scoring} erfolgt die Auswertung der HTTP-Anfrage nicht in Form einer direkten Aktion, stattdessen wird als Ergebnis ein numerischer Wert festgelegt. Erst im Nachgang entscheidet die Firewall wie mit der Anfrage umzugehen ist. Das Scoring erlaubt es verschiedene Ergebnisse miteinander zu kombinieren und in Abhängigkeit voneinander auszuwerten. Technisch gesehen ist das HTTP-Protokoll zustandslos und bei einfachen Konfigurationen der WAF  wird jede einzelne Anfrage unabhängig von anderen Anfragen betrachtet. Der Scoring-Mechanismus erlaubt es in gewissem Maße auch zustandsbezogen zu agieren. \anno{Zu undeutlich? Blöd ausgedrückt?}\\

% weiter erklären!
% Idee: Ablaufdiagramme für Version 1(direkte Antwort) und 2 Scoring Mechanismus

\textcolor{bhtGray}{\ding{110} Beispiel} Eine Anfrage auf eine bestimmte Ressource bzw. URI soll abgelehnt werden wenn der Nutzer nicht eingeloggt ist. Dieser Fall wäre könnte mit drei einfachen Regeln abgebildet werden. Es wird für den Fall \emph{nicht eingeloggt} ein Punkt vergeben und für den Fall \emph{Aufruf der Ressource} ein Punkt vergeben. Die dritte Regel besagt dass alle Aufrufe deren Summe mehr als zwei Punkte in Summe überschreiten, abgelehnt werden.\\

\textcolor{bhtGray}{\ding{110} Beispiel} Angenommen ein unbekannter Internet-Nutzer versucht eine Anwendung durch nicht manipulierte Anfragen zu attackieren oder einen Brute-Force-Angriff auf eine sehr einfache Login-Maske und jeder misslungene Versuch wird erkannt bzw. bestätigt. In diesem Fall erzeugt die Anwendung mehrfach HTTP-Status-Codes abseits des üblichen Rückgabe-Wertes \verb=HTTP200-OK=. Mit einem Scoring ist möglich diesen Nutzer bereits an den Versuchen zu hindern.
\begin{align}
  P = \frac{Scoring-Wert}{Anzahl_Anfragen gesamt}
\end{align}
\anno{Formel überprüfen?Weitermachen!}

In der ersten Version des CRS wurden alle der WAF bekannten Regeln auch aktiv umgesetzt und die Konfiguration der Firewall musste in jedem Fall für die jeweilige Anwendung angepasst werden. Nicht benötigte oder verkehrte (falsch-positiv liefernde) Regeln wurden der Konfiguration entnommen oder angepasst. Die erstmalige Inbetriebnahme hatte damit einen erhöhten Arbeitsaufwand zur Folge. Die Einführung der Paranoia-Level erlaubte eine einfache und schnellere Konfiguration. Dazu wurden die Regeln nach Schweregraden kategorisiert und deren Ausführung in Abhängigkeit von vorkonfigurierten Werten gesteuert werden. Der Nutzer bekam die Möglichkeit eine sofort (bzw. mit sehr wenig Aufwand) einsetzbare Lösung zur erhalten, mit der seine Anwendung Schritt für Schritt sicherer werden konnte.

\begin{figure}[ht]
  %% \begin{center}
  \centering
  \includesvg[width=0.5\textwidth]{pl_onion_no_fonts}
  \caption{Übersicht Paranoia Level~\cite{owaspcrs}}
  \label{fig.paranoia}
%%  \end{center}
\end{figure}


% Logik mit Zugriff auf die Anwendung -> WebCastellum?

\begin{neu}
  Kruegel Absatz über Anomalie-Detektoren? Hier einfügen!
\end{neu}

\subsubsection{Hybride Ansätze} %Nicht immer schwarz oder weiss
Ein Ergebnis muss aber nicht immer schwarz oder weiß sein.  Grundsätzlich besteht für Web Application Firewalls die Möglichkeit HTTP-Anfragen zu manipulieren bevor diese die eigentliche Anwendung erreichen. Gleiches gilt für die Antworten der Anwendung bevor diese an den Client ausgeliefert werden. Für den Apache httpd-Server existiert mit dem \emph{modRewrite}-Modul sogar eine Komponente einzig und allein für diesen Zweck. Ende 2010 erschien mit \glqq\emph{TokDoc: A Self-Healing Web Application Firewall}\grqq~\cite{Krueger2010} ein Konferenzpaper mit dem Vorschlag, eingehende Anfragen in einzelne Bestandteile zu zerlegen, die einzelnen Bestandteile zu untersuchen und ggf. durch \emph{korrigierte} Äquivalente auszutauschen bevor diese an die zu schützende Anwendung weitergeleitet werden. Zusätzlich haben die Autoren des Papers die Bestandteile nach ihrem Ursprung in vier Kategorien eingeteilt:\\

\begin{table}[h]
  \centering
  \begin{tabular}{|l|p{8cm}|}
    \hline
    \emph{Constants} & In the simplest case the values of a token take the same value, for example the header field \verb=host= when monitoring a particular web host. \\
    \hline
    \emph{Enumerations} &  A second type of tokens carries data that takes on only a small set of values dependent either on the HTTP protocol itself or on the web application. An example of such token is the \verb=accept-language= header.\\
    \hline
    \emph{Machine input} & The third type of tokens comprises machine-generated data, such as session numbers, identifiers and cookies. \\
    \hline
    \emph{Human input} & The most complex token type is induced by human input, such as free-text fields, querystrings, comments and names. The entered data does not exhibit any semantical structure except for being generated by a natural language.  \\
    \hline
  \end{tabular}
  \caption{Kategorien für Bestandteile eines HTTP-Request  nach~\cite{Krueger2010}}
  \label{tab:tocdoc}
\end{table}

%\begin{align}
%  \overbrace{
%    \underbrace{http}_\text{Konstante} \underbrace{localhost:8080}_\text{maschine}\underbrace{seite}_\text{human}\underbrace{blog}_\text{maschine}\underbrace{search}_\text{maschine}\underbrace{=}_\text{Konstante}\underbrace{Hello}_\text{mensch}
%  }^\text{GET-Request}
%\end{align}

Die Auswertung der einzelnen Bestandteile erfolgte nicht mehr allein durch einen Abgleich mit Regeln bzw. regulären Ausdrücken sondern mittels sogenannten \emph{Anomalie-Detektoren}, die verschiedene Ausprägungen annehmen können. Beispielsweise lassen sich Bestandteile wie \emph{Konstanten} und \emph{Aufzählungen} effektiv mit Hilfe von einfachen Listen abgleichen. Die Anomalie beginnt wenn der entsprechende nicht in der Liste vorhanden ist.\\

\textcolor{bhtGray}{\ding{110} Beispiel} Der Protokoll-Bereich einer URL kann nur die Werte \verb=http=, \verb=https=, \verb=webdav= und ein paar weitere Werte\footnote{in Bezug auf Web-Server} annehmen, eine umfangreiche Analyse ist hier nicht notwendig. \\

Bei \emph{maschinellen} und \emph{menschlichen Eingaben} nutzen die Anomaliedetektoren nun auch Methoden aus dem Bereich der Stochastik und der Computerlinguistik. In ihrer Firewall wurden für diesen Zweck drei Anomaliedetektoren implementiert die die Anfragen an den Server und deren Inhalte überprüfen. Passen einzelne Bestandteile sprachlich nicht zueinander (\emph{N-gram Centroid Anomaly Detector}), entsprechen diese nicht der üblichen Form (\emph{Markov Chain Anomaly Detector}) oder der üblichen Länge (\emph{Length Anomaly Detector}) versucht das System von~\cite{Krueger2010} diese Bestandteile zu \glqq\emph{reparieren}\grqq bevor sie an die Anwendung weitergeleitet werden. Je nach Konfiguration bedeutet die Reparatur eines Bestandteils dass dieser \emph{verworfen}, \emph{verändert} oder \emph{ausgetauscht} wird. \anno{verwerfen und austauschen erklären?}


\anno{und weiter?}



% Krueger -> fehlerhafte Request werden automatisch korregiert und weiter geleitet
% Manaseer etc. (Artikel ist von 2018... besser ans Ende?)

\subsubsection{Fortschritte in Richtung Intelligenz}

Tiefere Einblicke in die Anwendung stochastischer Methoden bei der Erkennung von Angriffen auf Webanwendungen finden sich auch in~\cite{Giménez2015}. Die Doktorarbeit \glqq\emph{Study of Stochastic and Machine Learning Techniques for Anomaly-based Web Attack Detection}\grqq von Carmen Torrano Giménez stellt zwei stochastische Techniken und eine - auf maschinellem Lernen basierende - Technik gegenüber um deren Leistungsfähigkeit am praktischen Beispiel zu vergleichen. Dazu entwickelte sie einen eigenen Datensatz~\cite{csic2010} und nutzte diesen zum Training eines Algorithmus aus dem Bereich des überwachten Lernens.\\

\textcolor{bhtGray}{\ding{110} Beschreibung des Datensatzes (Auszug aus~\cite{csic2010})} The HTTP dataset CSIC 2010 contains the generated traffic targeted to an e- Commerce web application developed at our department. (...) The dataset is generated automatically and contains 36,000 normal requests and more than 25,000 anomalous requests. The HTTP requests are labeled as normal or anomalous and the dataset includes attacks such as SQL injection, buffer overflow, information gathering, files disclosure, CRLF injection, XSS, server side include, parameter tampering and so on.\\

Die Firewall \emph{erlernte} das als normal bzw. üblich geltende Verhalten und eine Konfiguration\footnote{z.B. über fest definierte Regeln} war (beschränkt auf die verwendete Anwendung) nicht mehr notwendig. Dabei wurden die notwendigen Daten für den Datensatz mit Hilfe von \emph{Dictionaries} für eine interne Anwendung generiert. \anno{nicht korrekt, CSIC webseite verweist auf pharos und w3af}

% kruegel gimenez appelt kozik testen mit ML Ansätzen

\subsubsection{Umgekehrt gedacht}

Dennis Appelt spezialisierte sich in seiner Arbeit~\cite{Appelt2016} auf das Testen von WAFs und die Vermeidung von SQL-Injections. Im Gegensatz zu Giménez verwendete er jedoch ML-Techniken zum Generieren von Angriffen auf zwei regelbasierte WAFs um deren Schutzmechanismen zu umgehen. Mit einem weiteren Proxy zwischen Anwendung und Datenbank wurde überprüft ob dementsprechende \emph{Angriffe} von der WAF abgefangen oder durchgelassen wurden. Wie bei \emph{Kruegel et al.} soll auch hier die WAF weiter optimiert (repariert/geheilt) werden. Dazu sollen die Regeln automatisch und mit Hilfe eines genetischen Algorithmus verbessert werden.

\subsubsection{Schwarmwissen}
\anno{Übergang ist irgendwie doof}
Zur weiteren Vereinfachung der Konfiguration von Web Application Firewalls schlugen \emph{Saher Manaseer} und \emph{Ahmad Al Hwaitat} in~\cite{Manaseer2018} vor, mehreren Instanzen den Datenaustausch über ein zentrales System zu ermöglichen. 


% Thema 2

%ansaetze und kombination?

% Thema 3 optional die andere seite
\subsection{hacking Wafs}

\begin{neu}
  Blick über den Tellerand? z.B. Fingerprinting von Interesse?
\end{neu}


% Zusammenfassung (ca. 0,5 Seiten)
\section{Zusammenfassung}

% ggf. ditaa tabelle ueber den Zeitverlauf der verschiedenen Arbeiten nach Attack-Defend-Muster

\anno{Tabellarische Schnellübersicht}
\begin{tabular}{lp{7cm}}
  \toprule
  Jahr & Ereignis\\
  \midrule
  2002 & modSecurity 1st release \\
  2003 & OWASP Top 10 erscheinen zum ersten Mal\\
  2006 & OWASP Core Rule Set \\
  2009 & Core Rule Set V2 \\
  2010 & TokDoc \\
  2015 & Gimemez Dataset \\
  2016 & CRS 3.0 CRS Sampling Mode \\
  2016 & Automatisierte Tests Appelt \\
  2018 & Centralized Manaseer\\
  2019 & OWASP API Security Top 10\\
  \bottomrule
\end{tabular}

Dataset

DARPA
CSIC 2010
ECML/PKDD

%%The first version of the OWASP Top 10 list was published in 2003. Updates followed in 2004, 2007, 2010, 2013 and 2017. The most recent update was published in 2021.