\chapter{Zusammenfassung und Ausblick}

\anno{5 Seiten} %Ref. 3

% TODO Zusammenfassung
% TODO Lessons learned

Das der Einsatz einer Web Application Firewall ein Sicherheitsgewinn für jegliche Anwendungen darstellt sollte nicht überraschen. Erfreulich ist jedoch dass eine Software wie WebCastellum auch Jahre nach der eigentlichen Einstellung ihrer Entwicklung und mit einer hohen Last an technischen Schulden auf einen aktuelleren Stand gebracht werden konnte. Im Rahmen dieser Arbeit 

\section{Lessons Learned}
Eine wesentliche Schwierigkeit bei der praktischen Umsetzung eines Projektes auf Basis der Programmiersprache Java mit Techniken aus dem Bereich des \emph{Maschinellen Lernens} ist, dass dieser Bereich sehr \glqq\emph{Python}\grqq-lastig daher kommt. Die meisten Beispiele, Bibliotheken und Tutorials nutzen Python als \emph{Sprache der Wahl}. Der häufig vertretene Standpunkt - man sollte die Sprache wählen, die sich für das jeweilige Projekt am besten eignet - steht für ein \emph{Migrationsprojekt} wie WebCastellum leider nicht zur Verfügung. Zwar ähneln sich beide Sprachen in ihren Paradigmen, vielen Konzepten und in ihrer Anwendung, aber allein die Wahl einer geeigneten Bibliothek mit Unterstützung des maschinellen Lernens ist in diesem Fall doch etwas schwieriger. Umso erfreulicher ist die Tatsache dass mit dem Artikel \glqq\emph{Java & KI - Das Beste zweier Welten}\grqq{} in der aktuellen Ausgabe des \emph{Javamagazin}s (10/23) die neue Serie \glqq\emph{Künstliche Intelligenz in der Praxis}\grqq{} startet. Offen bleibt dabei ob der Inhalt dieser Arbeit sich eventuell etwas \emph{unterschieden} hätte, wäre diese Serie etwas früher erschienen.\\

Die zweite Schwierigkeit bei der Arbeit war der enorme \emph{Leistungshunger} der Anwendung bei der Erstellung der Entscheidungsbäume. Das der Rechen- und Speicherbedarf bei Anwendungen dieser Art etwas höher ausfällt war zu erwarten, dass der mir zu Verfügung stehende Rechner bei einem Entscheidungsbaum mit gerade mal zwei bis drei Attributen (bezogen auf den CSIC2010-Datensatz) bereits an seine Grenzen geriet, eher überraschend. Für einen praktischen Einsatz der Software sollten in diesem Bereich dringend weitere Verbesserungen erfolgen.

% Ausblick
\section{Ausblick}

Am Ende dieser Arbeit soll es noch zu einem kleinen Ausblick auf aktuelle Entwicklungen


\begin{neu}
  Nacharbeit Überführung der Evaluation als arquillian test?!
  Bereitstellung als (neuere) WebCastellum Version und Bereitstellung des zentralen Sammelpunktes.
\end{neu}

% https://www.dev-insider.de/wie-angreifbar-ist-webgoat-wirklich-a-7a385bc1e48bb04b7fdc9034029555e1/