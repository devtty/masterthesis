\chapter{Einleitung}

\anno{ca. 5 Seiten}

\begin{neu}
lässt sich hier eine Einleitung aus den Papers ableiten? ggf. Master Expose mit hinzuziehen!
  
\end{neu}
Jährlich erscheint mit den OWASP Top Ten \cite{owasptopten}  eine Liste der meist genutzten und bekannten Sicherheitsrisiken für Webanwendungen. In den meisten Fällen wird eine Anwendung sehr indiviuell gegen solche Risiken abgesichert und häufig wird dabei auch jedes einzelne Risiko separat adressiert. WebApplicationFirewalls (WAF) bieten einen relativ einfachen Weg einen Großteil der Risiken allgemein auszuschließen. Die Konfiguration einer WebApplicationFirewall muß jedoch auf die jeweilig zu schützende Anwendung angepasst werden. Mit dieser Masterarbeit soll eine Möglichkeit zur einfacheren Einbindung dieser Schutzmaßnahme bereitgestellt werden. Es wird gezeigt wie eine ältere WAF-Software auf einen neuen Stand gebracht wird und mehrere Instanzen dieser Software gekoppelt werden können. Mittels selbständiger Kategorisierung wird dem Nutzer die individuelle Konfiguration abgenommen und ein messbarer Gewinn an Sicherheit für viele Anwendungen bereitgestellt. \anno{aus Exposee übernommen}

% das grosse Problem
%% Was ist der Markt
%%% zahlreiche Anbieter teils spezialisiert auf bestimmte Bereiche; zahlreiche Arten von WAF
%% Wem hilft es?
%% Warum jetzt?
%% Ist das Problem lösbarer geworden?

\subsection{Zielsetzung}
% Ziel der Arbeit
%% Was ist Ihr relevantes Teilproblem? Was ist die zentrale Frage?
%% Ihr Ziel in zwei Absätzen.
%%% Warum genau dieses Problem?
%%% Ist Ihr Beitrag völlig neu, oder nur ein Baustein?
%%% Ist Ihr Proble schwer zu lösen oder straight forward?
%%% Eher Forschung oder Anwendung?
%%% Was machen Sie nicht? Und warum haben Sie sich entschieden das nicht zu machen.
%% Wenn Sie ein System bauen...
%%% Welche Anfragen/Aufgaben wollen Sie beantworten/lösen können?
%%% Welche Kernfunktionalität soll Ihr System haben?
%%% Was ist ein typischer (Bedienungs-)Prozess für Ihr System

% Methodik (1 Seite)
%% Wie wollen Sie das Problem lösen?
%% Welche Grundlagen müssen Sie beachten?
%% Wie ist Ihre Vorgehensweise?

% Gliederung und Aufbau (ca. 0,5 Seiten)
%% Wann lesen wir was und warum?
\subsection{Aufbau der Arbeit}

% erster Teile Grundbegriffe, verwandte Vorarbeiten und

Dieses \textbf{erste Kapitel} der Arbeit lieferte bereits einen kurzen Einblick in die Thematik. Das folgendende \textbf{zweite Kapitel} erläutert noch wichtige Grundbegriffe aus den Bereichen der IT-Sicherheit.