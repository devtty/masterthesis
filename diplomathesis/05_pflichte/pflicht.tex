\section{Ziel}

Zu entwickeln ist ein Lehrsystem das als zentrale Komponente einen Chat enth\"alt. Das System
soll insbesondere dem Lehreinsatz an Schulen, Hochschulen und sonstigen Instituten der
Lehre dienen in dem es Teile der Kommunikation auf das Medium Internet ver- bzw. auslagert.
 Desweiteren soll das System Informationen sammeln und gegebenenfalls der \"Offentlichkeit
zur Verf\"ugung stellen. 

%------------------------------------------------------------------------------------------
\section{Funktionen des Gesamtsystems}
Das System selbst gliedert sich in mehrere Bereiche, welche mehr oder weniger 
zum Umfeld des Chats geh\"oren. Die folgend aufgef\"uhrten Hauptmerkmale des Chats sind so oder
in \"ahnlicher Weise auch in fast jedem anderen Chat, den man im Internet findet, verf\"ugbar.

\subsection{Chatten}
Hauptfunktion der Anwendung ist das Chatten. Der Begriff Chatten stammt aus dem Englischen
und bedeutet soviel wie ``Plaudern''. Chatten erm\"oglicht demzufolge Plaudereien bzw. Kommunikation
innerhalb einer Gruppe.

\subsection{User-Management}



\subsection{Klassenr\"aume}



\subsection{Benutzer}

\subsection{Protokolle}

Einen weiteren wichtigen Punkt stellen die Protokolle der Chatsitzungen dar. Im Klassenverband
kann es schon vorkommen das Sch\"uler am Unterricht verhindert sind und trotzdem den Inhalt
der Sitzung im nachhinein verarbeiten sollen. Aber auch demjenigen der am Unterricht teilnahm,
insbesondere der Lehrkraft, sollte die M\"oglichkeit der Protokolleinsicht gew\"ahrt sein. Ein
weiterer Grund f\"ur die Notwendigkeit von Protokollen ist das Ziel der Sammlung von Informationen,
da diese zwangsl\"aufig w\"ahrend des Unterrichts anfallen und im Chat diskutiert werden sollen.

Protokolle sollten aber kein Zwang sein, da zum Beispiel Sch\"ulern auch die M\"oglichkeit des 
Chattens au{\ss}erhalb der Unterrichtszeiten geben wird. Die Entscheidung \"uber die Erstellung 
eines Protokolls liegt ausschlie{\ss}lich bei der Lehrkraft und sollte nur w\"ahrend der aktiven 
Unterrichtsstunde erfolgen. In diesem Fall jedoch mu{\ss} das Protokoll verl\"a{\ss}lich sein, d.h.
auch wenn die Lehrkraft w\"ahrend einer Unterrichtsstunde den Raum verl\"a{\ss}t sollte das Protokoll 
weiterlaufen.



%------------------------------------------------------------------------------------------
\section{Funktionen des Chats}

\subsection{Fl\"ustern}

Das Fl\"ustern sollte die M\"oglichkeit einer privaten Unterhaltung zwischen zwei Chatteilnehmern
erm\"oglichen ohne andere Chatteilnehmer mit einzubeziehen. Gefl\"usterte Kommunikation darf nicht
in die Unterrichtsprotokolle aufgenommen werden, da dies einer Verletzung der Privatsph\"are 
entsprechen w\"urde. Allerdings sollte der Lehrkraft die M\"oglichkeit gegeben werden das Fl\"ustern
zu unterbinden unter der Vorraussetzung dies vor Beginn der Chatsitzung bzw. allgemein innerhalb
eines Klassenverbandes zu tun. Dabei sollte die Lehrkraft aber auch ber\"ucksichtigen das das
Fl\"ustern unter Umst\"anden zu einer Verbesserung des Unterrichts beitragen kann, da unterrichtsfremde
Diskussionen nicht offensichtlich entstehen und demzufolge nicht f\"ur alle Chatteilnehmer lesbar
sind. Dies erm\"oglicht auch eine bessere \"Ubersichtlichkeit im Chat bei hohen Teilnehmerzahlen.


%-------------------------------------------------------------------------------------------
\subsection{Ignorieren}
Ein weitere Funktion die jedem Chatteilnehmer zur Verf\"ugung stehen sollte ist das Ignorieren
anderer Teilnehmer. Das System an sich soll zwar der Lehre und Ausbildung dienen und bietet
auch nicht den Grad an Anonymit\"at von \"offentlichen Chatsystemen aber dennoch kann es
vorkommen das mal ein Teilnehmer, vor allem in j\"ungeren Altersgruppen, mit allen Mitteln die 
Aufmerksamkeit auf sich lenken m\"ochte. Weit verbreitete Techniken um, innerhalb des Chats, 
Aufmerksamkeit zu erlangen sind zum Beispiel die Provokation anderer Chatteilnehmer oder das 
\"Uberfluten des Chats mit sinnlosen S\"atzen oder Zeichenfolgen. Diese Art von St\"orung l\"ost 
h\"aufig eine Massenflucht aus dem Chatraum oder Gegenattacken in Form von Beschwerden aus und 
der St\"orer selbst hat seine gew\"unschte Aufmerksamkeit erlangt und sieht sich in seinem Handeln 
best\"atigt.

Im Fall das dieser St\"orer alle anderen Chatteilnehmer mit einbezieht kann der St\"orer aus dem Chat 
verbannt (siehe \ref{sec:bannen}) werden. Sollte er aber zum Beispiel durch gezieltes Fl\"ustern
provozieren hat der betroffene Teilnehmer nicht das Recht ihn vom gesamten Chat auszuschliessen. In
diesem Fall sollte der Betroffene dennoch die M\"oglichkeit erhalten sich vor solchen St\"orungen
zu sch\"utzen in dem er den St\"orer einfach f\"ur sich pers\"onlich aus dem Chat ausschlie{\ss}t
bzw. ignoriert.

%-------------------------------------------------------------------------------------------
\subsection{Kicken}
Sehr beliebt in bereits existierenden Chatl\"osungen ist das Kicken bzw. Ausschlie{\ss}en von Personen
aus dem Chat. Diese Funktion sollte aber nicht allen Chatteilnehmern zur Verf\"ugung stehen
sondern nur demjenigen der die Funktion des Moderators im Chat \"ubernimmt, dem Lehrenden bzw.
der Person die den Chatraum er\"offnete.


%-------------------------------------------------------------------------------------------
\subsection{\label{sec:bannen}Bannen}

Im realen Unterricht hat der Lehrende die freie Wahl wie er seinen Unterricht gestaltet und das
Kicken als Mittel zur Schaffung von Autorit\"at ist nicht immer der bessere Weg (s. \ref{tab:lewinstile}).