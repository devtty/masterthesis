\documentclass[a4paper, 11pt]{report}
\usepackage{times, mathptm}
\usepackage{ngerman}
\usepackage[latin1]{inputenc}
\usepackage{graphicx}
%\usepackage{anysize}
\sloppy

\title{{\sc Teaching By Chat\\Entwicklung eines Lehrsystems}
\\[10ex]
  Diplomarbeit \\[7ex]
  {\small zur Erlangung des akademischen Grades\\}
  Diplom-Informatiker (FH)\\[5ex]
  {\small
  an der\\
  Fachhochschule f\"{u}r Technik und Wirtschaft Berlin\\
  Fachbereich Wirtschaftswissenschaften II\\
  Studiengang Internationale Medieninformatik\\
  }
}

\author{
  1. Betreuer: Prof. Dr. Weber-Wulff\\
  2. Betreuer: Prof. Dr.-Ing. Barthel\\
  Eingereicht von Denis Renning\\[5ex]
  }

\date{Berlin, \today}

%Befehle, die den Zeilenabstand im Text, bzw. in
%Tabellen ver�ndern k�nnen.
%Fakor 1.0 ist default, also sind sie Befehle
%im Moment wirkungslos...
\renewcommand{\baselinestretch}{1.0}
\renewcommand{\arraystretch}{1.0}


\renewcommand{\textwidth}{15.0cm}

%Befehle, die Abstand und Einr�ckung eines neuen
%Paragraphen bestimmen. Der Abstand (parskip) ist dabei
%ein sogenanntes dehnbares Ma�, die Einr�ckung (parindent)
%jedoch nicht. Es werden keine absoluten Ma�e (wie cm)
%verwendet, sondern ex und em (H�he eines "x" bzw.
%Breite eines sog. Geviertstrichs "-")
\setlength{\parskip}{1.5ex plus 0.5ex minus 0.5ex}
\setlength{\parindent}{1.5em}

%\oddsidemargin 0.5in
%\evensidemargin 0.5in
%\topmargin 0.5pt
%\textheight 8.1in
%\textwidth 6in


%Hiergeht's los:
\begin{document}

\DeclareGraphicsExtensions{.gif}


%Erstellt das Titelblatt:
\maketitle

%Inhaltsverzeichnis soll r�misch numeriert sein:
%\pagenumbering{roman}\setcounter{page}{1}

%Erstellt das Inhaltsverzeichnis:
\tableofcontents

\newpage



%Der Rest soll arabisch numeriert sein:
%\pagenumbering{arabic}\setcounter{page}{1}
%Die Einleitung hat keine Kapitelnummer, daher \chapter*...
\chapter*{Einleitung}
\label{kap:einleitung}

%...es soll aber im Inhaltsverzeichnis auftauchen:
\addcontentsline{toc}{chapter}{Einleitung}

%Die Kopfzeile f�r die Einleitung:
%\markright{\sc Einleitung}


Das Thema e-Learning
Diese Diplomarbeit dient der Entwicklung eines Lehrsystems mit dem sich eine Unterrichtsstunde
im Internet halten l\"{a}sst.

%Informationen, die nicht f�r LaTeX bestimmt sind, die aber
%wichtig f�r AUCTeX und REFTeX sind:
 
% Local Variables:
% TeX-master: "diplom"
% End:


%Systementwurf
%Modulhierarchie, Modulspezifikationen, Struktogramme/Pseudocode 
\chapter{Systementwurf}
\section{Basismodell}

Um eine erste Basis f\"ur das Gesamtsystem zu schaffen sollten erst die real
existierenden Bedingungen umgesetzt werden, um die in sp\"ateren Phasen das System
aufgebaut wird. Anwendungen, egal ob Software oder in anderen Bereichen, sollten
sich immer den Bed\"urfnissen des Benutzers anpassen, dazu mu{\ss} dem sp\"ateren
System der Benutzer auch bekannt sein. Da Softwareentwicklung auch viel mit der realen
Welt zu tun hat ,und eigentlich durch Abstraktion dieser erst entsteht, wird der
Benutzer, mit den Eigenschaften die f\"ur das Lehrsystem als wichtig erachtet werden,
als Erster entworfen bzw. von der real existierenden Welt \"ubernommen.

%------------------------------------------------------------------------------
\subsection{Teilnehmende Personen}

Erster Schritt zur L\"osung ist das Aufsp\"uren von Akteuren, dabei gilt es Personen, 
Objekte und andere Ausl\"oser von Ereignissen zu finden. Bei den Personen ist dies
wesentlich einfacher als in anderen Bereichen, wahrscheinlich weil der reale Bezug
hier noch sehr stark vorhanden ist, da diese Personen existieren. Alle Akteure
definieren sich ebenfalls durch ihre Eigenschaften bzw. Attribute. Eine gute Quelle
f\"ur Akteure sind immer Dokumente aus den Vor-Entwicklungs-Phasen wie Produktanforderungen,
Pflichtenhefte oder Dokumente der Analyse.

Die Personen, als wichtigste Akteure, sind bereits bekannt: der {\sl Lerner}, der {\sl Lehrende},
der {\sl Administrator} und der {\sl Hospitant}. 

Im n\"achsten Schritt versucht man m\"oglichst viele F\"alle der Anwendung f\"ur die
Akteure herauszuarbeiten. Ideal f\"ur diesen Arbeitschritt ist das Anwendungsfall - Diagramm,
auch {\it Use Case Diagramm} genannt, das von der UML bereitgestellt wird. Im Diagramm
lassen sich Akteure, Anwendungsf\"alle und ihre Beziehungen zueinander in einfacher Weise 
grafisch erfassen und \"ubersichtlich darstellen.
Das UML-Tool Together erlaubt zus\"atzlich eine weitere Verlinkung von Akteuren und Anwendungsf\"allen
mit anderen Anwendungsf\"all - Diagrammen, dadurch lassen sich die Anwendungsf\"alle weiter
spezialisieren. 

\begin{figure}[h]
%\includegraphics[viewport=0 0 376 185]{systemen/grob}
\includegraphics[width=15cm]{systemen/grob}
\caption{Funktionalit\"at (erster Entwurf)}
\end{figure}


%------------------------------------------------------------------------------

\subsection{Klassen}

%------------------------------------------------------------------------------

\subsection{Protokolle}

%------------------------------------------------------------------------------
%Systementwurf
%Modulhierarchie, Modulspezifikationen, Struktogramme/Pseudocode 
\chapter{Systementwurf}
\section{Basismodell}

Um eine erste Basis f\"ur das Gesamtsystem zu schaffen sollten erst die real
existierenden Bedingungen umgesetzt werden, um die in sp\"ateren Phasen das System
aufgebaut wird. Anwendungen, egal ob Software oder in anderen Bereichen, sollten
sich immer den Bed\"urfnissen des Benutzers anpassen, dazu mu{\ss} dem sp\"ateren
System der Benutzer auch bekannt sein. Da Softwareentwicklung auch viel mit der realen
Welt zu tun hat ,und eigentlich durch Abstraktion dieser erst entsteht, wird der
Benutzer, mit den Eigenschaften die f\"ur das Lehrsystem als wichtig erachtet werden,
als Erster entworfen bzw. von der real existierenden Welt \"ubernommen.

%------------------------------------------------------------------------------
\subsection{Teilnehmende Personen}

Erster Schritt zur L\"osung ist das Aufsp\"uren von Akteuren, dabei gilt es Personen, 
Objekte und andere Ausl\"oser von Ereignissen zu finden. Bei den Personen ist dies
wesentlich einfacher als in anderen Bereichen, wahrscheinlich weil der reale Bezug
hier noch sehr stark vorhanden ist, da diese Personen existieren. Alle Akteure
definieren sich ebenfalls durch ihre Eigenschaften bzw. Attribute. Eine gute Quelle
f\"ur Akteure sind immer Dokumente aus den Vor-Entwicklungs-Phasen wie Produktanforderungen,
Pflichtenhefte oder Dokumente der Analyse.

Die Personen, als wichtigste Akteure, sind bereits bekannt: der {\sl Lerner}, der {\sl Lehrende},
der {\sl Administrator} und der {\sl Hospitant}. 

Im n\"achsten Schritt versucht man m\"oglichst viele F\"alle der Anwendung f\"ur die
Akteure herauszuarbeiten. Ideal f\"ur diesen Arbeitschritt ist das Anwendungsfall - Diagramm,
auch {\it Use Case Diagramm} genannt, das von der UML bereitgestellt wird. Im Diagramm
lassen sich Akteure, Anwendungsf\"alle und ihre Beziehungen zueinander in einfacher Weise 
grafisch erfassen und \"ubersichtlich darstellen.
Das UML-Tool Together erlaubt zus\"atzlich eine weitere Verlinkung von Akteuren und Anwendungsf\"allen
mit anderen Anwendungsf\"all - Diagrammen, dadurch lassen sich die Anwendungsf\"alle weiter
spezialisieren. 

\begin{figure}[h]
%\includegraphics[viewport=0 0 376 185]{systemen/grob}
\includegraphics[width=15cm]{systemen/grob}
\caption{Funktionalit\"at (erster Entwurf)}
\end{figure}


%------------------------------------------------------------------------------

\subsection{Klassen}

%------------------------------------------------------------------------------

\subsection{Protokolle}

%------------------------------------------------------------------------------
%Systementwurf
%Modulhierarchie, Modulspezifikationen, Struktogramme/Pseudocode 
\chapter{Systementwurf}
\section{Basismodell}

Um eine erste Basis f\"ur das Gesamtsystem zu schaffen sollten erst die real
existierenden Bedingungen umgesetzt werden, um die in sp\"ateren Phasen das System
aufgebaut wird. Anwendungen, egal ob Software oder in anderen Bereichen, sollten
sich immer den Bed\"urfnissen des Benutzers anpassen, dazu mu{\ss} dem sp\"ateren
System der Benutzer auch bekannt sein. Da Softwareentwicklung auch viel mit der realen
Welt zu tun hat ,und eigentlich durch Abstraktion dieser erst entsteht, wird der
Benutzer, mit den Eigenschaften die f\"ur das Lehrsystem als wichtig erachtet werden,
als Erster entworfen bzw. von der real existierenden Welt \"ubernommen.

%------------------------------------------------------------------------------
\subsection{Teilnehmende Personen}

Erster Schritt zur L\"osung ist das Aufsp\"uren von Akteuren, dabei gilt es Personen, 
Objekte und andere Ausl\"oser von Ereignissen zu finden. Bei den Personen ist dies
wesentlich einfacher als in anderen Bereichen, wahrscheinlich weil der reale Bezug
hier noch sehr stark vorhanden ist, da diese Personen existieren. Alle Akteure
definieren sich ebenfalls durch ihre Eigenschaften bzw. Attribute. Eine gute Quelle
f\"ur Akteure sind immer Dokumente aus den Vor-Entwicklungs-Phasen wie Produktanforderungen,
Pflichtenhefte oder Dokumente der Analyse.

Die Personen, als wichtigste Akteure, sind bereits bekannt: der {\sl Lerner}, der {\sl Lehrende},
der {\sl Administrator} und der {\sl Hospitant}. 

Im n\"achsten Schritt versucht man m\"oglichst viele F\"alle der Anwendung f\"ur die
Akteure herauszuarbeiten. Ideal f\"ur diesen Arbeitschritt ist das Anwendungsfall - Diagramm,
auch {\it Use Case Diagramm} genannt, das von der UML bereitgestellt wird. Im Diagramm
lassen sich Akteure, Anwendungsf\"alle und ihre Beziehungen zueinander in einfacher Weise 
grafisch erfassen und \"ubersichtlich darstellen.
Das UML-Tool Together erlaubt zus\"atzlich eine weitere Verlinkung von Akteuren und Anwendungsf\"allen
mit anderen Anwendungsf\"all - Diagrammen, dadurch lassen sich die Anwendungsf\"alle weiter
spezialisieren. 

\begin{figure}[h]
%\includegraphics[viewport=0 0 376 185]{systemen/grob}
\includegraphics[width=15cm]{systemen/grob}
\caption{Funktionalit\"at (erster Entwurf)}
\end{figure}


%------------------------------------------------------------------------------

\subsection{Klassen}

%------------------------------------------------------------------------------

\subsection{Protokolle}

%------------------------------------------------------------------------------
%Systementwurf
%Modulhierarchie, Modulspezifikationen, Struktogramme/Pseudocode 
\chapter{Systementwurf}
\section{Basismodell}

Um eine erste Basis f\"ur das Gesamtsystem zu schaffen sollten erst die real
existierenden Bedingungen umgesetzt werden, um die in sp\"ateren Phasen das System
aufgebaut wird. Anwendungen, egal ob Software oder in anderen Bereichen, sollten
sich immer den Bed\"urfnissen des Benutzers anpassen, dazu mu{\ss} dem sp\"ateren
System der Benutzer auch bekannt sein. Da Softwareentwicklung auch viel mit der realen
Welt zu tun hat ,und eigentlich durch Abstraktion dieser erst entsteht, wird der
Benutzer, mit den Eigenschaften die f\"ur das Lehrsystem als wichtig erachtet werden,
als Erster entworfen bzw. von der real existierenden Welt \"ubernommen.

%------------------------------------------------------------------------------
\subsection{Teilnehmende Personen}

Erster Schritt zur L\"osung ist das Aufsp\"uren von Akteuren, dabei gilt es Personen, 
Objekte und andere Ausl\"oser von Ereignissen zu finden. Bei den Personen ist dies
wesentlich einfacher als in anderen Bereichen, wahrscheinlich weil der reale Bezug
hier noch sehr stark vorhanden ist, da diese Personen existieren. Alle Akteure
definieren sich ebenfalls durch ihre Eigenschaften bzw. Attribute. Eine gute Quelle
f\"ur Akteure sind immer Dokumente aus den Vor-Entwicklungs-Phasen wie Produktanforderungen,
Pflichtenhefte oder Dokumente der Analyse.

Die Personen, als wichtigste Akteure, sind bereits bekannt: der {\sl Lerner}, der {\sl Lehrende},
der {\sl Administrator} und der {\sl Hospitant}. 

Im n\"achsten Schritt versucht man m\"oglichst viele F\"alle der Anwendung f\"ur die
Akteure herauszuarbeiten. Ideal f\"ur diesen Arbeitschritt ist das Anwendungsfall - Diagramm,
auch {\it Use Case Diagramm} genannt, das von der UML bereitgestellt wird. Im Diagramm
lassen sich Akteure, Anwendungsf\"alle und ihre Beziehungen zueinander in einfacher Weise 
grafisch erfassen und \"ubersichtlich darstellen.
Das UML-Tool Together erlaubt zus\"atzlich eine weitere Verlinkung von Akteuren und Anwendungsf\"allen
mit anderen Anwendungsf\"all - Diagrammen, dadurch lassen sich die Anwendungsf\"alle weiter
spezialisieren. 

\begin{figure}[h]
%\includegraphics[viewport=0 0 376 185]{systemen/grob}
\includegraphics[width=15cm]{systemen/grob}
\caption{Funktionalit\"at (erster Entwurf)}
\end{figure}


%------------------------------------------------------------------------------

\subsection{Klassen}

%------------------------------------------------------------------------------

\subsection{Protokolle}

%------------------------------------------------------------------------------
%Systementwurf
%Modulhierarchie, Modulspezifikationen, Struktogramme/Pseudocode 
\chapter{Systementwurf}
\section{Basismodell}

Um eine erste Basis f\"ur das Gesamtsystem zu schaffen sollten erst die real
existierenden Bedingungen umgesetzt werden, um die in sp\"ateren Phasen das System
aufgebaut wird. Anwendungen, egal ob Software oder in anderen Bereichen, sollten
sich immer den Bed\"urfnissen des Benutzers anpassen, dazu mu{\ss} dem sp\"ateren
System der Benutzer auch bekannt sein. Da Softwareentwicklung auch viel mit der realen
Welt zu tun hat ,und eigentlich durch Abstraktion dieser erst entsteht, wird der
Benutzer, mit den Eigenschaften die f\"ur das Lehrsystem als wichtig erachtet werden,
als Erster entworfen bzw. von der real existierenden Welt \"ubernommen.

%------------------------------------------------------------------------------
\subsection{Teilnehmende Personen}

Erster Schritt zur L\"osung ist das Aufsp\"uren von Akteuren, dabei gilt es Personen, 
Objekte und andere Ausl\"oser von Ereignissen zu finden. Bei den Personen ist dies
wesentlich einfacher als in anderen Bereichen, wahrscheinlich weil der reale Bezug
hier noch sehr stark vorhanden ist, da diese Personen existieren. Alle Akteure
definieren sich ebenfalls durch ihre Eigenschaften bzw. Attribute. Eine gute Quelle
f\"ur Akteure sind immer Dokumente aus den Vor-Entwicklungs-Phasen wie Produktanforderungen,
Pflichtenhefte oder Dokumente der Analyse.

Die Personen, als wichtigste Akteure, sind bereits bekannt: der {\sl Lerner}, der {\sl Lehrende},
der {\sl Administrator} und der {\sl Hospitant}. 

Im n\"achsten Schritt versucht man m\"oglichst viele F\"alle der Anwendung f\"ur die
Akteure herauszuarbeiten. Ideal f\"ur diesen Arbeitschritt ist das Anwendungsfall - Diagramm,
auch {\it Use Case Diagramm} genannt, das von der UML bereitgestellt wird. Im Diagramm
lassen sich Akteure, Anwendungsf\"alle und ihre Beziehungen zueinander in einfacher Weise 
grafisch erfassen und \"ubersichtlich darstellen.
Das UML-Tool Together erlaubt zus\"atzlich eine weitere Verlinkung von Akteuren und Anwendungsf\"allen
mit anderen Anwendungsf\"all - Diagrammen, dadurch lassen sich die Anwendungsf\"alle weiter
spezialisieren. 

\begin{figure}[h]
%\includegraphics[viewport=0 0 376 185]{systemen/grob}
\includegraphics[width=15cm]{systemen/grob}
\caption{Funktionalit\"at (erster Entwurf)}
\end{figure}


%------------------------------------------------------------------------------

\subsection{Klassen}

%------------------------------------------------------------------------------

\subsection{Protokolle}

%------------------------------------------------------------------------------
%Systementwurf
%Modulhierarchie, Modulspezifikationen, Struktogramme/Pseudocode 
\chapter{Systementwurf}
\section{Basismodell}

Um eine erste Basis f\"ur das Gesamtsystem zu schaffen sollten erst die real
existierenden Bedingungen umgesetzt werden, um die in sp\"ateren Phasen das System
aufgebaut wird. Anwendungen, egal ob Software oder in anderen Bereichen, sollten
sich immer den Bed\"urfnissen des Benutzers anpassen, dazu mu{\ss} dem sp\"ateren
System der Benutzer auch bekannt sein. Da Softwareentwicklung auch viel mit der realen
Welt zu tun hat ,und eigentlich durch Abstraktion dieser erst entsteht, wird der
Benutzer, mit den Eigenschaften die f\"ur das Lehrsystem als wichtig erachtet werden,
als Erster entworfen bzw. von der real existierenden Welt \"ubernommen.

%------------------------------------------------------------------------------
\subsection{Teilnehmende Personen}

Erster Schritt zur L\"osung ist das Aufsp\"uren von Akteuren, dabei gilt es Personen, 
Objekte und andere Ausl\"oser von Ereignissen zu finden. Bei den Personen ist dies
wesentlich einfacher als in anderen Bereichen, wahrscheinlich weil der reale Bezug
hier noch sehr stark vorhanden ist, da diese Personen existieren. Alle Akteure
definieren sich ebenfalls durch ihre Eigenschaften bzw. Attribute. Eine gute Quelle
f\"ur Akteure sind immer Dokumente aus den Vor-Entwicklungs-Phasen wie Produktanforderungen,
Pflichtenhefte oder Dokumente der Analyse.

Die Personen, als wichtigste Akteure, sind bereits bekannt: der {\sl Lerner}, der {\sl Lehrende},
der {\sl Administrator} und der {\sl Hospitant}. 

Im n\"achsten Schritt versucht man m\"oglichst viele F\"alle der Anwendung f\"ur die
Akteure herauszuarbeiten. Ideal f\"ur diesen Arbeitschritt ist das Anwendungsfall - Diagramm,
auch {\it Use Case Diagramm} genannt, das von der UML bereitgestellt wird. Im Diagramm
lassen sich Akteure, Anwendungsf\"alle und ihre Beziehungen zueinander in einfacher Weise 
grafisch erfassen und \"ubersichtlich darstellen.
Das UML-Tool Together erlaubt zus\"atzlich eine weitere Verlinkung von Akteuren und Anwendungsf\"allen
mit anderen Anwendungsf\"all - Diagrammen, dadurch lassen sich die Anwendungsf\"alle weiter
spezialisieren. 

\begin{figure}[h]
%\includegraphics[viewport=0 0 376 185]{systemen/grob}
\includegraphics[width=15cm]{systemen/grob}
\caption{Funktionalit\"at (erster Entwurf)}
\end{figure}


%------------------------------------------------------------------------------

\subsection{Klassen}

%------------------------------------------------------------------------------

\subsection{Protokolle}

%------------------------------------------------------------------------------
%Systementwurf
%Modulhierarchie, Modulspezifikationen, Struktogramme/Pseudocode 
\chapter{Systementwurf}
\section{Basismodell}

Um eine erste Basis f\"ur das Gesamtsystem zu schaffen sollten erst die real
existierenden Bedingungen umgesetzt werden, um die in sp\"ateren Phasen das System
aufgebaut wird. Anwendungen, egal ob Software oder in anderen Bereichen, sollten
sich immer den Bed\"urfnissen des Benutzers anpassen, dazu mu{\ss} dem sp\"ateren
System der Benutzer auch bekannt sein. Da Softwareentwicklung auch viel mit der realen
Welt zu tun hat ,und eigentlich durch Abstraktion dieser erst entsteht, wird der
Benutzer, mit den Eigenschaften die f\"ur das Lehrsystem als wichtig erachtet werden,
als Erster entworfen bzw. von der real existierenden Welt \"ubernommen.

%------------------------------------------------------------------------------
\subsection{Teilnehmende Personen}

Erster Schritt zur L\"osung ist das Aufsp\"uren von Akteuren, dabei gilt es Personen, 
Objekte und andere Ausl\"oser von Ereignissen zu finden. Bei den Personen ist dies
wesentlich einfacher als in anderen Bereichen, wahrscheinlich weil der reale Bezug
hier noch sehr stark vorhanden ist, da diese Personen existieren. Alle Akteure
definieren sich ebenfalls durch ihre Eigenschaften bzw. Attribute. Eine gute Quelle
f\"ur Akteure sind immer Dokumente aus den Vor-Entwicklungs-Phasen wie Produktanforderungen,
Pflichtenhefte oder Dokumente der Analyse.

Die Personen, als wichtigste Akteure, sind bereits bekannt: der {\sl Lerner}, der {\sl Lehrende},
der {\sl Administrator} und der {\sl Hospitant}. 

Im n\"achsten Schritt versucht man m\"oglichst viele F\"alle der Anwendung f\"ur die
Akteure herauszuarbeiten. Ideal f\"ur diesen Arbeitschritt ist das Anwendungsfall - Diagramm,
auch {\it Use Case Diagramm} genannt, das von der UML bereitgestellt wird. Im Diagramm
lassen sich Akteure, Anwendungsf\"alle und ihre Beziehungen zueinander in einfacher Weise 
grafisch erfassen und \"ubersichtlich darstellen.
Das UML-Tool Together erlaubt zus\"atzlich eine weitere Verlinkung von Akteuren und Anwendungsf\"allen
mit anderen Anwendungsf\"all - Diagrammen, dadurch lassen sich die Anwendungsf\"alle weiter
spezialisieren. 

\begin{figure}[h]
%\includegraphics[viewport=0 0 376 185]{systemen/grob}
\includegraphics[width=15cm]{systemen/grob}
\caption{Funktionalit\"at (erster Entwurf)}
\end{figure}


%------------------------------------------------------------------------------

\subsection{Klassen}

%------------------------------------------------------------------------------

\subsection{Protokolle}

%------------------------------------------------------------------------------
%Systementwurf
%Modulhierarchie, Modulspezifikationen, Struktogramme/Pseudocode 
\chapter{Systementwurf}
\section{Basismodell}

Um eine erste Basis f\"ur das Gesamtsystem zu schaffen sollten erst die real
existierenden Bedingungen umgesetzt werden, um die in sp\"ateren Phasen das System
aufgebaut wird. Anwendungen, egal ob Software oder in anderen Bereichen, sollten
sich immer den Bed\"urfnissen des Benutzers anpassen, dazu mu{\ss} dem sp\"ateren
System der Benutzer auch bekannt sein. Da Softwareentwicklung auch viel mit der realen
Welt zu tun hat ,und eigentlich durch Abstraktion dieser erst entsteht, wird der
Benutzer, mit den Eigenschaften die f\"ur das Lehrsystem als wichtig erachtet werden,
als Erster entworfen bzw. von der real existierenden Welt \"ubernommen.

%------------------------------------------------------------------------------
\subsection{Teilnehmende Personen}

Erster Schritt zur L\"osung ist das Aufsp\"uren von Akteuren, dabei gilt es Personen, 
Objekte und andere Ausl\"oser von Ereignissen zu finden. Bei den Personen ist dies
wesentlich einfacher als in anderen Bereichen, wahrscheinlich weil der reale Bezug
hier noch sehr stark vorhanden ist, da diese Personen existieren. Alle Akteure
definieren sich ebenfalls durch ihre Eigenschaften bzw. Attribute. Eine gute Quelle
f\"ur Akteure sind immer Dokumente aus den Vor-Entwicklungs-Phasen wie Produktanforderungen,
Pflichtenhefte oder Dokumente der Analyse.

Die Personen, als wichtigste Akteure, sind bereits bekannt: der {\sl Lerner}, der {\sl Lehrende},
der {\sl Administrator} und der {\sl Hospitant}. 

Im n\"achsten Schritt versucht man m\"oglichst viele F\"alle der Anwendung f\"ur die
Akteure herauszuarbeiten. Ideal f\"ur diesen Arbeitschritt ist das Anwendungsfall - Diagramm,
auch {\it Use Case Diagramm} genannt, das von der UML bereitgestellt wird. Im Diagramm
lassen sich Akteure, Anwendungsf\"alle und ihre Beziehungen zueinander in einfacher Weise 
grafisch erfassen und \"ubersichtlich darstellen.
Das UML-Tool Together erlaubt zus\"atzlich eine weitere Verlinkung von Akteuren und Anwendungsf\"allen
mit anderen Anwendungsf\"all - Diagrammen, dadurch lassen sich die Anwendungsf\"alle weiter
spezialisieren. 

\begin{figure}[h]
%\includegraphics[viewport=0 0 376 185]{systemen/grob}
\includegraphics[width=15cm]{systemen/grob}
\caption{Funktionalit\"at (erster Entwurf)}
\end{figure}


%------------------------------------------------------------------------------

\subsection{Klassen}

%------------------------------------------------------------------------------

\subsection{Protokolle}

%------------------------------------------------------------------------------
%Systementwurf
%Modulhierarchie, Modulspezifikationen, Struktogramme/Pseudocode 
\chapter{Systementwurf}
\section{Basismodell}

Um eine erste Basis f\"ur das Gesamtsystem zu schaffen sollten erst die real
existierenden Bedingungen umgesetzt werden, um die in sp\"ateren Phasen das System
aufgebaut wird. Anwendungen, egal ob Software oder in anderen Bereichen, sollten
sich immer den Bed\"urfnissen des Benutzers anpassen, dazu mu{\ss} dem sp\"ateren
System der Benutzer auch bekannt sein. Da Softwareentwicklung auch viel mit der realen
Welt zu tun hat ,und eigentlich durch Abstraktion dieser erst entsteht, wird der
Benutzer, mit den Eigenschaften die f\"ur das Lehrsystem als wichtig erachtet werden,
als Erster entworfen bzw. von der real existierenden Welt \"ubernommen.

%------------------------------------------------------------------------------
\subsection{Teilnehmende Personen}

Erster Schritt zur L\"osung ist das Aufsp\"uren von Akteuren, dabei gilt es Personen, 
Objekte und andere Ausl\"oser von Ereignissen zu finden. Bei den Personen ist dies
wesentlich einfacher als in anderen Bereichen, wahrscheinlich weil der reale Bezug
hier noch sehr stark vorhanden ist, da diese Personen existieren. Alle Akteure
definieren sich ebenfalls durch ihre Eigenschaften bzw. Attribute. Eine gute Quelle
f\"ur Akteure sind immer Dokumente aus den Vor-Entwicklungs-Phasen wie Produktanforderungen,
Pflichtenhefte oder Dokumente der Analyse.

Die Personen, als wichtigste Akteure, sind bereits bekannt: der {\sl Lerner}, der {\sl Lehrende},
der {\sl Administrator} und der {\sl Hospitant}. 

Im n\"achsten Schritt versucht man m\"oglichst viele F\"alle der Anwendung f\"ur die
Akteure herauszuarbeiten. Ideal f\"ur diesen Arbeitschritt ist das Anwendungsfall - Diagramm,
auch {\it Use Case Diagramm} genannt, das von der UML bereitgestellt wird. Im Diagramm
lassen sich Akteure, Anwendungsf\"alle und ihre Beziehungen zueinander in einfacher Weise 
grafisch erfassen und \"ubersichtlich darstellen.
Das UML-Tool Together erlaubt zus\"atzlich eine weitere Verlinkung von Akteuren und Anwendungsf\"allen
mit anderen Anwendungsf\"all - Diagrammen, dadurch lassen sich die Anwendungsf\"alle weiter
spezialisieren. 

\begin{figure}[h]
%\includegraphics[viewport=0 0 376 185]{systemen/grob}
\includegraphics[width=15cm]{systemen/grob}
\caption{Funktionalit\"at (erster Entwurf)}
\end{figure}


%------------------------------------------------------------------------------

\subsection{Klassen}

%------------------------------------------------------------------------------

\subsection{Protokolle}

%------------------------------------------------------------------------------
%\input{entstehung}
%\input{lehreinsatz}
%\input{realitaet}


%Nun noch Informationen f�r BiBTeX zum Erstellen des
%Literaturverzeichnis:

%Die Literaturdatenbank (auch mehrere,
%durch Kommata getrennt, sind erlaubt)
\listoftables
\addcontentsline{toc}{chapter}{Tabellenverzeichnis}
\listoffigures
\addcontentsline{toc}{chapter}{Abbildungsverzeichnis}
\bibliography{diplom}
%Das Aussehen des Literaturverzeichnisses
%wird durch das File gergipl.bst (bst f�r BibliographSTyle)
%bestimmt.
%Alternativ m�glich: "gerdiplalpha" f�r Literaturverzeichnisse,
%wo jeder Eintrag nicht eine Nummer, sondern die Anfangsbuchstaben
%der Autoren tr�gt.
\bibliographystyle{alpha}

%Eintrag ins Inhaltsverzeichnis:
\addcontentsline{toc}{chapter}{Literaturverzeichnis}

\newpage
\subsection*{} 
Hiermit erkl\"{a}re ich, die vorliegende Diplomarbeit selbstst\"{a}ndig verfasst und keine anderen als die angegebenen Quellen und Hilfsmittel verwendet zu haben.\\[5ex]
Berlin, den \today

%Schluss:
\end{document}

Hier darf nun alles m�gliche folgen...


