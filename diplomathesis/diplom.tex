\documentclass[a4paper, 11pt]{report}
\usepackage{times, mathptm}
\usepackage{ngerman}
\usepackage[latin1]{inputenc}
\usepackage{graphicx}
%\usepackage{anysize}
\sloppy

\title{{\sc Teaching By Chat\\Entwicklung eines Lehrsystems}
\\[10ex]
  Diplomarbeit \\[7ex]
  {\small zur Erlangung des akademischen Grades\\}
  Diplom-Informatiker (FH)\\[5ex]
  {\small
  an der\\
  Fachhochschule f\"{u}r Technik und Wirtschaft Berlin\\
  Fachbereich Wirtschaftswissenschaften II\\
  Studiengang Internationale Medieninformatik\\
  }
}

\author{
  1. Betreuer: Prof. Dr. Weber-Wulff\\
  2. Betreuer: Prof. Dr.-Ing. Barthel\\
  Eingereicht von Denis Renning\\[5ex]
  }

\date{Berlin, \today}

%Befehle, die den Zeilenabstand im Text, bzw. in
%Tabellen ver�ndern k�nnen.
%Fakor 1.0 ist default, also sind sie Befehle
%im Moment wirkungslos...
\renewcommand{\baselinestretch}{1.0}
\renewcommand{\arraystretch}{1.0}


\renewcommand{\textwidth}{15.0cm}

%Befehle, die Abstand und Einr�ckung eines neuen
%Paragraphen bestimmen. Der Abstand (parskip) ist dabei
%ein sogenanntes dehnbares Ma�, die Einr�ckung (parindent)
%jedoch nicht. Es werden keine absoluten Ma�e (wie cm)
%verwendet, sondern ex und em (H�he eines "x" bzw.
%Breite eines sog. Geviertstrichs "-")
\setlength{\parskip}{1.5ex plus 0.5ex minus 0.5ex}
\setlength{\parindent}{1.5em}

%\oddsidemargin 0.5in
%\evensidemargin 0.5in
%\topmargin 0.5pt
%\textheight 8.1in
%\textwidth 6in


%Hiergeht's los:
\begin{document}

\DeclareGraphicsExtensions{.gif}


%Erstellt das Titelblatt:
\maketitle

%Inhaltsverzeichnis soll r�misch numeriert sein:
%\pagenumbering{roman}\setcounter{page}{1}

%Erstellt das Inhaltsverzeichnis:
\tableofcontents

\newpage



%Der Rest soll arabisch numeriert sein:
%\pagenumbering{arabic}\setcounter{page}{1}
\chapter{Einleitung}

\anno{ca. 5 Seiten}

% das grosse Problem

% Ziel der Arbeit

% Methodik (1 Seite)

% Gliederung und Aufbau (ca. 0,5 Seiten)

%Zusammenfassung und Ausblick
%Was war besonders wichtig? Was ist offen geblieben? Wo sind Erweiterungen/Fortschreibungen m�glich oder sinnvoll? 
\chapter{Zusammenfassung und Ausblick}
%Zusammenfassung und Ausblick
%Was war besonders wichtig? Was ist offen geblieben? Wo sind Erweiterungen/Fortschreibungen m�glich oder sinnvoll? 
\chapter{Zusammenfassung und Ausblick}
%Zusammenfassung und Ausblick
%Was war besonders wichtig? Was ist offen geblieben? Wo sind Erweiterungen/Fortschreibungen m�glich oder sinnvoll? 
\chapter{Zusammenfassung und Ausblick}
%Zusammenfassung und Ausblick
%Was war besonders wichtig? Was ist offen geblieben? Wo sind Erweiterungen/Fortschreibungen m�glich oder sinnvoll? 
\chapter{Zusammenfassung und Ausblick}
%Zusammenfassung und Ausblick
%Was war besonders wichtig? Was ist offen geblieben? Wo sind Erweiterungen/Fortschreibungen m�glich oder sinnvoll? 
\chapter{Zusammenfassung und Ausblick}
%Zusammenfassung und Ausblick
%Was war besonders wichtig? Was ist offen geblieben? Wo sind Erweiterungen/Fortschreibungen m�glich oder sinnvoll? 
\chapter{Zusammenfassung und Ausblick}
%Zusammenfassung und Ausblick
%Was war besonders wichtig? Was ist offen geblieben? Wo sind Erweiterungen/Fortschreibungen m�glich oder sinnvoll? 
\chapter{Zusammenfassung und Ausblick}
%Zusammenfassung und Ausblick
%Was war besonders wichtig? Was ist offen geblieben? Wo sind Erweiterungen/Fortschreibungen m�glich oder sinnvoll? 
\chapter{Zusammenfassung und Ausblick}
%Zusammenfassung und Ausblick
%Was war besonders wichtig? Was ist offen geblieben? Wo sind Erweiterungen/Fortschreibungen m�glich oder sinnvoll? 
\chapter{Zusammenfassung und Ausblick}
%\input{entstehung}
%\input{lehreinsatz}
%\input{realitaet}


%Nun noch Informationen f�r BiBTeX zum Erstellen des
%Literaturverzeichnis:

%Die Literaturdatenbank (auch mehrere,
%durch Kommata getrennt, sind erlaubt)
\listoftables
\addcontentsline{toc}{chapter}{Tabellenverzeichnis}
\listoffigures
\addcontentsline{toc}{chapter}{Abbildungsverzeichnis}
\bibliography{diplom}
%Das Aussehen des Literaturverzeichnisses
%wird durch das File gergipl.bst (bst f�r BibliographSTyle)
%bestimmt.
%Alternativ m�glich: "gerdiplalpha" f�r Literaturverzeichnisse,
%wo jeder Eintrag nicht eine Nummer, sondern die Anfangsbuchstaben
%der Autoren tr�gt.
\bibliographystyle{alpha}

%Eintrag ins Inhaltsverzeichnis:
\addcontentsline{toc}{chapter}{Literaturverzeichnis}

\newpage
\subsection*{} 
Hiermit erkl\"{a}re ich, die vorliegende Diplomarbeit selbstst\"{a}ndig verfasst und keine anderen als die angegebenen Quellen und Hilfsmittel verwendet zu haben.\\[5ex]
Berlin, den \today

%Schluss:
\end{document}

Hier darf nun alles m�gliche folgen...


