\section{Datenmodelle}

Die Speicherung der Daten im System sollte auf Basis von XML erfolgen, da dies einen sehr
guten Kompromi{\ss} zwischen Flexibilit\"at und Effizienz bietet. Im Gegensatz zu relationellen
Datenmodellen lassen sich XML-Daten beliebig erweitern und ausbauen. Ein weiterer Vorteil von
XML ist in der Bearbeitung zu finden. XML ist ein Textformat und kann dadurch mit jedem einfachen 
Texteditor gelesen und bearbeitet werden.

\subsection{Benutzer}

Bei den Benutzerdaten ist die Flexibilit\"at von XML zwar noch nicht zwingend erforderlich aber
kommt dennoch zum Vorschein. Da das System verschiedene Nutzerrollen vorsieht bedarf es nat\"urlich
auch verschiedener Benutzerdaten. So ben\"otigt die Person des Unterrichtenden im Gegensatz zum 
Studenten keine Matrikelnummer und keine Angabe zum Semester, daf\"ur aber zum Beispiel eine Mitarbeiter
im Unternehmen bzw. \"ahnliches f\"ur staatliche Institute.

\begin{figure}[h]
\begin{verbatim}
<user id="1" password="FcjDienj" username="mmuster" role="student">
  <firstName>Max</firstName>
  <lastName>Muster</lastName>
  <email>max.muster@mail.com</email>
  <address>Steinweg 1</address>
  <city>Musterstadt</city>
  <zip>12345</zip>
  <location>Berlin</location>
  <phone>555-12345</phone>
  <matrikeln>s12345678</matrikeln>
  <semester>8</semester>
</user>
\end{verbatim}
\caption{Beispiel f\"ur Benutzerdaten}
\end{figure}

%---------------------------------------------------------------------------------------
\subsection{Klassenr\"aume}

Die Daten eines Klassenraums m\"ussen da schon wesentlich flexibler sein, da diese auch 
Referenzen zu den zugeh\"origen Personen beinhalten sollten und die Anzahl der Referenzen
nicht festgelegt sein sollte.  

\begin{figure}[h]
\begin{verbatim}
<class id="1">
  <name>Media Theory</name>
  <description>Im Bereich Medientheorie ...</description>
  <start>12/27/02</start>
  <end>05/15/05</end>
  <teacher>2</teacher>
  <lvn>1800</lvn>
  <member>1</member>
  <member>2</member>
  <member>15</member>
  <member>16</member>
</class>
\end{verbatim}
\caption{Beispiel f\"ur einen Klassenraum}
\end{figure}


%-----------------------------------------------------------------------------------------
\subsection{Protokolle}

\begin{figure}[h]
\begin{verbatim}
<protocol id="1" class="1" status="open">
  <head>
    <start>05-06-99 09:43:34</start>
    <end>05-06-99 11:44:20</end>
    <lesson>Media Theory - 6th May 1999</lesson>
    <teacher>2</teacher>
    <meta>1</meta>
  </head>
  <message id="1">
    <username>Uwe</username>
    <body>Hi! My name is Uwe and this is our chat ;)</body>
  </message>
  <message id="2">
    <username>Albert</username>
    <body>Hello</body>
  </message>
  <markup>
    <message id="3">
      <username>Albert</username>
      <body>Lets talk about the import stuff...</body>
    </message>
  </markup>
     .
     .
     .
  <message id="291">
    <username>Albert</username>
    <body>Bye bye and a nice day.</body>
  </message>
</protocol>
\end{verbatim}
\caption{Beispiel f\"ur ein Chat-Protokoll}
\end{figure}
%-----------------------------------------------------------------------------------------

\subsection{Metadaten}




\begin{figure}[h]
\begin{verbatim}
<record>
  <general>
    <title>
      <langstring lang="en">Media Theory - 6th May 1999</langstring>
    </title>
    <language>english</language>
    <description>
      <langstring lang="en">colours in media</langstring>
    </description>
    <keywords>
      <langstring lang="en">media</langstring>
      <langstring lang="en">colours</langstring>
      <langstring lang="en">colors</langstring>
      <langstring lang="de">farben</langstring>
    </keywords>
  </general>
</record>
\end{verbatim}
\caption{Beispiel f\"ur Protokoll-Metadaten}
\end{figure}
%-----------------------------------------------------------------------------------------
\section{Datenzugriffe}




