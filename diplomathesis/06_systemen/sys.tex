\section{Gesamtsystem}

Das System sollte f\"ur eine m\"oglichst gro{\ss}e Zielgruppe und von jedem Ort aus
erreichbar sein, dies sind wohl die zwei wichtigsten Aspekte, die f\"ur das Internet
sprechen. Die Zahl derer, die den Umgang mit dem Medium Internet beherrschen, steigt
t\"aglich und Internet-Browser sind mittlerweile f\"ur fast jeden Menschen beherrschbar,
also w\"are eine Web-Anwendung die ideale Wahl als Umgebung f\"ur das Lehrsystem.

Au{\ss}er der leichten Bedienung durch den Benutzer und die Erreichbarkeit sprechen
allerdings noch andere Eigenschaften einer Web-Anwendung f\"ur die Verwendung. Sie
sind f\"ur den Benutzer nicht an dessen Umgebung gebunden und laufen auf fast jedem
Rechner mit Internetanschlu{\ss}. Der Benutzer ben\"otigt keine spezielle Software.
Der Aufbau der Benutzerschnittstelle l\"a{\ss}t sich jederzeit, auch nach Fertigstellung 
der Anwendung, einfach beeinflussen und umstellen. Desweiteren lassen sich Web-Anwendungen
in andere Webseiten einfach einbetten, z.B. durch Frames. 

\subsection{Web-Applikation - aber wie?}
Zentrale zu entwickelnde Komponente ist der Chat, dessen Hauptfunktion das Chatten, nicht zwingend
eine gro{\ss}e Webanwendung fordert. Grunds\"atzlich soll der Chat nur den Austausch von Nachrichten
bewerkstelligen und das l\"asst sich auch mit Hilfe von Applets realisieren (s. \ref{sec:entwchat}).
Aus diesem Grund entschied ich mich gegen eine direkte Verkn\"upfung des Chats mit den
administrativen und informativen Funktionen des Lehrsystems.

Die Funktionen, die nicht den Chat selbst betreffen, sind lassen sich besser in der Web-Anwendung
integrieren. Als Beispiel m\"ochte ich an dieser Stelle auf die Benutzeradministration hinweisen,
da es wesentlich einfacher f\"ur den Anwender ist ein Formular auf einer Webseite auszuf\"ullen als
die Daten in einem Applet einzutragen und an die Anwendung zu schicken. Formulare lie{\ss}en
sich zwar auch innerhalb von Applets realisieren, aber dies w\"are f\"ur den Entwickler ein
erheblich gr\"osserer Aufwand und w\"urde die Gr\"o{\ss}e des Applets, und damit die Ladezeit, wesentlich 
beeinflussen.

Desweiteren wird eine Modularit\"at erm\"oglicht, die es erlaubt den Chat unabh\"angig vom
Rest der Anwendung weiterzuentwickeln und umgekehrt. Den Chat selbst, beispielsweise, interessiert es
nicht ob und welche Datenbank f\"ur die Benutzerdaten genutzt wird und sollte mal eine andere Datenbank
oder gar ein anderes Datenbanksystem genutzt werden entsteht kein Anpassungsbedarf im Chat selbst. Ebenso
ben\"otigt der Chat, au{\ss}er den Benutzernamen und die Rechte, keine weiteren Informationen \"uber einen
Teilnehmer und k\"onnte dadurch auch in anderen Umgebungen oder {\it stand-alone} eingesetzt werden.







