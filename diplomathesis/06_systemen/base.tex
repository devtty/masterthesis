\section{Basismodell}

Um eine erste Basis f\"ur das Gesamtsystem zu schaffen sollten erst die real
existierenden Bedingungen umgesetzt werden, um die in sp\"ateren Phasen das System
aufgebaut wird. Anwendungen, egal ob Software oder in anderen Bereichen, sollten
sich immer den Bed\"urfnissen des Benutzers anpassen, dazu mu{\ss} dem sp\"ateren
System der Benutzer auch bekannt sein. Da Softwareentwicklung auch viel mit der realen
Welt zu tun hat ,und eigentlich durch Abstraktion dieser erst entsteht, wird der
Benutzer, mit den Eigenschaften die f\"ur das Lehrsystem als wichtig erachtet werden,
als Erster entworfen bzw. von der real existierenden Welt \"ubernommen.

%------------------------------------------------------------------------------
\subsection{Teilnehmende Personen}

Erster Schritt zur L\"osung ist das Aufsp\"uren von Akteuren, dabei gilt es Personen, 
Objekte und andere Ausl\"oser von Ereignissen zu finden. Bei den Personen ist dies
wesentlich einfacher als in anderen Bereichen, wahrscheinlich weil der reale Bezug
hier noch sehr stark vorhanden ist, da diese Personen existieren. Alle Akteure
definieren sich ebenfalls durch ihre Eigenschaften bzw. Attribute. Eine gute Quelle
f\"ur Akteure sind immer Dokumente aus den Vor-Entwicklungs-Phasen wie Produktanforderungen,
Pflichtenhefte oder Dokumente der Analyse.

Die Personen, als wichtigste Akteure, sind bereits bekannt: der {\sl Lerner}, der {\sl Lehrende},
der {\sl Administrator} und der {\sl Hospitant}. 

Im n\"achsten Schritt versucht man m\"oglichst viele F\"alle der Anwendung f\"ur die
Akteure herauszuarbeiten. Ideal f\"ur diesen Arbeitschritt ist das Anwendungsfall - Diagramm,
auch {\it Use Case Diagramm} genannt, das von der UML bereitgestellt wird. Im Diagramm
lassen sich Akteure, Anwendungsf\"alle und ihre Beziehungen zueinander in einfacher Weise 
grafisch erfassen und \"ubersichtlich darstellen.
Das UML-Tool Together erlaubt zus\"atzlich eine weitere Verlinkung von Akteuren und Anwendungsf\"allen
mit anderen Anwendungsf\"all - Diagrammen, dadurch lassen sich die Anwendungsf\"alle weiter
spezialisieren. 

\begin{figure}[h]
%\includegraphics[viewport=0 0 376 185]{systemen/grob}
\includegraphics[width=15cm]{systemen/grob}
\caption{Funktionalit\"at (erster Entwurf)}
\end{figure}


%------------------------------------------------------------------------------

\subsection{Klassen}

%------------------------------------------------------------------------------

\subsection{Protokolle}

%------------------------------------------------------------------------------