\section{Jakarta Struts}

Um die Arbeit mit dem zweiten MVC Modell zu erleichtern entstand innerhalb des Jakarta Projekts
der Apache Software Foundation ein einfach zuhandhabendes Framework zur Entwicklung von 
Web-Applikationen das auf den Namen Struts getauft wurde. Struts liefert die Grundlagen f�r
die Entwicklung einer Web-Anwendung sowohl f�r den Anwendungsentwickler als auch f�r den 
HTML-Programmierer der f�r die Ansicht zust�ndig ist.


\subsection{Kontrolle und Steuerung}

Eine der wichtigsten Aufgaben im Framework erf�llen die Klassen des Pakets {\tt org.apache.struts.action},
insbesondere die Klassen {\tt Action},{\tt ActionForm} und {\tt ActionServlet}, welche zur Entwicklung 
der Steuerungsschicht beitragen und diese von den anderen Schichten abgrenzen.

\subsubsection{Die Klasse {\tt Action}}

Die Klasse {\tt Action} ist eine Adapterklasse die s�mtliche Aktionen, die in einer Struts-basierenden
Anwendung auftreten k�nnen, in das Framework integriert. Solche Aktionen k�nnen beispielsweise das
Klicken auf einen Link, das Abschicken eines Formulars oder auch Aktionen die von Struts selbst 
ausgel�st werden sein.

Jede Aktion mu� von der Klasse {\tt Action} bzw. eine ihrer Unterklassen abgeleitet werden und die
Methode {\tt execute} �berschreiben, damit die Aktion implementiert ist.


\subsubsection{Die Klasse {\tt ActionForm}}

Von {\tt ActionForm} abgeleitete Klasse definieren in der Anwendung vorkommende Formulare beispielsweise
das Anmelde-Formular. Diese Klassen besitzen den typischen Aufbau einer JavaBean, da der Entwickler f�r
jedes verwendete Formularfeld ein Attribut inklusive der n�tigen {\tt get} und {\tt set}-Methode in der 
Klasse reservieren mu�. Zus�tzlich kann die Methode {\tt validate} �berschrieben werden um eine 
automatische �berpr�fung der Formulardaten zu gew�hrleisten.

\subsubsection{Das {\tt ActionServlet}}

Die zentrale Komponente jeder Struts-basierenden Anwendung ist das ActionServlet, da es jede Aktion in
der Anwendung steuert und koordiniert. Sendet der Benutzer eine Anfrage an die Anwendung so nimmt das
ActionServlet diese erst entgegen und leitet sie an die entsprechende Aktion weiter. Der Entwickler mu�
dazu nur die entsprechenden Aktionen in der Konfigurationsdatei {\tt struts-config.xml} eintragen.
