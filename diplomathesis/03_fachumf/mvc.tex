\section{Das MVC-Modell}

Eine wesentliche Erleichterung bei der Entwicklung gr��erer Projekte bietet die Aufteilung
der Anwendung in verschiedene Bereiche. Insbesondere die Aufteilung nach Modell, Ansicht und
Steuerung hat sich als sinnvoll in der objekt-orientierten Programmierung erwiesen. Die Idee
dazu wurde erstmals in der Sprache Smalltalk verwirklicht und stammt aus den Jahren 1978/79
urspr�nglich von Prof. Trygve Reenskaug \cite[S.~495]{ullenboom_00}. Von Bedeutung wurde
dieses Modell f�r Java vor allem im Zusammenhang mit der Programmierung von Applikationen
mit grafischer Benutzeroberfl�che.

Der Bekanntheitsgrad von Java bei der Erstellung von Anwendungen f�r das Internet stieg mit
der Einf�hrung der Servlet-Technologie. Bis zu diesem Zeitpunkt waren haupts�chlich CGI-Skripte\footnotemark[1] 
f�r die Verarbeitung von Informationen verantwortlich. Die Servlets brachten vor allem Vorteile
in der Geschwindigkeit bei der Ausf�hrung und erm�glichten Web-Anwendungen die Nutzung von Java
in vollem Umfang. Da ein Servlet eine komplette Java-Klasse ist und vor der Nutzung wie jede andere
Klasse erst kompiliert werden mu�te hatte die Servlet-Technologie der damaligen Zeit einen wesentlichen
Nachteil: die Ansicht bzw. die sich ergebenden HTML-Seiten mu�ten komplett im Quelltext der Klasse 
generiert werden und das Layout der Webseiten konnte nach der Kompilierung nicht mehr ge�ndert werden.

Kurze Zeit sp�ter entwickelte Sun die JavaServerPages die es dem Entwickler erm�glichten den 
umgekehrten Weg zu gehen. Mit JSP war der Programmierer in der Lage seinen Java-Code innerhalb
von HTML-Seiten einzubetten bzw. andere Klassen (Beans) aus den Seiten aufzurufen. Diese JSP-Seiten
werden dann bei Bedarf vom Server in Servlets umgewandelt und kompiliert. Das Modell
und die Steuerung wurde innerhalb der Beans gekapselt, die Ansicht in der JSP-Seite. Doch auch
diese erste Umsetzung des MVC-Konzepts f�r Webapplikationen in Java  erzeugte neuen �rger, denn
gerade bei gr��eren Projekten h�uften sich Unmengen von Java-Code in den JSP-Seiten an, sehr zum
�rger der HTML-Programmierer.

Um die Probleme beider Ans�tze zu l�sen k�ndigte Sun Microsystems im Juni 1999 das Application 
Programming Model an, welches auch unter dem Namen J2EE Blueprints bekannt ist\cite[S.~284]{bergsten_01}.
Dies beinhaltete unter anderem auch die Spezifikation f�r das MVC Modell 2, welches die Vorteile
der Servlets mit den Vorteilen von JavaServerPages verband.

Das MVC Modell 2 lagert die Steuerung in ein Servlet aus, wobei das Datenmodell in den
JavaBeans bleibt und die Darstellung weiterhin �ber JSP-Seiten erfolgt. Diese Steuer-Servlets,
auch Controller genannt,  sind f�r den Programmierer wesentlich einfacher zu warten und die
Ansicht kann jederzeit unabh�ngig von der Anwendungslogik ver�ndert werden.



\footnotetext[1]{CGI - Common Gateway Interface}






