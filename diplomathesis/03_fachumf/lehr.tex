\section{Didaktisches Umfeld}
Zur Entwicklung eines Lehrsystems sind auch einige Aspekte des didaktischen Bereichs zu 
bedenken. Bei der Entwicklung eines Chats, mit der Absicht das dieser den Unterricht unterst\"utzen
oder eventuell auch ersetzen k\"onnte, hat der Entwickler sogar die Pflicht sich wenigstens einen
guten \"Uberblick \"uber das Thema zu verschaffen und Unterschiede bzw. Gemeinsamkeiten zu erkennen.

%--------------------------------------------------------------------------------------------------
\subsection{Die virtuelle Welt}
Pr\"asenzgebundener Unterricht erlaubt es, sowohl dem Lerner als auch der Lehrkraft, sein Gegen\"uber
auf andere Weise zu begegnen und kennenzulernen. Im Fall das der Unterricht ausschlie{\ss}lich in
der virtuellen Welt stattfindet besteht kein realer Bezug zwischen den teilnehmenden Personen, dies
k\"onnte zum Beispiel das Wegfallen der Autorit\"at des Unterrichtenden bewirken aber auch Teilnahmslosigkeit
der Lernenden. 

Nat\"urlich kann die virtuelle, multimediale Welt in Ma{\ss}en den Pr\"asenzunterricht wesentlich
aufwerten aber in den meisten F\"allen hat sie nur einen ``Werkzeugcharakter''. Das liegt zum Teil
am Menschen selbst, denn im Lehrbereich herrscht gr\"o{\ss}tenteils noch traditionelles Denken und es
fehlen h\"aufig Erfahrungen mit dem Medium Internet, aber auch an der Art wie multimediale Lehre
angeboten wird. Eine Erhebung der Bertelsmann-Nixdorf Stiftung zum Thema ``Virtuelles Lehren und 
Lernen an deutschen Hochschulen'' \cite[S.~45]{schulmeister_01} ergab folgendes Angebot an Projekten
zur (vorlesungsbegleitenden) Lehre:

\begin{table}[h]
\begin{tabular}{l|r}
Typ der Mediennutzung\\
\hline\\
Vorlesungsbegleitendes Skript & 39\\
Multimedia-CD                 & 35\\
Internet-Nutzung (pur)        & 10\\
Teleteaching                  & 8\\
Internet-Nutzung gest\"utzt durch Interaktion & 4\\
Lernen mit Werkzeugen & 4\\
Pr\"asentationen & 2\\
\end{tabular}
\caption{Typen der Mediennutzung (Auszug aus \cite[S.~46]{schulmeister_01})}
\label{tab:mediennutzung}
\end{table}

Dies zeigt deutlich das das ``Werkzeug'' gr\"o{\ss}tenteils auch wie ein Werkzeug genutzt wird. Der 
Lernende wird in seinem Lernproze{\ss} durch ein Skript bzw. eine Multimedia-CD, in die man mal 
reinschaut, unterst\"utzt und ist sich \"uber den ``Wert und die Unersetzbarkeit des sozialen Lernens 
in realen Gruppen'' bewu{\ss}t \cite[S.~50]{schulmeister_01}. Das auch in der virtuellen Welt soziale
Begegnungen und Beziehungen, \"ahnlich denen in realen Gruppen, stattfinden ist bewiesen durch
die zahlreiche Existenz von Communities und E-Mail-Freundschaften. Eine solche soziale Begegnung
beschreibt Derek Powazek in seinem Buch ``Design for Community''\footnotemark[1] folgenderma{\ss}en:
\footnotetext[1]{\cite[S.~136]{powazek_02}}

\begin{quotation}
This random, disembodied, virtual exchange had changed from a splash of ASCII across a telnet window
into something personal, sincere ...  It's the moment when you realize that computers aren't just
for homework and business...
\end{quotation}

Wenn also soziale Gruppen in der virtuellen Welt existieren k\"onnen, warum dann nicht auch f\"ur den 
Bereich der Lehre? Das nicht gen\"ugend Gemeinsamkeiten zwischen beiden Bereichen existieren, w�re eine
m\"ogliche Antwort. Es m\"ussen also Gemeinsamkeiten gefunden bzw. neue Gemeinsamkeiten geschaffen werden.
Doch wodurch zeichnet sich der Unterricht aus? Was ist Unterricht?

%--------------------------------------------------------------------------------------------------
\subsection{Kommunikation}
Peter Hubwieser interpretiert das Wesen des Unterrichts in seinem Buch ``Didaktik der Informatik''\footnotemark[1]:
\footnotetext[1]{\cite[S.~29]{hubwieser_01}}
\begin{quotation}
Unterricht ist ein hochkomplexer Prozess mit zahlreichen, auch zyklischen Wechselwirkungen, bei dem Lehrende
und Lernende unter gewissen gesellschaftlichen, b\"urokratischen und materiellen Vorgaben und Rahmenbedingungen
im Hinblick auf eine bestimmte Zielsetzung interagieren. Langfristig hat dieser Prozess wiederum
Auswirkungen auf die gesamte Gesellschaft und die von ihr formulierten Vorgaben.
\end{quotation}
``Unterricht ist ein hochkomplexer Prozess... bei dem Lehrende und Lernende... interagieren.'' Das
Kommunikation, als interaktiver Prozess, wesentlicher Bestandteil des Unterrichts sein sollte begr\"undet
er durch ``wissenschaftliche und industrielle T\"atigkeiten (die) mittlerweile \"uberwiegend im
Team durchgef\"uhrt werden'' \cite[S.~37]{hubwieser_01} und auf den Lernenden zukommen\footnotemark[2].
\footnotetext[2]{Anm.: Hubwiesers Ausgangspunkt ist der Pflichtunterricht an \"offentlichen Schulen}
Insbesondere bezeichnet er Gruppenarbeiten und Unterrichtsgespr\"ache als kommunikationsunterst\"utzende
Sozialformen. Die Sozialformen unterscheiden \"au{\ss}ere Formen von Kommunikation und Interaktion zwischen Lehrenden
und Lernenden. Im Klassenverband kann dadurch zum Beispiel zwischen lehrerzentriertem und sch\"ulerzentriertem
Unterricht unterschieden werden und der Lehrende hat die M\"oglichkeit die Sozialform w\"ahrend des Unterrichts
zu wechseln, in dem er beispielsweise nach l\"angerer Redezeit (lehrerzentriert) das Wort an einen Sch\"uler
weitergibt (sch\"ulerzentriert).

Wirft man nun wieder einen Blick auf Tabelle \ref{tab:mediennutzung} stellt man fest das einige Sozialformen in der 
virtuellen Welt nicht oder nur teilweise m\"oglich sind. Die freie Wahl der Sozialform w\"ahrend des Unterrichts 
ist nur bei wenigen der aufgez\"ahlten Projekte eventuell m\"oglich, bei den beiden Spitzenreitern, dem Begleitskript
und der Multimedia-CD, ist Kommunikation nicht m\"oglich. 


%--------------------------------------------------------------------------------------------------
\subsection{F\"uhrungsstile}
\begin{table}[h]
\begin{tabular}{lll}
F\"uhrungsstil & Lehrerverhalten                         & Auswirkungen \\
%-----------------------------------------------------------------------------------------------
\hline         &                                         &                       \\
autorit\"ar    & Die Lehrkraft                           & gr\"o{\ss}ere Leistungsquantit\"at,\\
               & - legt alle Richtlinien fest,           & geringere Qualit\"at, \\
               & - schreibt Techniken/T\"atigkeiten vor, & geringere Arbeitsmoral\\
               & - lobt und tadelt nach pers\"onlichen,  & mehr Konflikte und \\
               & \hspace{1ex}Gesichtspunkten,            & Aggressionen\\
               & - h\"alt sich abseits von der Gruppe.   & kaum Arbeitseinsatz bei\\
               &                                         & Abwesenheit des Lehrers\\
\hline         &                                         &                        \\
demokratisch   & Die Lehrkraft                           & h\"ohere Arbeitsmoral\\
               & - l\"asst Richtlinien nach Diskussion   & weniger Aggressionen\\
               & \hspace{1ex}entscheiden,                & besserer Arbeitseinsatz bei\\
               & - hilft beim Entscheiden,               & Abwesenheit des Lehrers\\
               & - schl\"agt evtl. Alternativen vor,     & geringere Produktionsmenge\\
               & - orientiert sich bei der Bewertung     & h\"ohere Qualit\"at\\
               & \hspace{1ex}an objektiven Kriterien,    &                    \\
               & - versucht, Mitglied der Gruppe zu sein.&                    \\
\hline         &                                         &                    \\
laissez-faire  & Die Lehrkraft                           & geringere Arbeits- \\
               & -\"uberl\"asst alle Entscheidungen den  & und Gruppenmoral\\
               & \hspace{1ex}Sch\"ulern,                 & geringe Produktivit\"at\\
               & - beschafft lediglich Material,         &                    \\
               & - gibt Informationen nur auf Befragen,  &                    \\
               & - zeigt keine Teilnahme, keine          &                    \\
               & \hspace{1ex}Beurteilung,                &                    \\
               & - verzichtet auf jede Regelung.         &                    \\
\hline
nach \cite[S.~38]{baumann_00}
\end{tabular}
\caption{F\"uhrungsstile nach Lewin(1953)}
\label{tab:lewinstile}
\end{table}




