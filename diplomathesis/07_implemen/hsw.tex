\section{Entwicklungs-Werkzeuge}

\subsection{CVS}
Aus Zeit- und Qualit\"atsaspekten ist es bei jedem Projekt von Vorteil den Verlauf der Arbeit
jederzeit zur\"uckverfolgen zu k\"onnen oder in verschiedene Richtungen zu lenken. Eine h\"aufig
praktizierte aber denkbar schlechte L\"osung daf\"ur ist das Speichern verschiedener Versionen
in verschiedenen Dateien, da dies einen enormen Bedarf an Speicherplatz ben\"otigt und 
zus\"atzlich einer eigenverantwortlichen Katalogisierung bedarf. Sehr gerne wird hierbei das
jeweilige Datum an den Dateinamen angeh\"angt. 

Eine bessere Methode ist die Nutzung eines Systems zur Versionskontrolle. Von zahlreichen
auf dem Markt bekannten Systemen ist CVS eines der bekannten und frei erh\"altlichen
Systeme. Die grunds\"atzliche Funktionsweise von CVS ist dabei denkbar einfach, da es alle 
Versionen einer Datei in einer einzigen Datei speichert und dabei nur die Unterschiede 
zwischen verschiedenen Versionen abspeichert. Dadurch ist man in der Lage jegliche 
vorhergehende Version zu jedem Zeitpunkt wiederherzustellen. Erste Schritte machte CVS
1986, seitdem entwickelte sich das System immer weiter und bietet mittlerweile zahlreiche
Zusatzfunktionen. Beipielsweise lassen sich auf einfache Weise Unterschiede zwischen
verschiedenen Versionen darstellen und damit beispielsweise eventuelle Fehler, die
in fr\"uheren Versionen nicht auftraten, leichter feststellen. Ebenso erm\"oglicht CVS die
Arbeit mehrerer Personen an einer Datei zur selben Zeit.

Bei der Entwicklung des Lehrsystems war CVS inbesondere im Entwurf und der Implementierung
besonders hilfreich.

%---------------------------------------------------------------------------------------
\subsection{Ant}

Eine der zeitaufw\"andigen Aufgaben der Implementation ist das Kompilieren der Quellen und das
sich anschlie{\ss}ende Einspielen der fertigen Klassen auf das Dateisystem des Servers (Deployment).
Abhilfe dabei schaffen Build-Tools wie das ebenfalls vom Jakarta Projekt stammende Ant. In einer
Konfigurations-Datei (build.xml) werden alle n\"otigen Schritte die zu einer fertigen Anwendung
f\"uhren eingestellt und bei Bedarf ausgef\"uhrt. Ant \"ubernimmt dabei sowohl das Kompilieren, das
Erstellen von ben\"otigten Verzeichnissen und das Kopieren auf den Server.

%---------------------------------------------------------------------------------------
\subsection{Together Control Center}

Die von der TogetherSoft Corporation entwickelte Model-Build-Deploy Platform in der Version 5.5
war das meistgenutzte Werkzeug bei der Entwicklung des Lehrsystems, teilweise auch dadurch
das es als Frontend f\"ur CVS und Ant diente. Beide Werkzeuge lassen sich einfach und komfortabel
mit dem Control Center bedienen und aufrufen.

Die Hauptaufgabe des Werkzeugs lag allerdings in der Entwicklung und Dokumentation. Das Werkzeug
erm\"oglicht die einfache Erstellung von UML-Diagrammen nach dem Standard UML 1.3. Die erstellten
Diagramme lassen sich leicht beschreiben und im Falle von einigen Diagrammen wie zum Beispiel
Klassendiagrammen generiert das Together Control Center auf Wunsch die n\"otigen Quelltexte f\"ur
die Implementation. Auf Knopfdruck kann ebenfalls eine Dokumentation in verschiedenen Formaten
generiert werden oder der Quelltext nach bestimmten Qualit\"atsmerkmalen formatiert werden. 
Desweiteren stellt das Programm eine gro{\ss}e Bibliothek an Entwurfsmustern f\"ur die Entwicklung
zur Verf\"ugung.

%--------------------------------------------------------------------------------------
%--------------------------------------------------------------------------------------
\section{Anwendungs-Komponenten}
In diesem Abschnitt m\"ochte ich etwas n\"aher auf die Komponenten eingehen, die zur
Entwicklung, zum Test und zum Betrieb des Lehrsystems ben\"otigt werden bzw. wurden.
Dabei wird zwischen Hardware und Software unterschieden. Insbesondere wird
auf die verwendete Programmiersprache, den Webserver und die Datenbank eingangen, die
zum Betrieb des Lehrsystems in der derzeitigen Version verwendet werden sollten. 

Weiterhin wird noch kurz auf die verwendete Hardware eingegangen, da das System jedoch weitestgehend
plattformunabh\"angig ist sollte es auch in anderen Hardwareumgebungen lauff\"ahig sein. Dies
wurde jedoch nicht getestet.

\newpage
\subsection{Java}
Eine der ersten Entscheidungen bei der Entwicklung von Software sind die Wahl der passenden
Produktumgebung und die damit h\"aufig verbundene Wahl der Programmiersprache, die f\"ur
die Kernfunktionen des Systems zust\"andig sein soll. Das das Internet die passende Produktumgebung f\"ur
das Lehrsystem ist, wurde bereits durch die Art der Anwendung festgelegt und schaut man sich nun nach
geeigneten Programmiersprachen f\"ur das Internet um, so wird man von einer Flut von M\"oglichkeiten
\"uberschwemmt. \"Uber die CGI-Schnittstelle stehen alle Programmiersprachen zur Verf\"ugung, die
eine Ausgabe an die Standardausgabe eines Computers unterst\"utzen. Dadurch k\"onnen sowohl
interpretierte Sprachen, wie z.B. Perl oder Shell-Skripte, aber auch Hochsprachen wie C oder Pascal
f\"ur eine Web-Anwendung in Frage kommen. Letztendlich hat der CGI-Ansatz aber auch ein paar Nachteile,
bei der Verwendung von interpretierten Sprachen mu{\ss} das Betriebsystem nat\"urlich auch in der
Lage sein die Skripte zu interpretieren, ein einfacher Wechsel des Betriebssystems ist so mit gr\"o{\ss}erem
Aufwand verbunden. Im Fall der Verwendung von Hochsprachen wie C m\"u{\ss}te die gesamte Anwendung sogar
f\"ur die jeweilige Plattform neu kompiliert und gelinkt werden. Ein weiterer Nachteil des CGI liegt
in der Art wie Anfragen an die Anwendung bearbeitet werden, jede Anfrage kann bei einigen Plattformen (z.B.
Apache auf Windows) auf dem Server einen neuen Prozess erzeugen, bei sehr vielen Anfragen in einem 
kurzen Zeitraum ben\"otigt der Server dadurch enorm viel Rechen- und Speicherleistung.

Wesentlich komfortableres Arbeiten sollte die Programmiersprache Java erm\"oglichen, wenn man die
folgende Umfrage auf einer Fach-Konferenz\footnotemark[1] betrachtet:
\footnotetext[1]{ACM SIGCSE-Konferenz; Tabelle aus  \cite[S.~23]{ullenboom_00}}

\begin{table}[h]
\begin{tabular}{lr}
\hline
Lobenswerte Eigenschaft von Java                               & \%\\
\hline\\
Programme sind im Netz ladbar                                  & 51\\
Java ist plattformunabh\"angig                                 & 43\\
Ist sicherer als C++                                           & 21\\
Die Sprache ist einfach ``In''                                 & 16\\
Entfernt unsichere Eigenschaften von C++                       & 11\\
Compiler und VM von Sun frei                                   & 11\\
Erlaubt die Erstellung von grafischen Benutzungsschnittstellen & 9\\
Allgemeine Ablehnung gegen\"uber C++                           & 9\\
Ist ein richtiges Produkt zur richtigen Zeit                   & 9\\
Sprache mit durchdachten Elementen                             & 8\\
Qualit\"at der Bibliotheken                                    & 7\\
Elegante Speicherverwaltung mit Garbage-Collector              & 5\\
Applikations- und Appletf\"ahigkeit                            & 4\\
Unterst\"utzt Threads                                          & 3\\
Erm\"oglicht verteiltes Rechnen                                & 3\\
Exception-Verwaltung                                           & 0\\
\end{tabular}
\caption{Welche Eigenschaften an Java wie gesch\"atzt werden.}
\end{table}

Punkt zwei: ``Java ist plattformunabh\"angig'' - meiner Meinung nach, eines der wichtigsten
Argumente f\"ur die Nutzung von Java. Wenn man eine Anwendung schreibt ist es immer von Vorteil
wenn sie auf m\"oglichst vielen unterschiedlichen Plattformen ohne gro{\ss}e Umst\"ande verwendbar
ist. Java wird zwar auch, wie C und C++, kompiliert aber umgeht das Problem der Plattformabh\"angigkeit
durch die Erstellung von sogenanntem {\it Bytecode}. Im Gegensatz zu den Ergebnissen das Compilers
von C oder Pascal ist der Bytecode nicht an ein Betriebssystem oder eine spezielle Systemarchitektur
gebunden sondern nur an eine bestimmte {\it Virtuelle Maschine}, die den Bytecode f\"ur das 
jeweilige Betriebssystem interpretiert. Die Virtuelle Maschinen (VM) sind f\"ur eine Vielzahl
von Betriebssystemen bereits vorhanden und kostenlos verf\"ugbar.

Die Unabh\"angigkeit von der Plattform, die {\it Appletf\"ahigkeit} und die M\"oglichkeit {\it Benutzungsschnittstellen}
zu erstellen bieten eine ideale Vorraussetzung f\"ur einen Chat-Client, da Applets ``kleine'' Applikationen
sind, die in vielen Internet-Browsern laufen. Zahlreich existieren bereits Chat-Clients auf Basis von Java Applets
im Internet. 

Das die Punkte ``Qualit\"at der Bibliotheken'' und ``Unterst\"utzung von Threads'' weit unten in der
Umfrage landeten liegt h\"ochstwahrscheinlich daran, das die Umfrage im M\"arz 1997 stattfand
und das erste {\it Java Developement Kit} im Januar 1996 freigegeben wurde. Jedoch hat sich seitdem
vieles getan, gerade durch die beiden angesprochenen Punkte, und ``als Sun die Technik der Java Servlets
der Welt\"offentlichkeit vorstellte, waren viele Programmierer hoch erfreut, weil diese schneller und
m\"achtiger war als der bis dahin vorherrschende CGI-Ansatz'' \cite[S.~48]{oeztuerk_02}. Die Unterst\"utzung
von Threads als einer der Gr\"unde f\"ur den Geschwindigkeitsvorteil liegt auf der Hand, es ist nicht mehr n\"otig
f\"ur einzelne Anfragen einen eigenen Prozess zu starten, stattdessen k\"onnen mehrere Anfragen von ein
und demselben Prozess, innerhalb verschiedener Threads, bearbeitet werden. Mit der Servlet-Technologie
entstanden auch die {\it Servlet-Engines}, Container-Anwendungen die f\"ur die Ausf\"uhrung von
Servlets, und weit mehr, verantwortlich sind. 


%--------------------------------------------------------------------------------------
\subsection{\label{sec:webserver}Web-Server}
Grundvoraussetzung f\"ur jede Web-Anwendung ist ein Web-Server mit Servlet-Engine, die sowohl Servlets
als auch Java Server Pages unterst\"utzt. Als Referenzimplementierung f\"ur solche Servlet-Engines
gilt der Tomcat-Server des Jakarta Projekts\footnotemark[1]\footnotetext[1]{http://jakarta.apache.org}.

%--------------------------------------------------------------------------------------
\subsection{Tamino XML-Server}
Der Tamino XML-Server ist ein Produkt der Software AG und, wie der Name schon sagt, ebenfalls ein Server.
Die Hauptaufgabe dieses Servers ist jedoch nicht das Bereitstellen von Internetseiten. Die Definition der 
Software AG - T{\it ransaktions-}A{\it rchitektur f\"ur das}M{\it anagement von} IN{\it ternet-}O{\it bjekten}
- zeigt bereits in groben Z\"ugen den Verwendungszweck von Tamino. Transaktions-Architektur bedeutet das
Tamino Transaktionen erm\"oglicht, wie zum Beispiel die Bearbeitung, Speicherung aber auch den Abruf von 
Internet-Objekten. Unter Internet-Objekten verstehen sich dabei Daten wie HTML-, XML- aber auch Bin\"ardaten, 
wie z.B. Bilder. In der Tat lassen sich in Tamino auch Bilddaten und alle anderen Bin\"ardaten speichern
und lesen, allerdings liegen die Hauptgr\"unde f\"ur die Verwendung in der F\"ahigkeit XML-Daten in ihrer
nativen bzw. nat\"urlichen Form, als XML-Daten, abzuspeichern. Im Gegensatz zu den vorherrschenden relationellen
Datenbanksystemen bildet Tamino XML-Daten nicht auf Tabellen ab oder speichert sie komplett in einem Textfeld.

\begin{table}[h]
\begin{tabular}{|l|l|}
\hline
XML & RDBMS (normalized)\\&\\
- Data in single hierarchical structure      & - Data in multiple tables\\
- Nodes have element and/or attribute values & - Cells have a single value\\
- Elements can be nested                     & - Atomic cell values\\
- Elements are ordered                       & - Row/column order not defined\\
- Elements can be recursive                  & - Little support for recursive elements\\
- Schema optional                            & - Schema required\\
- Direct storage/retrieval of XML Documents  & - Joins often necessary to retrieve XML Documents\\
- Query with XML standards                   & - Query with SQL retrofitted for XML\\
\hline
\end{tabular}
\caption{Hauptunterschiede zwischen XML-Daten und relationalen DBMS \cite[S.~9]{tamino_01}}
\end{table}

%--------------------------------------------------------------------------------------
%--------------------------------------------------------------------------------------
\newpage
\section{Verwendete Hardware}

\subsection{Der Server}

Als allgemeiner Server f\"ur die Entwicklung stand eine Siemens SCENIC 400 mit einem Intel Pentium 4 
Prozessor zur Verf\"ugung. 





